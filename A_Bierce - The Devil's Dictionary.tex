\documentclass[11pt]{article}

\usepackage{a4wide}

\begin{document}

%--------------- title page

\begin{titlepage}

\newlength{\centeroffset}
\setlength{\centeroffset}{-0.5\oddsidemargin}
\addtolength{\centeroffset}{0.5\evensidemargin}
%\addtolength{\textwidth}{-\centeroffset}
\thispagestyle{empty}
%\vspace*{\stretch{1}}
\vspace*{3 cm}
\noindent\hspace*{\centeroffset}\begin{minipage}{\textwidth}
\flushright
{\Huge\bfseries The Devil's Dictionary
}
\noindent\rule[-1ex]{\textwidth}{5pt}\\[2.5ex]
\hfill
{\Large Ambrose Bierce}
\end{minipage}


\end{titlepage}

\newpage

%---------------- typographer's comment

\pagenumbering{roman}

\section*{Typographer's comments}

\paragraph{}
This book was downloaded as a plain text file having only "ascii-art"-like
formatting. This was however enough so that a PERL script could do most
typeset for use with \LaTeX.
After initial preprocessing with PERL, the typeset was manually re-read and
corrected (since script cannot distinguish between verses of a poem and
prose).

\paragraph{}
Note that the typeset is in no way complete, there are still many things to
be done better (you may notice occasional {\em orphan} and {\em widow} lines).
I've tried to keep as close as possible to the formatting of the
original text file, which, I guess, was copied off the actual book. As far
as I know, it resembles the original book (at least my copy). You should have
also received the \LaTeX~sources, so you can make the typeset better. If you
have not, search the Internet, it will lie around there somewhere.


\newpage

%------------------ preface


\vspace*{7 cm}

\section*{Author's preface}

\paragraph{}
{\em The Devil's Dictionary} was begun in a weekly paper in 1881, and was
continued in a desultory way at long intervals until 1906.  In that
year a large part of it was published in covers with the title {\em The
Cynic's Word Book}, a name which the author had not the power to
reject or happiness to approve.  To quote the publishers of the
present work:

\paragraph{}
"This more reverent title had previously been forced upon him by
the religious scruples of the last newspaper in which a part of the
work had appeared, with the natural consequence that when it came out
in covers the country already had been flooded by its imitators with a
score of 'cynic' books -- {\em The Cynic's This}, {\em The Cynic's That}, and
{\em The Cynic's t'Other}.  Most of these books were merely stupid, though
some of them added the distinction of silliness.  Among them, they
brought the word 'cynic' into disfavor so deep that any book bearing
it was discredited in advance of publication."

\paragraph{}
Meantime, too, some of the enterprising humorists of the country
had helped themselves to such parts of the work as served their needs,
and many of its definitions, anecdotes, phrases and so forth, had
become more or less current in popular speech.  This explanation is
made, not with any pride of priority in trifles, but in simple denial
of possible charges of plagiarism, which is no trifle.  In merely
resuming his own the author hopes to be held guiltless by those to
whom the work is addressed -- enlightened souls who prefer dry wines
to sweet, sense to sentiment, wit to humor and clean English to slang.

\paragraph{}
A conspicuous, and it is hoped not unpleasant, feature of the book
is its abundant illustrative quotations from eminent poets, chief of
whom is that learned and ingenius cleric, Father Gassalasca Jape,
S.J., whose lines bear his initials.  To Father Jape's kindly
encouragement and assistance the author of the prose text is greatly
indebted.

\paragraph{}
A.B.


\newpage

%--------------- text of the book itself

\pagenumbering{arabic}

\section*{A}

\paragraph{ABASEMENT, n.}  A decent and customary mental attitude in the presence
of wealth of power.  Peculiarly appropriate in an employee when
addressing an employer.

\paragraph{ABATIS, n.}  Rubbish in front of a fort, to prevent the rubbish outside
from molesting the rubbish inside.

\paragraph{ABDICATION, n.}  An act whereby a sovereign attests his sense of the
high temperature of the throne.

\begin{quote}   Poor Isabella's Dead, whose abdication \\
  Set all tongues wagging in the Spanish nation. \\
  For that performance 'twere unfair to scold her: \\
  She wisely left a throne too hot to hold her. \\
  To History she'll be no royal riddle -- \\
  Merely a plain parched pea that jumped the griddle. \\
 \\
G.J. \end{quote}


\paragraph{ABDOMEN, n.}  The temple of the god Stomach, in whose worship, with
sacrificial rights, all true men engage.  From women this ancient
faith commands but a stammering assent.  They sometimes minister at
the altar in a half-hearted and ineffective way, but true reverence
for the one deity that men really adore they know not.  If woman had a
free hand in the world's marketing the race would become
graminivorous.

\paragraph{ABILITY, n.}  The natural equipment to accomplish some small part of
the meaner ambitions distinguishing able men from dead ones.  In the
last analysis ability is commonly found to consist mainly in a high
degree of solemnity.  Perhaps, however, this impressive quality is
rightly appraised; it is no easy task to be solemn.

\paragraph{ABNORMAL, adj.}  Not conforming to standard.  In matters of thought and
conduct, to be independent is to be abnormal, to be abnormal is to be
detested.  Wherefore the lexicographer adviseth a striving toward the
straiter [sic] resemblance of the Average Man than he hath to himself.
Whoso attaineth thereto shall have peace, the prospect of death and
the hope of Hell.

\paragraph{ABORIGINIES, n.}  Persons of little worth found cumbering the soil of a
newly discovered country.  They soon cease to cumber; they fertilize.

\paragraph{ABRACADABRA.}

\begin{quote}   By {\em Abracadabra} we signify \\
      An infinite number of things. \\
  'Tis the answer to What? and How? and Why? \\
  And Whence? and Whither? -- a word whereby \\
      The Truth (with the comfort it brings) \\
  Is open to all who grope in night, \\
  Crying for Wisdom's holy light. \\
 \\
  Whether the word is a verb or a noun \\
      Is knowledge beyond my reach. \\
  I only know that 'tis handed down. \\
          From sage to sage, \\
          From age to age -- \\
      An immortal part of speech! \\
 \\
  Of an ancient man the tale is told \\
  That he lived to be ten centuries old, \\
      In a cave on a mountain side. \\
      (True, he finally died.) \\
  The fame of his wisdom filled the land, \\
  For his head was bald, and you'll understand \\
      His beard was long and white \\
      And his eyes uncommonly bright. \\
 \\
  Philosophers gathered from far and near \\
  To sit at his feet and hear and hear, \\
          Though he never was heard \\
          To utter a word \\
      But "{\em Abracadabra, abracadab}, \\
          {\em Abracada, abracad}, \\
      {\em Abraca, abrac, abra, ab!}" \\
          'Twas all he had, \\
  'Twas all they wanted to hear, and each \\
  Made copious notes of the mystical speech, \\
          Which they published next -- \\
          A trickle of text \\
  In the meadow of commentary. \\
      Mighty big books were these, \\
      In a number, as leaves of trees; \\
  In learning, remarkably -- very! \\
 \\
          He's dead, \\
          As I said, \\
  And the books of the sages have perished, \\
  But his wisdom is sacredly cherished. \\
  In {\em Abracadabra} it solemnly rings, \\
  Like an ancient bell that forever swings. \\
  \\
  O, I love to hear \\
  That word make clear \\
  Humanity's General Sense of Things. \\
 \\
Jamrach Holobom \end{quote}


\paragraph{ABRIDGE, v.t.}  To shorten.

\begin{quote}       When in the course of human events it becomes necessary for \\
  people to abridge their king, a decent respect for the opinions of \\
  mankind requires that they should declare the causes which impel \\
  them to the separation. \\
 \\
Oliver Cromwell \end{quote}


\paragraph{ABRUPT, adj.}  Sudden, without ceremony, like the arrival of a cannon-
shot and the departure of the soldier whose interests are most
affected by it.  Dr. Samuel Johnson beautifully said of another
author's ideas that they were "concatenated without abruption."

\paragraph{ABSCOND, v.i.}  To "move in a mysterious way," commonly with the
property of another.

\begin{quote}   Spring beckons!  All things to the call respond; \\
  The trees are leaving and cashiers abscond. \\
 \\
Phela Orm \end{quote}


\paragraph{ABSENT, adj.}  Peculiarly exposed to the tooth of detraction; vilifed;
hopelessly in the wrong; superseded in the consideration and affection
of another.

\begin{quote}   To men a man is but a mind.  Who cares \\
  What face he carries or what form he wears? \\
  But woman's body is the woman.  O, \\
  Stay thou, my sweetheart, and do never go, \\
  But heed the warning words the sage hath said: \\
  A woman absent is a woman dead. \\
 \\
Jogo Tyree \end{quote}


\paragraph{ABSENTEE, n.}  A person with an income who has had the forethought to
remove himself from the sphere of exaction.

\paragraph{ABSOLUTE, adj.}  Independent, irresponsible.  An absolute monarchy is
one in which the sovereign does as he pleases so long as he pleases
the assassins.  Not many absolute monarchies are left, most of them
having been replaced by limited monarchies, where the sovereign's
power for evil (and for good) is greatly curtailed, and by republics,
which are governed by chance.

\paragraph{ABSTAINER, n.}  A weak person who yields to the temptation of denying
himself a pleasure.  A total abstainer is one who abstains from
everything but abstention, and especially from inactivity in the
affairs of others.

\begin{quote}   Said a man to a crapulent youth:  "I thought \\
      You a total abstainer, my son." \\
  "So I am, so I am," said the scrapgrace caught -- \\
      "But not, sir, a bigoted one." \\
 \\
G.J. \end{quote}


\paragraph{ABSURDITY, n.}  A statement or belief manifestly inconsistent with
one's own opinion.

\paragraph{ACADEME, n.}  An ancient school where morality and philosophy were
taught.

\paragraph{ACADEMY, n.}  [from ACADEME]   A modern school where football is
taught.

\paragraph{ACCIDENT, n.}  An inevitable occurrence due to the action of immutable
natural laws.

\paragraph{ACCOMPLICE, n.}  One associated with another in a crime, having guilty
knowledge and complicity, as an attorney who defends a criminal,
knowing him guilty.  This view of the attorney's position in the
matter has not hitherto commanded the assent of attorneys, no one
having offered them a fee for assenting.

\paragraph{ACCORD, n.}  Harmony.

\paragraph{ACCORDION, n.}  An instrument in harmony with the sentiments of an
assassin.

\paragraph{ACCOUNTABILITY, n.}  The mother of caution.

\begin{quote}   "My accountability, bear in mind," \\
      Said the Grand Vizier:  "Yes, yes," \\
  Said the Shah:  "I do -- 'tis the only kind \\
      Of ability you possess." \\
 \\
Joram Tate \end{quote}


\paragraph{ACCUSE, v.t.}  To affirm another's guilt or unworth; most commonly as a
justification of ourselves for having wronged him.

\paragraph{ACEPHALOUS, adj.}  In the surprising condition of the Crusader who
absently pulled at his forelock some hours after a Saracen scimitar
had, unconsciously to him, passed through his neck, as related by de
Joinville.

\paragraph{ACHIEVEMENT, n.}  The death of endeavor and the birth of disgust.

\paragraph{ACKNOWLEDGE, v.t.}  To confess.  Acknowledgement of one another's
faults is the highest duty imposed by our love of truth.

\paragraph{ACQUAINTANCE, n.}  A person whom we know well enough to borrow from,
but not well enough to lend to.  A degree of friendship called slight
when its object is poor or obscure, and intimate when he is rich or
famous.

\paragraph{ACTUALLY, adv.}  Perhaps; possibly.

\paragraph{ADAGE, n.}  Boned wisdom for weak teeth.

\paragraph{ADAMANT, n.}  A mineral frequently found beneath a corset.  Soluble in
solicitate of gold.

\paragraph{ADDER, n.}  A species of snake.  So called from its habit of adding
funeral outlays to the other expenses of living.

\paragraph{ADHERENT, n.}  A follower who has not yet obtained all that he expects
to get.

\paragraph{ADMINISTRATION, n.}  An ingenious abstraction in politics, designed to
receive the kicks and cuffs due to the premier or president.  A man of
straw, proof against bad-egging and dead-catting.

\paragraph{ADMIRAL, n.}  That part of a war-ship which does the talking while the
figure-head does the thinking.

\paragraph{ADMIRATION, n.}  Our polite recognition of another's resemblance to
ourselves.

\paragraph{ADMONITION, n.}  Gentle reproof, as with a meat-axe.  Friendly warning.

\begin{quote}   Consigned by way of admonition, \\
  His soul forever to perdition. \\
 \\
Judibras \end{quote}


\paragraph{ADORE, v.t.}  To venerate expectantly.

\paragraph{ADVICE, n.}  The smallest current coin.

\begin{quote}   "The man was in such deep distress," \\
  Said Tom, "that I could do no less \\
  Than give him good advice."  Said Jim: \\
  "If less could have been done for him \\
  I know you well enough, my son, \\
  To know that's what you would have done." \\
 \\
Jebel Jocordy \end{quote}


\paragraph{AFFIANCED, pp.}  Fitted with an ankle-ring for the ball-and-chain.

\paragraph{AFFLICTION, n.}  An acclimatizing process preparing the soul for
another and bitter world.

\paragraph{AFRICAN, n.}  A nigger that votes our way.

\paragraph{AGE, n.}  That period of life in which we compound for the vices that
we still cherish by reviling those that we have no longer the
enterprise to commit.

\paragraph{AGITATOR, n.}  A statesman who shakes the fruit trees of his neighbors
-- to dislodge the worms.

\paragraph{AIM, n.}  The task we set our wishes to.
\begin{quote}   "Cheer up!  Have you no aim in life?" \\
      She tenderly inquired. \\
  "An aim?  Well, no, I haven't, wife; \\
      The fact is -- I have fired." \\
 \\
G.J. \end{quote}


\paragraph{AIR, n.}  A nutritious substance supplied by a bountiful Providence for
the fattening of the poor.

\paragraph{ALDERMAN, n.}  An ingenious criminal who covers his secret thieving
with a pretence of open marauding.

\paragraph{ALIEN, n.}  An American sovereign in his probationary state.

\paragraph{ALLAH, n.}  The Mahometan Supreme Being, as distinguished from the
Christian, Jewish, and so forth.

\begin{quote}   Allah's good laws I faithfully have kept, \\
  And ever for the sins of man have wept; \\
      And sometimes kneeling in the temple I \\
  Have reverently crossed my hands and slept. \\
 \\
Junker Barlow \end{quote}


\paragraph{ALLEGIANCE, n.}

\begin{quote}   This thing Allegiance, as I suppose, \\
  Is a ring fitted in the subject's nose, \\
  Whereby that organ is kept rightly pointed \\
  To smell the sweetness of the Lord's anointed. \\
 \\
G.J. \end{quote}


\paragraph{ALLIANCE, n.}  In international politics, the union of two thieves who
have their hands so deeply inserted in each other's pockets that they
cannot separately plunder a third.

\paragraph{ALLIGATOR, n.}  The crocodile of America, superior in every detail to
the crocodile of the effete monarchies of the Old World.  Herodotus
says the Indus is, with one exception, the only river that produces
crocodiles, but they appear to have gone West and grown up with the
other rivers.  From the notches on his back the alligator is called a
sawrian.

\paragraph{ALONE, adj.}  In bad company.

\begin{quote}   In contact, lo! the flint and steel, \\
  By spark and flame, the thought reveal \\
  That he the metal, she the stone, \\
  Had cherished secretly alone. \\
 \\
Booley Fito \end{quote}


\paragraph{ALTAR, n.}  The place whereupon the priest formerly raveled out the
small intestine of the sacrificial victim for purposes of divination
and cooked its flesh for the gods.  The word is now seldom used,
except with reference to the sacrifice of their liberty and peace by a
male and a female tool.

\begin{quote}   They stood before the altar and supplied \\
  The fire themselves in which their fat was fried. \\
  In vain the sacrifice! -- no god will claim \\
  An offering burnt with an unholy flame. \\
 \\
M.P. Nopput \end{quote}


\paragraph{AMBIDEXTROUS, adj.}  Able to pick with equal skill a right-hand pocket
or a left.

\paragraph{AMBITION, n.}  An overmastering desire to be vilified by enemies while
living and made ridiculous by friends when dead.

\paragraph{AMNESTY, n.}  The state's magnanimity to those offenders whom it would
be too expensive to punish.

\paragraph{ANOINT, v.t.}  To grease a king or other great functionary already
sufficiently slippery.

\begin{quote}   As sovereigns are anointed by the priesthood, \\
  So pigs to lead the populace are greased good. \\
 \\
Judibras \end{quote}


\paragraph{ANTIPATHY, n.}  The sentiment inspired by one's friend's friend.

\paragraph{APHORISM, n.}  Predigested wisdom.

\begin{quote}   The flabby wine-skin of his brain \\
  Yields to some pathologic strain, \\
  And voids from its unstored abysm \\
  The driblet of an aphorism. \\
 \\
"The Mad Philosopher," 1697 \end{quote}


\paragraph{APOLOGIZE, v.i.}  To lay the foundation for a future offence.

\paragraph{APOSTATE, n.}  A leech who, having penetrated the shell of a turtle
only to find that the creature has long been dead, deems it expedient
to form a new attachment to a fresh turtle.

\paragraph{APOTHECARY, n.}  The physician's accomplice, undertaker's benefactor
and grave worm's provider.

\begin{quote}   When Jove sent blessings to all men that are, \\
  And Mercury conveyed them in a jar, \\
  That friend of tricksters introduced by stealth \\
  Disease for the apothecary's health, \\
  Whose gratitude impelled him to proclaim: \\
  "My deadliest drug shall bear my patron's name!" \\
 \\
G.J. \end{quote}


\paragraph{APPEAL, v.t.}  In law, to put the dice into the box for another throw.

\paragraph{APPETITE, n.}  An instinct thoughtfully implanted by Providence as a
solution to the labor question.

\paragraph{APPLAUSE, n.}  The echo of a platitude.

\paragraph{APRIL FOOL, n.}  The March fool with another month added to his folly.

\paragraph{ARCHBISHOP, n.}  An ecclesiastical dignitary one point holier than a
bishop.

\begin{quote}   If I were a jolly archbishop, \\
  On Fridays I'd eat all the fish up -- \\
  Salmon and flounders and smelts; \\
  On other days everything else. \\
 \\
Jodo Rem \end{quote}


\paragraph{ARCHITECT, n.}  One who drafts a plan of your house, and plans a draft
of your money.

\paragraph{ARDOR, n.}  The quality that distinguishes love without knowledge.

\paragraph{ARENA, n.}  In politics, an imaginary rat-pit in which the statesman
wrestles with his record.

\paragraph{ARISTOCRACY, n.}  Government by the best men.  (In this sense the word
is obsolete; so is that kind of government.)  Fellows that wear downy
hats and clean shirts -- guilty of education and suspected of bank
accounts.

\paragraph{ARMOR, n.}  The kind of clothing worn by a man whose tailor is a
blacksmith.

\paragraph{ARRAYED, pp.}  Drawn up and given an orderly disposition, as a rioter
hanged to a lamppost.

\paragraph{ARREST, v.t.}  Formally to detain one accused of unusualness.

\begin{quote}   God made the world in six days and was arrested on the seventh. \\
 \\
{\em The Unauthorized Version} \end{quote}


\paragraph{ARSENIC, n.}  A kind of cosmetic greatly affected by the ladies, whom
it greatly affects in turn.

\begin{quote}   "Eat arsenic?  Yes, all you get," \\
      Consenting, he did speak up; \\
  "'Tis better you should eat it, pet, \\
      Than put it in my teacup." \\
 \\
Joel Huck \end{quote}


\paragraph{ART, n.}  This word has no definition.  Its origin is related as
follows by the ingenious Father Gassalasca Jape, S.J.

\begin{quote}   One day a wag -- what would the wretch be at? -- \\
  Shifted a letter of the cipher RAT, \\
  And said it was a god's name!  Straight arose \\
  Fantastic priests and postulants (with shows, \\
  And mysteries, and mummeries, and hymns, \\
  And disputations dire that lamed their limbs) \\
  To serve his temple and maintain the fires, \\
  Expound the law, manipulate the wires. \\
  Amazed, the populace that rites attend, \\
  Believe whate'er they cannot comprehend, \\
  And, inly edified to learn that two \\
  Half-hairs joined so and so (as Art can do) \\
  Have sweeter values and a grace more fit \\
  Than Nature's hairs that never have been split, \\
  Bring cates and wines for sacrificial feasts, \\
  And sell their garments to support the priests.  \end{quote}

\paragraph{ARTLESSNESS, n.}  A certain engaging quality to which women attain by
long study and severe practice upon the admiring male, who is pleased
to fancy it resembles the candid simplicity of his young.

\paragraph{ASPERSE, v.t.}  Maliciously to ascribe to another vicious actions which
one has not had the temptation and opportunity to commit.

\paragraph{ASS, n.}  A public singer with a good voice but no ear.  In Virginia
City, Nevada, he is called the Washoe Canary, in Dakota, the Senator,
and everywhere the Donkey.  The animal is widely and variously
celebrated in the literature, art and religion of every age and
country; no other so engages and fires the human imagination as this
noble vertebrate.  Indeed, it is doubted by some (Ramasilus, {\em lib.
II., De Clem.}, and C. Stantatus, {\em De Temperamente}) if it is not a
god; and as such we know it was worshiped by the Etruscans, and, if we
may believe Macrobious, by the Cupasians also.  Of the only two
animals admitted into the Mahometan Paradise along with the souls of
men, the ass that carried Balaam is one, the dog of the Seven Sleepers
the other.  This is no small distinction.  From what has been written
about this beast might be compiled a library of great splendor and
magnitude, rivalling that of the Shakespearean cult, and that which
clusters about the Bible.  It may be said, generally, that all
literature is more or less Asinine.

\begin{quote}   "Hail, holy Ass!" the quiring angels sing; \\
  "Priest of Unreason, and of Discords King!" \\
  Great co-Creator, let Thy glory shine: \\
  God made all else, the Mule, the Mule is thine!" \\
 \\
G.J. \end{quote}


\paragraph{AUCTIONEER, n.}  The man who proclaims with a hammer that he has picked
a pocket with his tongue.

\paragraph{AUSTRALIA, n.}  A country lying in the South Sea, whose industrial and
commercial development has been unspeakably retarded by an unfortunate
dispute among geographers as to whether it is a continent or an
island.

\paragraph{AVERNUS, n.}  The lake by which the ancients entered the infernal
regions.  The fact that access to the infernal regions was obtained by
a lake is believed by the learned Marcus Ansello Scrutator to have
suggested the Christian rite of baptism by immersion.  This, however,
has been shown by Lactantius to be an error.

\begin{quote}   {\em Facilis descensus Averni,} \\
      The poet remarks; and the sense \\
  Of it is that when down-hill I turn I \\
      Will get more of punches than pence. \\
 \\
Jehal Dai Lupe \end{quote}




\section*{B}



\paragraph{BAAL, n.}  An old deity formerly much worshiped under various names.
As Baal he was popular with the Phoenicians; as Belus or Bel he had
the honor to be served by the priest Berosus, who wrote the famous
account of the Deluge; as Babel he had a tower partly erected to his
glory on the Plain of Shinar.  From Babel comes our English word
"babble."  Under whatever name worshiped, Baal is the Sun-god.  As
Beelzebub he is the god of flies, which are begotten of the sun's rays
on the stagnant water.  In Physicia Baal is still worshiped as Bolus,
and as Belly he is adored and served with abundant sacrifice by the
priests of Guttledom.

\paragraph{BABE or BABY, n.}  A misshapen creature of no particular age, sex, or
condition, chiefly remarkable for the violence of the sympathies and
antipathies it excites in others, itself without sentiment or emotion.
There have been famous babes; for example, little Moses, from whose
adventure in the bulrushes the Egyptian hierophants of seven centuries
before doubtless derived their idle tale of the child Osiris being
preserved on a floating lotus leaf.

\begin{quote}           Ere babes were invented \\
          The girls were contended. \\
          Now man is tormented \\
  Until to buy babes he has squandered \\
  His money.  And so I have pondered \\
          This thing, and thought may be \\
          'T were better that Baby \\
  The First had been eagled or condored. \\
 \\
Ro Amil \end{quote}


\paragraph{BACCHUS, n.}  A convenient deity invented by the ancients as an excuse
for getting drunk.

\begin{quote}   Is public worship, then, a sin, \\
      That for devotions paid to Bacchus \\
  The lictors dare to run us in, \\
      And resolutely thump and whack us? \\
 \\
Jorace \end{quote}


\paragraph{BACK, n.}  That part of your friend which it is your privilege to
contemplate in your adversity.

\paragraph{BACKBITE, v.t.}  To speak of a man as you find him when he can't find
you.

\paragraph{BAIT, n.}  A preparation that renders the hook more palatable.  The
best kind is beauty.

\paragraph{BAPTISM, n.}  A sacred rite of such efficacy that he who finds himself
in heaven without having undergone it will be unhappy forever.  It is
performed with water in two ways -- by immersion, or plunging, and by
aspersion, or sprinkling.

\begin{quote}   But whether the plan of immersion \\
  Is better than simple aspersion \\
      Let those immersed \\
      And those aspersed \\
  Decide by the Authorized Version, \\
  And by matching their agues tertian. \\
 \\
G.J. \end{quote}


\paragraph{BAROMETER, n.}  An ingenious instrument which indicates what kind of
weather we are having.

\paragraph{BARRACK, n.}  A house in which soldiers enjoy a portion of that of
which it is their business to deprive others.

\paragraph{BASILISK, n.}  The cockatrice.  A sort of serpent hatched form the egg
of a cock.  The basilisk had a bad eye, and its glance was fatal.
Many infidels deny this creature's existence, but Semprello Aurator
saw and handled one that had been blinded by lightning as a punishment
for having fatally gazed on a lady of rank whom Jupiter loved.  Juno
afterward restored the reptile's sight and hid it in a cave.  Nothing
is so well attested by the ancients as the existence of the basilisk,
but the cocks have stopped laying.

\paragraph{BASTINADO, n.}  The act of walking on wood without exertion.

\paragraph{BATH, n.}  A kind of mystic ceremony substituted for religious worship,
with what spiritual efficacy has not been determined.

\begin{quote}   The man who taketh a steam bath \\
  He loseth all the skin he hath, \\
  And, for he's boiled a brilliant red, \\
  Thinketh to cleanliness he's wed, \\
  Forgetting that his lungs he's soiling \\
  With dirty vapors of the boiling. \\
 \\
Richard Gwow \end{quote}


\paragraph{BATTLE, n.}  A method of untying with the teeth of a political knot
that would not yield to the tongue.

\paragraph{BEARD, n.}  The hair that is commonly cut off by those who justly
execrate the absurd Chinese custom of shaving the head.

\paragraph{BEAUTY, n.}  The power by which a woman charms a lover and terrifies a
husband.

\paragraph{BEFRIEND, v.t.}  To make an ingrate.

\paragraph{BEG, v.}  To ask for something with an earnestness proportioned to the
belief that it will not be given.

\begin{quote}   Who is that, father? \\
                        A mendicant, child, \\
  Haggard, morose, and unaffable -- wild! \\
  See how he glares through the bars of his cell! \\
  With Citizen Mendicant all is not well. \\
 \\
  Why did they put him there, father? \\
 \\
                                       Because \\
  Obeying his belly he struck at the laws. \\
 \\
  His belly? \\
 \\
              Oh, well, he was starving, my boy -- \\
  A state in which, doubtless, there's little of joy. \\
  No bite had he eaten for days, and his cry \\
  Was "Bread!" ever "Bread!" \\
 \\
                              What's the matter with pie? \\
 \\
  With little to wear, he had nothing to sell; \\
  To beg was unlawful -- improper as well. \\
 \\
  Why didn't he work? \\
 \\
                       He would even have done that, \\
  But men said:  "Get out!" and the State remarked:  "Scat!" \\
  I mention these incidents merely to show \\
  That the vengeance he took was uncommonly low. \\
  Revenge, at the best, is the act of a Siou, \\
  But for trifles -- \\
 \\
                      Pray what did bad Mendicant do? \\
 \\
  Stole two loaves of bread to replenish his lack \\
  And tuck out the belly that clung to his back. \\
 \\
  Is that {\em all} father dear? \\
 \\
                              There's little to tell: \\
  They sent him to jail, and they'll send him to -- well, \\
  The company's better than here we can boast, \\
  And there's -- \\
 \\
                  Bread for the needy, dear father? \\
 \\
                                                     Um -- toast. \\
 \\
Atka Mip \end{quote}


\paragraph{BEGGAR, n.}  One who has relied on the assistance of his friends.

\paragraph{BEHAVIOR, n.}  Conduct, as determined, not by principle, but by
breeding.  The word seems to be somewhat loosely used in Dr. Jamrach
Holobom's translation of the following lines from the {\em Dies Irae}:

\begin{quote}       Recordare, Jesu pie, \\
      Quod sum causa tuae viae. \\
      Ne me perdas illa die. \\
 \\
  Pray remember, sacred Savior, \\
  Whose the thoughtless hand that gave your \\
  Death-blow.  Pardon such behavior.  \end{quote}

\paragraph{BELLADONNA, n.}  In Italian a beautiful lady; in English a deadly
poison.  A striking example of the essential identity of the two
tongues.

\paragraph{BENEDICTINES, n.}  An order of monks otherwise known as black friars.

\begin{quote}   She thought it a crow, but it turn out to be \\
      A monk of St. Benedict croaking a text. \\
  "Here's one of an order of cooks," said she -- \\
      "Black friars in this world, fried black in the next." \\
 \\
"The Devil on Earth" (London, 1712) \end{quote}


\paragraph{BENEFACTOR, n.}  One who makes heavy purchases of ingratitude, without,
however, materially affecting the price, which is still within the
means of all.

\paragraph{BERENICE'S HAIR, n.}  A constellation ({\em Coma Berenices}) named in honor
of one who sacrificed her hair to save her husband.

\begin{quote}   Her locks an ancient lady gave \\
  Her loving husband's life to save; \\
  And men -- they honored so the dame -- \\
  Upon some stars bestowed her name. \\
 \\
  But to our modern married fair, \\
  Who'd give their lords to save their hair, \\
  No stellar recognition's given. \\
  There are not stars enough in heaven. \\
 \\
G.J. \end{quote}


\paragraph{BIGAMY, n.}  A mistake in taste for which the wisdom of the future will
adjudge a punishment called trigamy.

\paragraph{BIGOT, n.}  One who is obstinately and zealously attached to an opinion
that you do not entertain.

\paragraph{BILLINGSGATE, n.}  The invective of an opponent.

\paragraph{BIRTH, n.}  The first and direst of all disasters.  As to the nature of
it there appears to be no uniformity.  Castor and Pollux were born
from the egg.  Pallas came out of a skull.  Galatea was once a block
of stone.  Peresilis, who wrote in the tenth century, avers that he
grew up out of the ground where a priest had spilled holy water.  It
is known that Arimaxus was derived from a hole in the earth, made by a
stroke of lightning.  Leucomedon was the son of a cavern in Mount
Aetna, and I have myself seen a man come out of a wine cellar.

\paragraph{BLACKGUARD, n.}  A man whose qualities, prepared for display like a box
of berries in a market -- the fine ones on top -- have been opened on
the wrong side.  An inverted gentleman.

\paragraph{BLANK-VERSE, n.}  Unrhymed iambic pentameters -- the most difficult
kind of English verse to write acceptably; a kind, therefore, much
affected by those who cannot acceptably write any kind.

\paragraph{BODY-SNATCHER, n.}  A robber of grave-worms.  One who supplies the
young physicians with that with which the old physicians have supplied
the undertaker.  The hyena.

\begin{quote}   "One night," a doctor said, "last fall, \\
  I and my comrades, four in all, \\
      When visiting a graveyard stood \\
  Within the shadow of a wall. \\
 \\
  "While waiting for the moon to sink \\
  We saw a wild hyena slink \\
      About a new-made grave, and then \\
  Begin to excavate its brink! \\
 \\
  "Shocked by the horrid act, we made \\
  A sally from our ambuscade, \\
      And, falling on the unholy beast, \\
  Dispatched him with a pick and spade." \\
 \\
Bettel K. Jhones \end{quote}


\paragraph{BONDSMAN, n.}  A fool who, having property of his own, undertakes to
become responsible for that entrusted to another to a third.

Philippe of Orleans wishing to appoint one of his favorites, a
dissolute nobleman, to a high office, asked him what security he would
be able to give.  "I need no bondsmen," he replied, "for I can give
you my word of honor."  "And pray what may be the value of that?"
inquired the amused Regent.  "Monsieur, it is worth its weight in gold."

\paragraph{BORE, n.}  A person who talks when you wish him to listen.

\paragraph{BOTANY, n.}  The science of vegetables -- those that are not good to
eat, as well as those that are.  It deals largely with their flowers,
which are commonly badly designed, inartistic in color, and ill-
smelling.

\paragraph{BOTTLE-NOSED, adj.}  Having a nose created in the image of its maker.

\paragraph{BOUNDARY, n.}  In political geography, an imaginary line between two
nations, separating the imaginary rights of one from the imaginary
rights of the other.

\paragraph{BOUNTY, n.}  The liberality of one who has much, in permitting one who
has nothing to get all that he can.

\begin{quote}       A single swallow, it is said, devours ten millions of insects \\
  every year.  The supplying of these insects I take to be a signal \\
  instance of the Creator's bounty in providing for the lives of His \\
  creatures. \\
 \\
Henry Ward Beecher \end{quote}


\paragraph{BRAHMA, n.}  He who created the Hindoos, who are preserved by Vishnu
and destroyed by Siva -- a rather neater division of labor than is
found among the deities of some other nations.  The Abracadabranese,
for example, are created by Sin, maintained by Theft and destroyed by
Folly.  The priests of Brahma, like those of Abracadabranese, are holy
and learned men who are never naughty.

\begin{quote}   O Brahma, thou rare old Divinity, \\
  First Person of the Hindoo Trinity, \\
  You sit there so calm and securely, \\
  With feet folded up so demurely -- \\
  You're the First Person Singular, surely. \\
 \\
Polydore Smith \end{quote}


\paragraph{BRAIN, n.} An apparatus with which we think what we think.  That which
distinguishes the man who is content to {\em be} something from the man
who wishes to {\em do} something.  A man of great wealth, or one who has
been pitchforked into high station, has commonly such a headful of
brain that his neighbors cannot keep their hats on.  In our
civilization, and under our republican form of government, brain is so
highly honored that it is rewarded by exemption from the cares of
office.

\paragraph{BRANDY, n.}  A cordial composed of one part thunder-and-lightning, one
part remorse, two parts bloody murder, one part death-hell-and-the-
grave and four parts clarified Satan.  Dose, a headful all the time.
Brandy is said by Dr. Johnson to be the drink of heroes.  Only a hero
will venture to drink it.

\paragraph{BRIDE, n.}  A woman with a fine prospect of happiness behind her.

\paragraph{BRUTE, n.}  See HUSBAND.



\section*{C}



\paragraph{CAABA, n.}  A large stone presented by the archangel Gabriel to the
patriarch Abraham, and preserved at Mecca.  The patriarch had perhaps
asked the archangel for bread.

\paragraph{CABBAGE, n.}  A familiar kitchen-garden vegetable about as large and
wise as a man's head.
\subparagraph{}   The cabbage is so called from Cabagius, a prince who on ascending
the throne issued a decree appointing a High Council of Empire
consisting of the members of his predecessor's Ministry and the
cabbages in the royal garden.  When any of his Majesty's measures of
state policy miscarried conspicuously it was gravely announced that
several members of the High Council had been beheaded, and his
murmuring subjects were appeased.

\paragraph{CALAMITY, n.}  A more than commonly plain and unmistakable reminder
that the affairs of this life are not of our own ordering.  Calamities
are of two kinds:  misfortune to ourselves, and good fortune to
others.

\paragraph{CALLOUS, adj.}  Gifted with great fortitude to bear the evils
afflicting another.
\subparagraph{}   When Zeno was told that one of his enemies was no more he was
observed to be deeply moved.  "What!" said one of his disciples, "you
weep at the death of an enemy?"  "Ah, 'tis true," replied the great
Stoic; "but you should see me smile at the death of a friend."

\paragraph{CALUMNUS, n.}  A graduate of the School for Scandal.

\paragraph{CAMEL, n.}  A quadruped (the {\em Splaypes humpidorsus}) of great value to
the show business.  There are two kinds of camels -- the camel proper
and the camel improper.  It is the latter that is always exhibited.

\paragraph{CANNIBAL, n.}  A gastronome of the old school who preserves the simple
tastes and adheres to the natural diet of the pre-pork period.

\paragraph{CANNON, n.}  An instrument employed in the rectification of national
boundaries.

\paragraph{CANONICALS, n.}  The motley worm by Jesters of the Court of Heaven.

\paragraph{CAPITAL, n.}  The seat of misgovernment.  That which provides the fire,
the pot, the dinner, the table and the knife and fork for the
anarchist; the part of the repast that himself supplies is the
disgrace before meat.  {\em Capital Punishment}, a penalty regarding the
justice and expediency of which many worthy persons -- including all
the assassins -- entertain grave misgivings.

\paragraph{CARMELITE, n.}  A mendicant friar of the order of Mount Carmel.

\begin{quote}   As Death was a-rising out one day, \\
  Across Mount Camel he took his way, \\
      Where he met a mendicant monk, \\
      Some three or four quarters drunk, \\
  With a holy leer and a pious grin, \\
  Ragged and fat and as saucy as sin, \\
      Who held out his hands and cried: \\
  "Give, give in Charity's name, I pray. \\
  Give in the name of the Church.  O give, \\
  Give that her holy sons may live!" \\
      And Death replied, \\
      Smiling long and wide: \\
      "I'll give, holy father, I'll give thee -- a ride." \\
 \\
      With a rattle and bang \\
      Of his bones, he sprang \\
  From his famous Pale Horse, with his spear; \\
      By the neck and the foot \\
      Seized the fellow, and put \\
  Him astride with his face to the rear. \\
 \\
  The Monarch laughed loud with a sound that fell \\
  Like clods on the coffin's sounding shell: \\
  "Ho, ho!  A beggar on horseback, they say, \\
      Will ride to the devil!" -- and {\em thump} \\
      Fell the flat of his dart on the rump \\
  Of the charger, which galloped away. \\
 \\
  Faster and faster and faster it flew, \\
  Till the rocks and the flocks and the trees that grew \\
  By the road were dim and blended and blue \\
      To the wild, wild eyes \\
      Of the rider -- in size \\
      Resembling a couple of blackberry pies. \\
  Death laughed again, as a tomb might laugh \\
      At a burial service spoiled, \\
      And the mourners' intentions foiled \\
      By the body erecting \\
      Its head and objecting \\
  To further proceedings in its behalf. \\
 \\
  Many a year and many a day \\
  Have passed since these events away. \\
  The monk has long been a dusty corse, \\
  And Death has never recovered his horse. \\
      For the friar got hold of its tail, \\
      And steered it within the pale \\
  Of the monastery gray, \\
  Where the beast was stabled and fed \\
  With barley and oil and bread \\
  Till fatter it grew than the fattest friar, \\
  And so in due course was appointed Prior. \\
 \\
G.J. \end{quote}


\paragraph{CARNIVOROUS, adj.}  Addicted to the cruelty of devouring the timorous
vegetarian, his heirs and assigns.

\paragraph{CARTESIAN, adj.}  Relating to Descartes, a famous philosopher, author
of the celebrated dictum, {\em Cogito ergo sum} -- whereby he was pleased
to suppose he demonstrated the reality of human existence.  The dictum
might be improved, however, thus:  {\em Cogito cogito ergo cogito sum} --
"I think that I think, therefore I think that I am;" as close an
approach to certainty as any philosopher has yet made.

\paragraph{CAT, n.}  A soft, indestructible automaton provided by nature to be
kicked when things go wrong in the domestic circle.

\begin{quote}   This is a dog, \\
      This is a cat. \\
  This is a frog, \\
      This is a rat. \\
  Run, dog, mew, cat. \\
  Jump, frog, gnaw, rat. \\
 \\
Elevenson \end{quote}


\paragraph{CAVILER, n.}  A critic of our own work.

\paragraph{CEMETERY, n.}  An isolated suburban spot where mourners match lies,
poets write at a target and stone-cutters spell for a wager.  The
inscriptions following will serve to illustrate the success attained
in these Olympian games:

\begin{quote}       His virtues were so conspicuous that his enemies, unable to \\
  overlook them, denied them, and his friends, to whose loose lives \\
  they were a rebuke, represented them as vices.  They are here \\
  commemorated by his family, who shared them. \\
      In the earth we here prepare a \\
      Place to lay our little Clara. \\
 \\
Thomas M. and Mary Frazer \end{quote}

\begin{quote}       P.S. -- Gabriel will raise her.  \end{quote}

\paragraph{CENTAUR, n.}  One of a race of persons who lived before the division of
labor had been carried to such a pitch of differentiation, and who
followed the primitive economic maxim, "Every man his own horse."  The
best of the lot was Chiron, who to the wisdom and virtues of the horse
added the fleetness of man.  The scripture story of the head of John
the Baptist on a charger shows that pagan myths have somewhat
sophisticated sacred history.

\paragraph{CERBERUS, n.}  The watch-dog of Hades, whose duty it was to guard the
entrance -- against whom or what does not clearly appear; everybody,
sooner or later, had to go there, and nobody wanted to carry off the
entrance.  Cerberus is known to have had three heads, and some of the
poets have credited him with as many as a hundred.  Professor
Graybill, whose clerky erudition and profound knowledge of Greek give
his opinion great weight, has averaged all the estimates, and makes
the number twenty-seven -- a judgment that would be entirely
conclusive is Professor Graybill had known (a) something about dogs,
and (b) something about arithmetic.

\paragraph{CHILDHOOD, n.}  The period of human life intermediate between the
idiocy of infancy and the folly of youth -- two removes from the sin
of manhood and three from the remorse of age.

\paragraph{CHRISTIAN, n.}  One who believes that the New Testament is a divinely
inspired book admirably suited to the spiritual needs of his neighbor.
One who follows the teachings of Christ in so far as they are not
inconsistent with a life of sin.

\begin{quote}   I dreamed I stood upon a hill, and, lo! \\
  The godly multitudes walked to and fro \\
  Beneath, in Sabbath garments fitly clad, \\
  With pious mien, appropriately sad, \\
  While all the church bells made a solemn din -- \\
  A fire-alarm to those who lived in sin. \\
  Then saw I gazing thoughtfully below, \\
  With tranquil face, upon that holy show \\
  A tall, spare figure in a robe of white, \\
  Whose eyes diffused a melancholy light. \\
  "God keep you, strange," I exclaimed.  "You are \\
  No doubt (your habit shows it) from afar; \\
  And yet I entertain the hope that you, \\
  Like these good people, are a Christian too." \\
  He raised his eyes and with a look so stern \\
  It made me with a thousand blushes burn \\
  Replied -- his manner with disdain was spiced: \\
  "What!  I a Christian?  No, indeed!  I'm Christ." \\
 \\
G.J. \end{quote}


\paragraph{CIRCUS, n.}  A place where horses, ponies and elephants are permitted
to see men, women and children acting the fool.

\paragraph{CLAIRVOYANT, n.}  A person, commonly a woman, who has the power of
seeing that which is invisible to her patron, namely, that he is a
blockhead.

\paragraph{CLARIONET, n.}  An instrument of torture operated by a person with
cotton in his ears.  There are two instruments that are worse than a
clarionet -- two clarionets.

\paragraph{CLERGYMAN, n.}  A man who undertakes the management of our spiritual
affairs as a method of better his temporal ones.

\paragraph{CLIO, n.}  One of the nine Muses.  Clio's function was to preside over
history -- which she did with great dignity, many of the prominent
citizens of Athens occupying seats on the platform, the meetings being
addressed by Messrs. Xenophon, Herodotus and other popular speakers.

\paragraph{CLOCK, n.}  A machine of great moral value to man, allaying his concern
for the future by reminding him what a lot of time remains to him.

\begin{quote}   A busy man complained one day: \\
  "I get no time!"  "What's that you say?" \\
  Cried out his friend, a lazy quiz; \\
  "You have, sir, all the time there is. \\
  There's plenty, too, and don't you doubt it -- \\
  We're never for an hour without it." \\
 \\
Purzil Crofe \end{quote}


\paragraph{CLOSE-FISTED, adj.}  Unduly desirous of keeping that which many
meritorious persons wish to obtain.

\begin{quote}   "Close-fisted Scotchman!" Johnson cried \\
      To thrifty J. Macpherson; \\
  "See me -- I'm ready to divide \\
      With any worthy person." \\
  Sad Jamie:  "That is very true -- \\
      The boast requires no backing; \\
  And all are worthy, sir, to you, \\
      Who have what you are lacking." \\
 \\
Anita M. Bobe \end{quote}


\paragraph{COENOBITE, n.}  A man who piously shuts himself up to meditate upon the
sin of wickedness; and to keep it fresh in his mind joins a
brotherhood of awful examples.

\begin{quote}   O Coenobite, O coenobite, \\
      Monastical gregarian, \\
  You differ from the anchorite, \\
      That solitudinarian: \\
  With vollied prayers you wound Old Nick; \\
  With dropping shots he makes him sick. \\
 \\
Quincy Giles \end{quote}


\paragraph{COMFORT, n.}  A state of mind produced by contemplation of a neighbor's
uneasiness.

\paragraph{COMMENDATION, n.}  The tribute that we pay to achievements that
resembles, but do not equal, our own.

\paragraph{COMMERCE, n.}  A kind of transaction in which A plunders from B the
goods of C, and for compensation B picks the pocket of D of money
belonging to E.

\paragraph{COMMONWEALTH, n.}  An administrative entity operated by an incalculable
multitude of political parasites, logically active but fortuitously
efficient.

\begin{quote}   This commonwealth's capitol's corridors view, \\
  So thronged with a hungry and indolent crew \\
  Of clerks, pages, porters and all attaches \\
  Whom rascals appoint and the populace pays \\
  That a cat cannot slip through the thicket of shins \\
  Nor hear its own shriek for the noise of their chins. \\
  On clerks and on pages, and porters, and all, \\
  Misfortune attend and disaster befall! \\
  May life be to them a succession of hurts; \\
  May fleas by the bushel inhabit their shirts; \\
  May aches and diseases encamp in their bones, \\
  Their lungs full of tubercles, bladders of stones; \\
  May microbes, bacilli, their tissues infest, \\
  And tapeworms securely their bowels digest; \\
  May corn-cobs be snared without hope in their hair, \\
  And frequent impalement their pleasure impair. \\
  Disturbed be their dreams by the awful discourse \\
  Of audible sofas sepulchrally hoarse, \\
  By chairs acrobatic and wavering floors -- \\
  The mattress that kicks and the pillow that snores! \\
  Sons of cupidity, cradled in sin! \\
  Your criminal ranks may the death angel thin, \\
  Avenging the friend whom I couldn't work in. \\
 \\
K.Q. \end{quote}


\paragraph{COMPROMISE, n.}  Such an adjustment of conflicting interests as gives
each adversary the satisfaction of thinking he has got what he ought
not to have, and is deprived of nothing except what was justly his
due.

\paragraph{COMPULSION, n.}  The eloquence of power.

\paragraph{CONDOLE, v.i.}  To show that bereavement is a smaller evil than
sympathy.

\paragraph{CONFIDANT, CONFIDANTE,} n.  One entrusted by A with the secrets of B,
confided by {\em him} to C.

\paragraph{CONGRATULATION, n.}  The civility of envy.

\paragraph{CONGRESS, n.}  A body of men who meet to repeal laws.

\paragraph{CONNOISSEUR, n.}  A specialist who knows everything about something and
nothing about anything else.
\subparagraph{}   An old wine-bibber having been smashed in a railway collision,
some wine was pouted on his lips to revive him.  "Pauillac, 1873," he
murmured and died.

\paragraph{CONSERVATIVE, n.}  A statesman who is enamored of existing evils, as
distinguished from the Liberal, who wishes to replace them with
others.

\paragraph{CONSOLATION, n.}  The knowledge that a better man is more unfortunate
than yourself.

\paragraph{CONSUL, n.}  In American politics, a person who having failed to secure
an office from the people is given one by the Administration on
condition that he leave the country.

\paragraph{CONSULT, v.i.}  To seek another's disapproval of a course already
decided on.

\paragraph{CONTEMPT, n.}  The feeling of a prudent man for an enemy who is too
formidable safely to be opposed.

\paragraph{CONTROVERSY, n.}  A battle in which spittle or ink replaces the
injurious cannon-ball and the inconsiderate bayonet.

\begin{quote}   In controversy with the facile tongue -- \\
  That bloodless warfare of the old and young -- \\
  So seek your adversary to engage \\
  That on himself he shall exhaust his rage, \\
  And, like a snake that's fastened to the ground, \\
  With his own fangs inflict the fatal wound. \\
  You ask me how this miracle is done? \\
  Adopt his own opinions, one by one, \\
  And taunt him to refute them; in his wrath \\
  He'll sweep them pitilessly from his path. \\
  Advance then gently all you wish to prove, \\
  Each proposition prefaced with, "As you've \\
  So well remarked," or, "As you wisely say, \\
  And I cannot dispute," or, "By the way, \\
  This view of it which, better far expressed, \\
  Runs through your argument."  Then leave the rest \\
  To him, secure that he'll perform his trust \\
  And prove your views intelligent and just. \\
 \\
Conmore Apel Brune \end{quote}


\paragraph{CONVENT, n.}  A place of retirement for woman who wish for leisure to
meditate upon the vice of idleness.

\paragraph{CONVERSATION, n.}  A fair to the display of the minor mental
commodities, each exhibitor being too intent upon the arrangement of
his own wares to observe those of his neighbor.

\paragraph{CORONATION, n.}  The ceremony of investing a sovereign with the outward
and visible signs of his divine right to be blown skyhigh with a
dynamite bomb.

\paragraph{CORPORAL, n.}  A man who occupies the lowest rung of the military
ladder.

\begin{quote}   Fiercely the battle raged and, sad to tell, \\
  Our corporal heroically fell! \\
  Fame from her height looked down upon the brawl \\
  And said:  "He hadn't very far to fall." \\
 \\
Giacomo Smith \end{quote}


\paragraph{CORPORATION, n.}  An ingenious device for obtaining individual profit
without individual responsibility.

\paragraph{CORSAIR, n.}  A politician of the seas.

\paragraph{COURT FOOL, n.}  The plaintiff.

\paragraph{COWARD, n.}  One who in a perilous emergency thinks with his legs.

\paragraph{CRAYFISH, n.}  A small crustacean very much resembling the lobster, but
less indigestible.

\begin{quote}       In this small fish I take it that human wisdom is admirably \\
  figured and symbolized; for whereas the crayfish doth move only \\
  backward, and can have only retrospection, seeing naught but the \\
  perils already passed, so the wisdom of man doth not enable him to \\
  avoid the follies that beset his course, but only to apprehend \\
  their nature afterward. \\
 \\
Sir James Merivale \end{quote}


\paragraph{CREDITOR, n.}  One of a tribe of savages dwelling beyond the Financial
Straits and dreaded for their desolating incursions.

\paragraph{CREMONA, n.}  A high-priced violin made in Connecticut.

\paragraph{CRITIC, n.}  A person who boasts himself hard to please because nobody
tries to please him.

\begin{quote}   There is a land of pure delight, \\
      Beyond the Jordan's flood, \\
  Where saints, apparelled all in white, \\
      Fling back the critic's mud. \\
 \\
  And as he legs it through the skies, \\
      His pelt a sable hue, \\
  He sorrows sore to recognize \\
      The missiles that he threw. \\
 \\
Orrin Goof \end{quote}


\paragraph{CROSS, n.}  An ancient religious symbol erroneously supposed to owe its
significance to the most solemn event in the history of Christianity,
but really antedating it by thousands of years.  By many it has been
believed to be identical with the {\em crux ansata} of the ancient phallic
worship, but it has been traced even beyond all that we know of that,
to the rites of primitive peoples.  We have to-day the White Cross as
a symbol of chastity, and the Red Cross as a badge of benevolent
neutrality in war.  Having in mind the former, the reverend Father
Gassalasca Jape smites the lyre to the effect following:

\begin{quote}   "Be good, be good!" the sisterhood \\
      Cry out in holy chorus, \\
  And, to dissuade from sin, parade \\
      Their various charms before us. \\
 \\
  But why, O why, has ne'er an eye \\
      Seen her of winsome manner \\
  And youthful grace and pretty face \\
      Flaunting the White Cross banner? \\
 \\
  Now where's the need of speech and screed \\
      To better our behaving? \\
  A simpler plan for saving man \\
      (But, first, is he worth saving?) \\
 \\
  Is, dears, when he declines to flee \\
      From bad thoughts that beset him, \\
  Ignores the Law as 't were a straw, \\
      And wants to sin -- don't let him. \\
 \\
CUI BONO?  [Latin]  What good would that do {\em me}? \end{quote}

\paragraph{CUNNING, n.}  The faculty that distinguishes a weak animal or person
from a strong one.  It brings its possessor much mental satisfaction
and great material adversity.  An Italian proverb says:  "The furrier
gets the skins of more foxes than asses."

\paragraph{CUPID, n.}  The so-called god of love.  This bastard creation of a
barbarous fancy was no doubt inflicted upon mythology for the sins of
its deities.  Of all unbeautiful and inappropriate conceptions this is
the most reasonless and offensive.  The notion of symbolizing sexual
love by a semisexless babe, and comparing the pains of passion to the
wounds of an arrow -- of introducing this pudgy homunculus into art
grossly to materialize the subtle spirit and suggestion of the work --
this is eminently worthy of the age that, giving it birth, laid it on
the doorstep of prosperity.

\paragraph{CURIOSITY, n.}  An objectionable quality of the female mind.  The
desire to know whether or not a woman is cursed with curiosity is one
of the most active and insatiable passions of the masculine soul.

\paragraph{CURSE, v.t.}  Energetically to belabor with a verbal slap-stick.  This
is an operation which in literature, particularly in the drama, is
commonly fatal to the victim.  Nevertheless, the liability to a
cursing is a risk that cuts but a small figure in fixing the rates of
life insurance.

\paragraph{CYNIC, n.}  A blackguard whose faulty vision sees things as they are,
not as they ought to be.  Hence the custom among the Scythians of
plucking out a cynic's eyes to improve his vision.



\section*{D}



\paragraph{DAMN, v.}  A word formerly much used by the Paphlagonians, the meaning
of which is lost.  By the learned Dr. Dolabelly Gak it is believed to
have been a term of satisfaction, implying the highest possible degree
of mental tranquillity.  Professor Groke, on the contrary, thinks it
expressed an emotion of tumultuous delight, because it so frequently
occurs in combination with the word {\em jod} or {\em god}, meaning "joy."  It
would be with great diffidence that I should advance an opinion
conflicting with that of either of these formidable authorities.

\paragraph{DANCE, v.i.}  To leap about to the sound of tittering music, preferably
with arms about your neighbor's wife or daughter.  There are many
kinds of dances, but all those requiring the participation of the two
sexes have two characteristics in common:  they are conspicuously
innocent, and warmly loved by the vicious.

\paragraph{DANGER, n.}

\begin{quote}   A savage beast which, when it sleeps, \\
      Man girds at and despises, \\
  But takes himself away by leaps \\
      And bounds when it arises. \\
 \\
Ambat Delaso \end{quote}


\paragraph{DARING, n.}  One of the most conspicuous qualities of a man in
security.

\paragraph{DATARY, n.}  A high ecclesiastic official of the Roman Catholic Church,
whose important function is to brand the Pope's bulls with the words
{\em Datum Romae}.  He enjoys a princely revenue and the friendship of
God.

\paragraph{DAWN, n.}  The time when men of reason go to bed.  Certain old men
prefer to rise at about that time, taking a cold bath and a long walk
with an empty stomach, and otherwise mortifying the flesh.  They then
point with pride to these practices as the cause of their sturdy
health and ripe years; the truth being that they are hearty and old,
not because of their habits, but in spite of them.  The reason we find
only robust persons doing this thing is that it has killed all the
others who have tried it.

\paragraph{DAY, n.}  A period of twenty-four hours, mostly misspent.  This period
is divided into two parts, the day proper and the night, or day
improper -- the former devoted to sins of business, the latter
consecrated to the other sort.  These two kinds of social activity
overlap.

\paragraph{DEAD, adj.}

\begin{quote}   Done with the work of breathing; done \\
  With all the world; the mad race run \\
  Though to the end; the golden goal \\
  Attained and found to be a hole! \\
 \\
Squatol Johnes \end{quote}


\paragraph{DEBAUCHEE, n.}  One who has so earnestly pursued pleasure that he has
had the misfortune to overtake it.

\paragraph{DEBT, n.}  An ingenious substitute for the chain and whip of the slave-
driver.

\begin{quote}   As, pent in an aquarium, the troutlet \\
  Swims round and round his tank to find an outlet, \\
  Pressing his nose against the glass that holds him, \\
  Nor ever sees the prison that enfolds him; \\
  So the poor debtor, seeing naught around him, \\
  Yet feels the narrow limits that impound him, \\
  Grieves at his debt and studies to evade it, \\
  And finds at last he might as well have paid it. \\
 \\
Barlow S. Vode \end{quote}


\paragraph{DECALOGUE, n.}  A series of commandments, ten in number -- just enough
to permit an intelligent selection for observance, but not enough to
embarrass the choice.  Following is the revised edition of the
Decalogue, calculated for this meridian.

\begin{quote}   Thou shalt no God but me adore: \\
  'Twere too expensive to have more. \\
 \\
  No images nor idols make \\
  For Robert Ingersoll to break. \\
 \\
  Take not God's name in vain; select \\
  A time when it will have effect. \\
 \\
  Work not on Sabbath days at all, \\
  But go to see the teams play ball. \\
 \\
  Honor thy parents.  That creates \\
  For life insurance lower rates. \\
 \\
  Kill not, abet not those who kill; \\
  Thou shalt not pay thy butcher's bill. \\
 \\
  Kiss not thy neighbor's wife, unless \\
  Thine own thy neighbor doth caress \\
 \\
  Don't steal; thou'lt never thus compete \\
  Successfully in business.  Cheat. \\
 \\
  Bear not false witness -- that is low -- \\
  But "hear 'tis rumored so and so." \\
 \\
  Cover thou naught that thou hast not \\
  By hook or crook, or somehow, got. \\
 \\
G.J. \end{quote}


\paragraph{DECIDE, v.i.}  To succumb to the preponderance of one set of influences
over another set.

\begin{quote}   A leaf was riven from a tree, \\
  "I mean to fall to earth," said he. \\
 \\
  The west wind, rising, made him veer. \\
  "Eastward," said he, "I now shall steer." \\
 \\
  The east wind rose with greater force. \\
  Said he:  "'Twere wise to change my course." \\
 \\
  With equal power they contend. \\
  He said:  "My judgment I suspend." \\
 \\
  Down died the winds; the leaf, elate, \\
  Cried:  "I've decided to fall straight." \\
 \\
  "First thoughts are best?"  That's not the moral; \\
  Just choose your own and we'll not quarrel. \\
 \\
  Howe'er your choice may chance to fall, \\
  You'll have no hand in it at all. \\
 \\
G.J. \end{quote}


\paragraph{DEFAME, v.t.}  To lie about another.  To tell the truth about another.

\paragraph{DEFENCELESS, adj.}  Unable to attack.

\paragraph{DEGENERATE, adj.}  Less conspicuously admirable than one's ancestors.
The contemporaries of Homer were striking examples of degeneracy; it
required ten of them to raise a rock or a riot that one of the heroes
of the Trojan war could have raised with ease.  Homer never tires of
sneering at "men who live in these degenerate days," which is perhaps
why they suffered him to beg his bread -- a marked instance of
returning good for evil, by the way, for if they had forbidden him he
would certainly have starved.

\paragraph{DEGRADATION, n.}  One of the stages of moral and social progress from
private station to political preferment.

\paragraph{DEINOTHERIUM, n.}  An extinct pachyderm that flourished when the
Pterodactyl was in fashion.  The latter was a native of Ireland, its
name being pronounced Terry Dactyl or Peter O'Dactyl, as the man
pronouncing it may chance to have heard it spoken or seen it printed.

\paragraph{DEJEUNER, n.}  The breakfast of an American who has been in Paris.
Variously pronounced.

\paragraph{DELEGATION, n.}  In American politics, an article of merchandise that
comes in sets.

\paragraph{DELIBERATION, n.}  The act of examining one's bread to determine which
side it is buttered on.

\paragraph{DELUGE, n.}  A notable first experiment in baptism which washed away
the sins (and sinners) of the world.

\paragraph{DELUSION, n.}  The father of a most respectable family, comprising
Enthusiasm, Affection, Self-denial, Faith, Hope, Charity and many
other goodly sons and daughters.

\begin{quote}   All hail, Delusion!  Were it not for thee \\
  The world turned topsy-turvy we should see; \\
  For Vice, respectable with cleanly fancies, \\
  Would fly abandoned Virtue's gross advances. \\
 \\
Mumfrey Mappel \end{quote}


\paragraph{DENTIST, n.}  A prestidigitator who, putting metal into your mouth,
pulls coins out of your pocket.

\paragraph{DEPENDENT, adj.}  Reliant upon another's generosity for the support
which you are not in a position to exact from his fears.

\paragraph{DEPUTY, n.}  A male relative of an office-holder, or of his bondsman.
The deputy is commonly a beautiful young man, with a red necktie and
an intricate system of cobwebs extending from his nose to his desk.
When accidentally struck by the janitor's broom, he gives off a cloud
of dust.

\begin{quote}   "Chief Deputy," the Master cried, \\
  "To-day the books are to be tried \\
  By experts and accountants who \\
  Have been commissioned to go through \\
  Our office here, to see if we \\
  Have stolen injudiciously. \\
  Please have the proper entries made, \\
  The proper balances displayed, \\
  Conforming to the whole amount \\
  Of cash on hand -- which they will count. \\
  I've long admired your punctual way -- \\
  Here at the break and close of day, \\
  Confronting in your chair the crowd \\
  Of business men, whose voices loud \\
  And gestures violent you quell \\
  By some mysterious, calm spell -- \\
  Some magic lurking in your look \\
  That brings the noisiest to book \\
  And spreads a holy and profound \\
  Tranquillity o'er all around. \\
  So orderly all's done that they \\
  Who came to draw remain to pay. \\
  But now the time demands, at last, \\
  That you employ your genius vast \\
  In energies more active.  Rise \\
  And shake the lightnings from your eyes; \\
  Inspire your underlings, and fling \\
  Your spirit into everything!" \\
  The Master's hand here dealt a whack \\
  Upon the Deputy's bent back, \\
  When straightway to the floor there fell \\
  A shrunken globe, a rattling shell \\
  A blackened, withered, eyeless head! \\
  The man had been a twelvemonth dead. \\
 \\
Jamrach Holobom \end{quote}


\paragraph{DESTINY, n.}  A tyrant's authority for crime and fool's excuse for
failure.

\paragraph{DIAGNOSIS, n.}  A physician's forecast of the disease by the patient's
pulse and purse.

\paragraph{DIAPHRAGM, n.}  A muscular partition separating disorders of the chest
from disorders of the bowels.

\paragraph{DIARY, n.}  A daily record of that part of one's life, which he can
relate to himself without blushing.

\begin{quote}   Hearst kept a diary wherein were writ \\
  All that he had of wisdom and of wit. \\
  So the Recording Angel, when Hearst died, \\
  Erased all entries of his own and cried: \\
  "I'll judge you by your diary."  Said Hearst: \\
  "Thank you; 'twill show you I am Saint the First" -- \\
  Straightway producing, jubilant and proud, \\
  That record from a pocket in his shroud. \\
  The Angel slowly turned the pages o'er, \\
  Each stupid line of which he knew before, \\
  Glooming and gleaming as by turns he hit \\
  On Shallow sentiment and stolen wit; \\
  Then gravely closed the book and gave it back. \\
  "My friend, you've wandered from your proper track: \\
  You'd never be content this side the tomb -- \\
  For big ideas Heaven has little room, \\
  And Hell's no latitude for making mirth," \\
  He said, and kicked the fellow back to earth. \\
 \\
"The Mad Philosopher" \end{quote}


\paragraph{DICTATOR, n.}  The chief of a nation that prefers the pestilence of
despotism to the plague of anarchy.

\paragraph{DICTIONARY, n.}  A malevolent literary device for cramping the growth
of a language and making it hard and inelastic.  This dictionary,
however, is a most useful work.

\paragraph{DIE, n.}  The singular of "dice."  We seldom hear the word, because
there is a prohibitory proverb, "Never say die."  At long intervals,
however, some one says:  "The die is cast," which is not true, for it
is cut.  The word is found in an immortal couplet by that eminent poet
and domestic economist, Senator Depew:

\begin{quote}   A cube of cheese no larger than a die \\
  May bait the trap to catch a nibbling mie.  \end{quote}

\paragraph{DIGESTION, n.}  The conversion of victuals into virtues.  When the
process is imperfect, vices are evolved instead -- a circumstance from
which that wicked writer, Dr. Jeremiah Blenn, infers that the ladies
are the greater sufferers from dyspepsia.

\paragraph{DIPLOMACY, n.}  The patriotic art of lying for one's country.

\paragraph{DISABUSE, v.t.}  The present your neighbor with another and better
error than the one which he has deemed it advantageous to embrace.

\paragraph{DISCRIMINATE, v.i.}  To note the particulars in which one person or
thing is, if possible, more objectionable than another.

\paragraph{DISCUSSION, n.}  A method of confirming others in their errors.

\paragraph{DISOBEDIENCE, n.}  The silver lining to the cloud of servitude.

\paragraph{DISOBEY, v.t.}  To celebrate with an appropriate ceremony the maturity
of a command.

\begin{quote}   His right to govern me is clear as day, \\
  My duty manifest to disobey; \\
  And if that fit observance e'er I shut \\
  May I and duty be alike undone. \\
 \\
Israfel Brown \end{quote}


\paragraph{DISSEMBLE, v.i.}  To put a clean shirt upon the character.
\begin{quote}   Let us dissemble. \\
 \\
Adam \end{quote}


\paragraph{DISTANCE, n.}  The only thing that the rich are willing for the poor to
call theirs, and keep.

\paragraph{DISTRESS, n.}  A disease incurred by exposure to the prosperity of a
friend.

\paragraph{DIVINATION, n.}  The art of nosing out the occult.  Divination is of as
many kinds as there are fruit-bearing varieties of the flowering dunce
and the early fool.

\paragraph{DOG, n.}  A kind of additional or subsidiary Deity designed to catch
the overflow and surplus of the world's worship.  This Divine Being in
some of his smaller and silkier incarnations takes, in the affection
of Woman, the place to which there is no human male aspirant.  The Dog
is a survival -- an anachronism.  He toils not, neither does he spin,
yet Solomon in all his glory never lay upon a door-mat all day long,
sun-soaked and fly-fed and fat, while his master worked for the means
wherewith to purchase the idle wag of the Solomonic tail, seasoned
with a look of tolerant recognition.

\paragraph{DRAGOON, n.}  A soldier who combines dash and steadiness in so equal
measure that he makes his advances on foot and his retreats on
horseback.

\paragraph{DRAMATIST, n.}  One who adapts plays from the French.

\paragraph{DRUIDS, n.}  Priests and ministers of an ancient Celtic religion which
did not disdain to employ the humble allurement of human sacrifice.
Very little is now known about the Druids and their faith.  Pliny says
their religion, originating in Britain, spread eastward as far as
Persia.  Caesar says those who desired to study its mysteries went to
Britain.  Caesar himself went to Britain, but does not appear to have
obtained any high preferment in the Druidical Church, although his
talent for human sacrifice was considerable.
\subparagraph{}   Druids performed their religious rites in groves, and knew nothing
of church mortgages and the season-ticket system of pew rents.  They
were, in short, heathens and -- as they were once complacently
catalogued by a distinguished prelate of the Church of England --
Dissenters.

\paragraph{DUCK-BILL, n.}  Your account at your restaurant during the canvas-back
season.

\paragraph{DUEL, n.}  A formal ceremony preliminary to the reconciliation of two
enemies.  Great skill is necessary to its satisfactory observance; if
awkwardly performed the most unexpected and deplorable consequences
sometimes ensue.  A long time ago a man lost his life in a duel.

\begin{quote}   That dueling's a gentlemanly vice \\
      I hold; and wish that it had been my lot \\
      To live my life out in some favored spot -- \\
  Some country where it is considered nice \\
  To split a rival like a fish, or slice \\
      A husband like a spud, or with a shot \\
      Bring down a debtor doubled in a knot \\
  And ready to be put upon the ice. \\
  Some miscreants there are, whom I do long \\
      To shoot, to stab, or some such way reclaim \\
  The scurvy rogues to better lives and manners, \\
  I seem to see them now -- a mighty throng. \\
      It looks as if to challenge {\em me} they came, \\
  Jauntily marching with brass bands and banners! \\
 \\
Xamba Q. Dar \end{quote}


\paragraph{DULLARD, n.}  A member of the reigning dynasty in letters and life.
The Dullards came in with Adam, and being both numerous and sturdy
have overrun the habitable world.  The secret of their power is their
insensibility to blows; tickle them with a bludgeon and they laugh
with a platitude.  The Dullards came originally from Boeotia, whence
they were driven by stress of starvation, their dullness having
blighted the crops.  For some centuries they infested Philistia, and
many of them are called Philistines to this day.  In the turbulent
times of the Crusades they withdrew thence and gradually overspread
all Europe, occupying most of the high places in politics, art,
literature, science and theology.  Since a detachment of Dullards came
over with the Pilgrims in the {\em Mayflower} and made a favorable report
of the country, their increase by birth, immigration, and conversion
has been rapid and steady.  According to the most trustworthy
statistics the number of adult Dullards in the United States is but
little short of thirty millions, including the statisticians.  The
intellectual centre of the race is somewhere about Peoria, Illinois,
but the New England Dullard is the most shockingly moral.

\paragraph{DUTY, n.}  That which sternly impels us in the direction of profit,
along the line of desire.

\begin{quote}   Sir Lavender Portwine, in favor at court, \\
  Was wroth at his master, who'd kissed Lady Port. \\
  His anger provoked him to take the king's head, \\
  But duty prevailed, and he took the king's bread, \\
          Instead. \\
 \\
G.J. \end{quote}




\section*{E}



\paragraph{EAT, v.i.}  To perform successively (and successfully) the functions of
mastication, humectation, and deglutition.

\subparagraph{}"I was in the drawing-room, enjoying my dinner," said Brillat-Savarin,
beginning an anecdote.  "What!" interrupted Rochebriant;
"eating dinner in a drawing-room?"  "I must beg you to observe,
monsieur," explained the great gastronome, "that I did not say I was
eating my dinner, but enjoying it.  I had dined an hour before."

\paragraph{EAVESDROP, v.i.}  Secretly to overhear a catalogue of the crimes and
vices of another or yourself.

\begin{quote}   A lady with one of her ears applied \\
  To an open keyhole heard, inside, \\
  Two female gossips in converse free -- \\
  The subject engaging them was she. \\
  "I think," said one, "and my husband thinks \\
  That she's a prying, inquisitive minx!" \\
  As soon as no more of it she could hear \\
  The lady, indignant, removed her ear. \\
  "I will not stay," she said, with a pout, \\
  "To hear my character lied about!" \\
 \\
Gopete Sherany \end{quote}


\paragraph{ECCENTRICITY, n.}  A method of distinction so cheap that fools employ
it to accentuate their incapacity.

\paragraph{ECONOMY, n.}  Purchasing the barrel of whiskey that you do not need for
the price of the cow that you cannot afford.

\paragraph{EDIBLE, adj.}  Good to eat, and wholesome to digest, as a worm to a
toad, a toad to a snake, a snake to a pig, a pig to a man, and a man
to a worm.

\paragraph{EDITOR, n.}  A person who combines the judicial functions of Minos,
Rhadamanthus and Aeacus, but is placable with an obolus; a severely
virtuous censor, but so charitable withal that he tolerates the
virtues of others and the vices of himself; who flings about him the
splintering lightning and sturdy thunders of admonition till he
resembles a bunch of firecrackers petulantly uttering his mind at the
tail of a dog; then straightway murmurs a mild, melodious lay, soft as
the cooing of a donkey intoning its prayer to the evening star.
Master of mysteries and lord of law, high-pinnacled upon the throne of
thought, his face suffused with the dim splendors of the
Transfiguration, his legs intertwisted and his tongue a-cheek, the
editor spills his will along the paper and cuts it off in lengths to
suit.  And at intervals from behind the veil of the temple is heard
the voice of the foreman demanding three inches of wit and six lines
of religious meditation, or bidding him turn off the wisdom and whack
up some pathos.

\begin{quote} O, the Lord of Law on the Throne of Thought, \\
       A gilded impostor is he. \\
  Of shreds and patches his robes are wrought, \\
              His crown is brass, \\
              Himself an ass, \\
      And his power is fiddle-dee-dee. \\
  Prankily, crankily prating of naught, \\
  Silly old quilly old Monarch of Thought. \\
      Public opinion's camp-follower he, \\
      Thundering, blundering, plundering free. \\
                  Affected, \\
                      Ungracious, \\
                  Suspected, \\
                      Mendacious, \\
  Respected contemporaree! \\
                                                    J.H. Bumbleshook  \end{quote}

\paragraph{EDUCATION, n.} That which discloses to the wise and disguises from the
foolish their lack of understanding.

\paragraph{EFFECT, n.}  The second of two phenomena which always occur together in
the same order.  The first, called a Cause, is said to generate the
other -- which is no more sensible than it would be for one who has
never seen a dog except in the pursuit of a rabbit to declare the
rabbit the cause of a dog.

\paragraph{EGOTIST, n.}  A person of low taste, more interested in himself than in me.

\begin{quote}   Megaceph, chosen to serve the State \\
  In the halls of legislative debate, \\
  One day with all his credentials came \\
  To the capitol's door and announced his name. \\
  The doorkeeper looked, with a comical twist \\
  Of the face, at the eminent egotist, \\
  And said:  "Go away, for we settle here \\
  All manner of questions, knotty and queer, \\
  And we cannot have, when the speaker demands \\
  To be told how every member stands, \\
  A man who to all things under the sky \\
  Assents by eternally voting 'I'."  \end{quote}

\paragraph{EJECTION, n.}  An approved remedy for the disease of garrulity.  It is
also much used in cases of extreme poverty.

\paragraph{ELECTOR, n.}  One who enjoys the sacred privilege of voting for the man
of another man's choice.

\paragraph{ELECTRICITY, n.}  The power that causes all natural phenomena not known
to be caused by something else.  It is the same thing as lightning,
and its famous attempt to strike Dr. Franklin is one of the most
picturesque incidents in that great and good man's career.  The memory
of Dr. Franklin is justly held in great reverence, particularly in
France, where a waxen effigy of him was recently on exhibition,
bearing the following touching account of his life and services to
science:

\begin{quote}       "Monsieur Franqulin, inventor of electricity.  This \\
  illustrious savant, after having made several voyages around the \\
  world, died on the Sandwich Islands and was devoured by savages, \\
  of whom not a single fragment was ever recovered."
\end{quote}
  
\subparagraph{}  Electricity seems destined to play a most important part in the
arts and industries.  The question of its economical application to
some purposes is still unsettled, but experiment has already proved
that it will propel a street car better than a gas jet and give more
light than a horse.

\paragraph{ELEGY, n.}  A composition in verse, in which, without employing any of
the methods of humor, the writer aims to produce in the reader's mind
the dampest kind of dejection.  The most famous English example begins
somewhat like this:

\begin{quote}   The cur foretells the knell of parting day; \\
      The loafing herd winds slowly o'er the lea; \\
  The wise man homeward plods; I only stay \\
      To fiddle-faddle in a minor key.  \end{quote}

\paragraph{ELOQUENCE, n.}  The art of orally persuading fools that white is the
color that it appears to be.  It includes the gift of making any color
appear white.

\paragraph{ELYSIUM, n.}  An imaginary delightful country which the ancients
foolishly believed to be inhabited by the spirits of the good.  This
ridiculous and mischievous fable was swept off the face of the earth
by the early Christians -- may their souls be happy in Heaven!

\paragraph{EMANCIPATION, n.}  A bondman's change from the tyranny of another to
the despotism of himself.

\begin{quote}   He was a slave:  at word he went and came; \\
      His iron collar cut him to the bone. \\
  Then Liberty erased his owner's name, \\
      Tightened the rivets and inscribed his own. \\
 \\
G.J. \end{quote}


\paragraph{EMBALM, v.i.}  To cheat vegetation by locking up the gases upon which
it feeds.  By embalming their dead and thereby deranging the natural
balance between animal and vegetable life, the Egyptians made their
once fertile and populous country barren and incapable of supporting
more than a meagre crew.  The modern metallic burial casket is a step
in the same direction, and many a dead man who ought now to be
ornamenting his neighbor's lawn as a tree, or enriching his table as a
bunch of radishes, is doomed to a long inutility.  We shall get him
after awhile if we are spared, but in the meantime the violet and rose
are languishing for a nibble at his {\em glutoeus maximus}.

\paragraph{EMOTION, n.}  A prostrating disease caused by a determination of the
heart to the head.  It is sometimes accompanied by a copious discharge
of hydrated chloride of sodium from the eyes.

\paragraph{ENCOMIAST, n.}  A special (but not particular) kind of liar.

\paragraph{END, n.}  The position farthest removed on either hand from the
Interlocutor.

\begin{quote}   The man was perishing apace \\
      Who played the tambourine; \\
  The seal of death was on his face -- \\
      'Twas pallid, for 'twas clean. \\
 \\
  "This is the end," the sick man said \\
      In faint and failing tones. \\
  A moment later he was dead, \\
      And Tambourine was Bones. \\
 \\
Tinley Roquot \end{quote}


\paragraph{ENOUGH, pro.}  All there is in the world if you like it.

\begin{quote}   Enough is as good as a feast -- for that matter \\
  Enougher's as good as a feast for the platter. \\
 \\
Arbely C. Strunk \end{quote}


\paragraph{ENTERTAINMENT, n.}  Any kind of amusement whose inroads stop short of
death by injection.

\paragraph{ENTHUSIASM, n.}  A distemper of youth, curable by small doses of
repentance in connection with outward applications of experience.
Byron, who recovered long enough to call it "entuzy-muzy," had a
relapse, which carried him off -- to Missolonghi.

\paragraph{ENVELOPE, n.}  The coffin of a document; the scabbard of a bill; the
husk of a remittance; the bed-gown of a love-letter.

\paragraph{ENVY, n.}  Emulation adapted to the meanest capacity.

\paragraph{EPAULET, n.}  An ornamented badge, serving to distinguish a military
officer from the enemy -- that is to say, from the officer of lower
rank to whom his death would give promotion.

\paragraph{EPICURE, n.}  An opponent of Epicurus, an abstemious philosopher who,
holding that pleasure should be the chief aim of man, wasted no time
in gratification from the senses.

\paragraph{EPIGRAM, n.}  A short, sharp saying in prose or verse, frequently
characterize by acidity or acerbity and sometimes by wisdom.
Following are some of the more notable epigrams of the learned and
ingenious Dr. Jamrach Holobom:

\begin{quote}       We know better the needs of ourselves than of others.  To \\
  serve oneself is economy of administration. \\
 \\
      In each human heart are a tiger, a pig, an ass and a \\
  nightingale.  Diversity of character is due to their unequal \\
  activity. \\
 \\
      There are three sexes; males, females and girls. \\
 \\
      Beauty in women and distinction in men are alike in this: \\
  they seem to be the unthinking a kind of credibility. \\
      Women in love are less ashamed than men.  They have less to be \\
  ashamed of. \\
 \\
      While your friend holds you affectionately by both your hands \\
  you are safe, for you can watch both his.  \end{quote}

\paragraph{EPITAPH, n.}  An inscription on a tomb, showing that virtues acquired
by death have a retroactive effect.  Following is a touching example:

\begin{quote}   Here lie the bones of Parson Platt, \\
  Wise, pious, humble and all that, \\
  Who showed us life as all should live it; \\
  Let that be said -- and God forgive it!  \end{quote}

\paragraph{ERUDITION, n.}  Dust shaken out of a book into an empty skull.

\begin{quote}   So wide his erudition's mighty span, \\
  He knew Creation's origin and plan \\
  And only came by accident to grief -- \\
  He thought, poor man, 'twas right to be a thief. \\
 \\
Romach Pute \end{quote}


\paragraph{ESOTERIC, adj.}  Very particularly abstruse and consummately occult.
The ancient philosophies were of two kinds, -- {\em exoteric}, those that
the philosophers themselves could partly understand, and {\em esoteric},
those that nobody could understand.  It is the latter that have most
profoundly affected modern thought and found greatest acceptance in
our time.

\paragraph{ETHNOLOGY, n.}  The science that treats of the various tribes of Man,
as robbers, thieves, swindlers, dunces, lunatics, idiots and
ethnologists.

\paragraph{EUCHARIST, n.}  A sacred feast of the religious sect of Theophagi.
\subparagraph{}   A dispute once unhappily arose among the members of this sect as
to what it was that they ate.  In this controversy some five hundred
thousand have already been slain, and the question is still unsettled.

\paragraph{EULOGY, n.}  Praise of a person who has either the advantages of wealth
and power, or the consideration to be dead.

\paragraph{EVANGELIST, n.}  A bearer of good tidings, particularly (in a religious
sense) such as assure us of our own salvation and the damnation of
our neighbors.

\paragraph{EVERLASTING, adj.}  Lasting forever.  It is with no small diffidence
that I venture to offer this brief and elementary definition, for I am
not unaware of the existence of a bulky volume by a sometime Bishop of
Worcester, entitled, {\em A Partial Definition of the Word "Everlasting,"
as Used in the Authorized Version of the Holy Scriptures}.  His book
was once esteemed of great authority in the Anglican Church, and is
still, I understand, studied with pleasure to the mind and profit of
the soul.

\paragraph{EXCEPTION, n.}  A thing which takes the liberty to differ from other
things of its class, as an honest man, a truthful woman, etc.  "The
exception proves the rule" is an expression constantly upon the lips
of the ignorant, who parrot it from one another with never a thought
of its absurdity.  In the Latin, "{\em Exceptio probat regulam}" means
that the exception {\em tests} the rule, puts it to the proof, not
{\em confirms} it.  The malefactor who drew the meaning from this
excellent dictum and substituted a contrary one of his own exerted an
evil power which appears to be immortal.

\paragraph{EXCESS, n.}  In morals, an indulgence that enforces by appropriate
penalties the law of moderation.

\begin{quote}   Hail, high Excess -- especially in wine, \\
      To thee in worship do I bend the knee \\
      Who preach abstemiousness unto me -- \\
  My skull thy pulpit, as my paunch thy shrine. \\
  Precept on precept, aye, and line on line, \\
      Could ne'er persuade so sweetly to agree \\
      With reason as thy touch, exact and free, \\
  Upon my forehead and along my spine. \\
  At thy command eschewing pleasure's cup, \\
      With the hot grape I warm no more my wit; \\
      When on thy stool of penitence I sit \\
  I'm quite converted, for I can't get up. \\
  Ungrateful he who afterward would falter \\
  To make new sacrifices at thine altar!  \end{quote}

\paragraph{EXCOMMUNICATION, n.}

\begin{quote}   This "excommunication" is a word \\
  In speech ecclesiastical oft heard, \\
  And means the damning, with bell, book and candle, \\
  Some sinner whose opinions are a scandal -- \\
  A rite permitting Satan to enslave him \\
  Forever, and forbidding Christ to save him. \\
 \\
Gat Huckle \end{quote}


\paragraph{EXECUTIVE, n.}  An officer of the Government, whose duty it is to
enforce the wishes of the legislative power until such time as the
judicial department shall be pleased to pronounce them invalid and of
no effect.  Following is an extract from an old book entitled, {\em The
Lunarian Astonished} -- Pfeiffer \& Co., Boston, 1803:

\begin{quote}   LUNARIAN:  Then when your Congress has passed a law it goes \\
      directly to the Supreme Court in order that it may at once be \\
      known whether it is constitutional? \\
  TERRESTRIAN:  O no; it does not require the approval of the \\
      Supreme Court until having perhaps been enforced for many \\
      years somebody objects to its operation against himself -- I \\
      mean his client.  The President, if he approves it, begins to \\
      execute it at once. \\
  LUNARIAN:  Ah, the executive power is a part of the legislative. \\
      Do your policemen also have to approve the local ordinances \\
      that they enforce? \\
  TERRESTRIAN:  Not yet -- at least not in their character of \\
      constables.  Generally speaking, though, all laws require the \\
      approval of those whom they are intended to restrain. \\
  LUNARIAN:  I see.  The death warrant is not valid until signed by \\
      the murderer. \\
  TERRESTRIAN:  My friend, you put it too strongly; we are not so \\
      consistent. \\
  LUNARIAN:  But this system of maintaining an expensive judicial \\
      machinery to pass upon the validity of laws only after they \\
      have long been executed, and then only when brought before the \\
      court by some private person -- does it not cause great \\
      confusion? \\
  TERRESTRIAN:  It does. \\
  LUNARIAN:  Why then should not your laws, previously to being \\
      executed, be validated, not by the signature of your \\
      President, but by that of the Chief Justice of the Supreme \\
      Court? \\
  TERRESTRIAN:  There is no precedent for any such course. \\
  LUNARIAN:  Precedent.  What is that? \\
  TERRESTRIAN:  It has been defined by five hundred lawyers in three \\
      volumes each.  So how can any one know?  \end{quote}

\paragraph{EXHORT, v.t.} In religious affairs, to put the conscience of another
upon the spit and roast it to a nut-brown discomfort.

\paragraph{EXILE, n.}  One who serves his country by residing abroad, yet is not
an ambassador.
\subparagraph{}   An English sea-captain being asked if he had read "The Exile of
Erin," replied:  "No, sir, but I should like to anchor on it."  Years 
afterwards, when he had been hanged as a pirate after a career of
unparalleled atrocities, the following memorandum was found in the
ship's log that he had kept at the time of his reply:

\subparagraph{}   Aug. 3d, 1842.  Made a joke on the ex-Isle of Erin.  Coldly
  received.  War with the whole world!

\paragraph{EXISTENCE, n.}

\begin{quote}   A transient, horrible, fantastic dream, \\
  Wherein is nothing yet all things do seem: \\
  From which we're wakened by a friendly nudge \\
  Of our bedfellow Death, and cry:  "O fudge!"  \end{quote}

\paragraph{EXPERIENCE, n.}  The wisdom that enables us to recognize as an
undesirable old acquaintance the folly that we have already embraced.

\begin{quote}   To one who, journeying through night and fog, \\
  Is mired neck-deep in an unwholesome bog, \\
  Experience, like the rising of the dawn, \\
  Reveals the path that he should not have gone. \\
 \\
Joel Frad Bink \end{quote}


\paragraph{EXPOSTULATION, n.}  One of the many methods by which fools prefer to
lose their friends.

\paragraph{EXTINCTION, n.}  The raw material out of which theology created the
future state.



\section*{F}



\paragraph{FAIRY, n.}  A creature, variously fashioned and endowed, that formerly
inhabited the meadows and forests.  It was nocturnal in its habits,
and somewhat addicted to dancing and the theft of children.  The
fairies are now believed by naturalist to be extinct, though a
clergyman of the Church of England saw three near Colchester as lately
as 1855, while passing through a park after dining with the lord of
the manor.  The sight greatly staggered him, and he was so affected
that his account of it was incoherent.  In the year 1807 a troop of
fairies visited a wood near Aix and carried off the daughter of a
peasant, who had been seen to enter it with a bundle of clothing.  The
son of a wealthy {\em bourgeois} disappeared about the same time, but
afterward returned.  He had seen the abduction been in pursuit of the
fairies.  Justinian Gaux, a writer of the fourteenth century, avers
that so great is the fairies' power of transformation that he saw one
change itself into two opposing armies and fight a battle with great
slaughter, and that the next day, after it had resumed its original
shape and gone away, there were seven hundred bodies of the slain
which the villagers had to bury.  He does not say if any of the
wounded recovered.  In the time of Henry III, of England, a law was
made which prescribed the death penalty for "Kyllynge, wowndynge, or
mamynge" a fairy, and it was universally respected.

\paragraph{FAITH, n.}  Belief without evidence in what is told by one who speaks
without knowledge, of things without parallel.

\paragraph{FAMOUS, adj.}  Conspicuously miserable.

\begin{quote}   Done to a turn on the iron, behold \\
      Him who to be famous aspired. \\
  Content?  Well, his grill has a plating of gold, \\
      And his twistings are greatly admired. \\
 \\
Hassan Brubuddy \end{quote}


\paragraph{FASHION, n.}  A despot whom the wise ridicule and obey.

\begin{quote}   A king there was who lost an eye \\
      In some excess of passion; \\
  And straight his courtiers all did try \\
      To follow the new fashion. \\
 \\
  Each dropped one eyelid when before \\
      The throne he ventured, thinking \\
  'Twould please the king.  That monarch swore \\
      He'd slay them all for winking. \\
 \\
  What should they do?  They were not hot \\
      To hazard such disaster; \\
  They dared not close an eye -- dared not \\
      See better than their master. \\
 \\
  Seeing them lacrymose and glum, \\
      A leech consoled the weepers: \\
  He spread small rags with liquid gum \\
      And covered half their peepers. \\
 \\
  The court all wore the stuff, the flame \\
      Of royal anger dying. \\
  That's how court-plaster got its name \\
      Unless I'm greatly lying. \\
 \\
Naramy Oof \end{quote}


\paragraph{FEAST, n.}  A festival.  A religious celebration usually signalized by
gluttony and drunkenness, frequently in honor of some holy person
distinguished for abstemiousness.  In the Roman Catholic Church
feasts are "movable" and "immovable," but the celebrants are uniformly
immovable until they are full.  In their earliest development these
entertainments took the form of feasts for the dead; such were held by
the Greeks, under the name {\em Nemeseia}, by the Aztecs and Peruvians,
as in modern times they are popular with the Chinese; though it is
believed that the ancient dead, like the modern, were light eaters.
Among the many feasts of the Romans was the {\em Novemdiale}, which was
held, according to Livy, whenever stones fell from heaven.

\paragraph{FELON, n.}  A person of greater enterprise than discretion, who in
embracing an opportunity has formed an unfortunate attachment.

\paragraph{FEMALE, n.}  One of the opposing, or unfair, sex.

\begin{quote}   The Maker, at Creation's birth, \\
  With living things had stocked the earth. \\
  From elephants to bats and snails, \\
  They all were good, for all were males. \\
  But when the Devil came and saw \\
  He said:  "By Thine eternal law \\
  Of growth, maturity, decay, \\
  These all must quickly pass away \\
  And leave untenanted the earth \\
  Unless Thou dost establish birth" -- \\
  Then tucked his head beneath his wing \\
  To laugh -- he had no sleeve -- the thing \\
  With deviltry did so accord, \\
  That he'd suggested to the Lord. \\
  The Master pondered this advice, \\
  Then shook and threw the fateful dice \\
  Wherewith all matters here below \\
  Are ordered, and observed the throw; \\
  Then bent His head in awful state, \\
  Confirming the decree of Fate. \\
  From every part of earth anew \\
  The conscious dust consenting flew, \\
  While rivers from their courses rolled \\
  To make it plastic for the mould. \\
  Enough collected (but no more, \\
  For niggard Nature hoards her store) \\
  He kneaded it to flexible clay, \\
  While Nick unseen threw some away. \\
  And then the various forms He cast, \\
  Gross organs first and finer last; \\
  No one at once evolved, but all \\
  By even touches grew and small \\
  Degrees advanced, till, shade by shade, \\
  To match all living things He'd made \\
  Females, complete in all their parts \\
  Except (His clay gave out) the hearts. \\
  "No matter," Satan cried; "with speed \\
  I'll fetch the very hearts they need" -- \\
  So flew away and soon brought back \\
  The number needed, in a sack. \\
  That night earth range with sounds of strife -- \\
  Ten million males each had a wife; \\
  That night sweet Peace her pinions spread \\
  O'er Hell -- ten million devils dead! \\
 \\
G.J. \end{quote}


\paragraph{FIB, n.}  A lie that has not cut its teeth.  An habitual liar's nearest
approach to truth:  the perigee of his eccentric orbit.

\begin{quote}   When David said:  "All men are liars," Dave, \\
      Himself a liar, fibbed like any thief. \\
      Perhaps he thought to weaken disbelief \\
  By proof that even himself was not a slave \\
  To Truth; though I suspect the aged knave \\
      Had been of all her servitors the chief \\
      Had he but known a fig's reluctant leaf \\
  Is more than e'er she wore on land or wave. \\
  No, David served not Naked Truth when he \\
      Struck that sledge-hammer blow at all his race; \\
          Nor did he hit the nail upon the head: \\
  For reason shows that it could never be, \\
      And the facts contradict him to his face. \\
          Men are not liars all, for some are dead. \\
 \\
Bartle Quinker \end{quote}


\paragraph{FICKLENESS, n.}  The iterated satiety of an enterprising affection.

\paragraph{FIDDLE, n.}  An instrument to tickle human ears by friction of a
horse's tail on the entrails of a cat.

\begin{quote}   To Rome said Nero:  "If to smoke you turn \\
  I shall not cease to fiddle while you burn." \\
  To Nero Rome replied:  "Pray do your worst, \\
  'Tis my excuse that you were fiddling first." \\
 \\
Orm Pludge \end{quote}


\paragraph{FIDELITY, n.}  A virtue peculiar to those who are about to be betrayed.

\paragraph{FINANCE, n.}  The art or science of managing revenues and resources for
the best advantage of the manager.  The pronunciation of this word
with the i long and the accent on the first syllable is one of
America's most precious discoveries and possessions.

\paragraph{FLAG, n.}  A colored rag borne above troops and hoisted on forts and
ships.  It appears to serve the same purpose as certain signs that one
sees and vacant lots in London -- "Rubbish may be shot here."

\paragraph{FLESH, n.}  The Second Person of the secular Trinity.

\paragraph{FLOP, v.}  Suddenly to change one's opinions and go over to another
party.  The most notable flop on record was that of Saul of Tarsus,
who has been severely criticised as a turn-coat by some of our
partisan journals.

\paragraph{FLY-SPECK, n.}  The prototype of punctuation.  It is observed by
Garvinus that the systems of punctuation in use by the various
literary nations depended originally upon the social habits and
general diet of the flies infesting the several countries.  These
creatures, which have always been distinguished for a neighborly and
companionable familiarity with authors, liberally or niggardly
embellish the manuscripts in process of growth under the pen,
according to their bodily habit, bringing out the sense of the work by
a species of interpretation superior to, and independent of, the
writer's powers.  The "old masters" of literature -- that is to say,
the early writers whose work is so esteemed by later scribes and
critics in the same language -- never punctuated at all, but worked
right along free-handed, without that abruption of the thought which
comes from the use of points.  (We observe the same thing in children
to-day, whose usage in this particular is a striking and beautiful
instance of the law that the infancy of individuals reproduces the
methods and stages of development characterizing the infancy of
races.)  In the work of these primitive scribes all the punctuation is
found, by the modern investigator with his optical instruments and
chemical tests, to have been inserted by the writers' ingenious and
serviceable collaborator, the common house-fly -- {\em Musca maledicta}.
In transcribing these ancient MSS, for the purpose of either making
the work their own or preserving what they naturally regard as divine
revelations, later writers reverently and accurately copy whatever
marks they find upon the papyrus or parchment, to the unspeakable
enhancement of the lucidity of the thought and value of the work.
Writers contemporary with the copyists naturally avail themselves of
the obvious advantages of these marks in their own work, and with such
assistance as the flies of their own household may be willing to
grant, frequently rival and sometimes surpass the older compositions,
in respect at least of punctuation, which is no small glory.  Fully to
understand the important services that flies perform to literature it
is only necessary to lay a page of some popular novelist alongside a
saucer of cream-and-molasses in a sunny room and observe "how the wit
brightens and the style refines" in accurate proportion to the
duration of exposure.

\paragraph{FOLLY, n.}  That "gift and faculty divine" whose creative and
controlling energy inspires Man's mind, guides his actions and adorns
his life.

\begin{quote}   Folly! although Erasmus praised thee once \\
      In a thick volume, and all authors known, \\
      If not thy glory yet thy power have shown, \\
  Deign to take homage from thy son who hunts \\
  Through all thy maze his brothers, fool and dunce, \\
      To mend their lives and to sustain his own, \\
      However feebly be his arrows thrown, \\
 \\
  Howe'er each hide the flying weapons blunts. \\
  All-Father Folly! be it mine to raise, \\
      With lusty lung, here on his western strand \\
      With all thine offspring thronged from every land, \\
  Thyself inspiring me, the song of praise. \\
  And if too weak, I'll hire, to help me bawl, \\
  Dick Watson Gilder, gravest of us all. \\
 \\
Aramis Loto Frope \end{quote}


\paragraph{FOOL, n.}  A person who pervades the domain of intellectual speculation
and diffuses himself through the channels of moral activity.  He is
omnific, omniform, omnipercipient, omniscience, omnipotent.  He it was
who invented letters, printing, the railroad, the steamboat, the
telegraph, the platitude and the circle of the sciences.  He created
patriotism and taught the nations war -- founded theology, philosophy,
law, medicine and Chicago.  He established monarchical and republican
government.  He is from everlasting to everlasting -- such as
creation's dawn beheld he fooleth now.  In the morning of time he sang
upon primitive hills, and in the noonday of existence headed the
procession of being.  His grandmotherly hand was warmly tucked-in the
set sun of civilization, and in the twilight he prepares Man's evening
meal of milk-and-morality and turns down the covers of the universal
grave.  And after the rest of us shall have retired for the night of
eternal oblivion he will sit up to write a history of human
civilization.

\paragraph{FORCE, n.}

\begin{quote}   "Force is but might," the teacher said -- \\
      "That definition's just." \\
  The boy said naught but through instead, \\
  Remembering his pounded head: \\
      "Force is not might but must!"  \end{quote}

\paragraph{FOREFINGER, n.}  The finger commonly used in pointing out two
malefactors.

\paragraph{FOREORDINATION, n.}  This looks like an easy word to define, but when I
consider that pious and learned theologians have spent long lives in
explaining it, and written libraries to explain their explanations;
when I remember the nations have been divided and bloody battles
caused by the difference between foreordination and predestination,
and that millions of treasure have been expended in the effort to
prove and disprove its compatibility with freedom of the will and the
efficacy of prayer, praise, and a religious life, -- recalling these
awful facts in the history of the word, I stand appalled before the
mighty problem of its signification, abase my spiritual eyes, fearing
to contemplate its portentous magnitude, reverently uncover and humbly
refer it to His Eminence Cardinal Gibbons and His Grace Bishop Potter.

\paragraph{FORGETFULNESS, n.}  A gift of God bestowed upon doctors in compensation
for their destitution of conscience.

\paragraph{FORK, n.}  An instrument used chiefly for the purpose of putting dead
animals into the mouth.  Formerly the knife was employed for this
purpose, and by many worthy persons is still thought to have many
advantages over the other tool, which, however, they do not altogether
reject, but use to assist in charging the knife.  The immunity of
these persons from swift and awful death is one of the most striking
proofs of God's mercy to those that hate Him.

\paragraph{FORMA PAUPERIS.}  [Latin]  In the character of a poor person -- a
method by which a litigant without money for lawyers is considerately
permitted to lose his case.

\begin{quote}   When Adam long ago in Cupid's awful court \\
      (For Cupid ruled ere Adam was invented) \\
  Sued for Eve's favor, says an ancient law report, \\
      He stood and pleaded unhabilimented. \\
 \\
  "You sue {\em in forma pauperis}, I see," Eve cried; \\
      "Actions can't here be that way prosecuted." \\
  So all poor Adam's motions coldly were denied: \\
      He went away -- as he had come -- nonsuited. \\
 \\
G.J. \end{quote}


\paragraph{FRANKALMOIGNE, n.}  The tenure by which a religious corporation holds
lands on condition of praying for the soul of the donor.  In mediaeval
times many of the wealthiest fraternities obtained their estates in
this simple and cheap manner, and once when Henry VIII of England sent
an officer to confiscate certain vast possessions which a fraternity
of monks held by frankalmoigne, "What!" said the Prior, "would you
master stay our benefactor's soul in Purgatory?"  "Ay," said the
officer, coldly, "an ye will not pray him thence for naught he must
e'en roast."  "But look you, my son," persisted the good man, "this
act hath rank as robbery of God!"  "Nay, nay, good father, my master
the king doth but deliver him from the manifold temptations of too
great wealth."

\paragraph{FREEBOOTER, n.}  A conqueror in a small way of business, whose
annexations lack of the sanctifying merit of magnitude.

\paragraph{FREEDOM, n.}  Exemption from the stress of authority in a beggarly half
dozen of restraint's infinite multitude of methods.  A political
condition that every nation supposes itself to enjoy in virtual
monopoly.  Liberty.  The distinction between freedom and liberty is
not accurately known; naturalists have never been able to find a
living specimen of either.

\begin{quote}   Freedom, as every schoolboy knows, \\
      Once shrieked as Kosciusko fell; \\
  On every wind, indeed, that blows \\
          I hear her yell. \\
 \\
  She screams whenever monarchs meet, \\
      And parliaments as well, \\
  To bind the chains about her feet \\
          And toll her knell. \\
 \\
  And when the sovereign people cast \\
      The votes they cannot spell, \\
  Upon the pestilential blast \\
          Her clamors swell. \\
 \\
  For all to whom the power's given \\
      To sway or to compel, \\
  Among themselves apportion Heaven \\
          And give her Hell. \\
 \\
Blary O'Gary \end{quote}


\paragraph{FREEMASONS, n.}  An order with secret rites, grotesque ceremonies and
fantastic costumes, which, originating in the reign of Charles II,
among working artisans of London, has been joined successively by the
dead of past centuries in unbroken retrogression until now it embraces
all the generations of man on the hither side of Adam and is drumming
up distinguished recruits among the pre-Creational inhabitants of
Chaos and Formless Void.  The order was founded at different times by
Charlemagne, Julius Caesar, Cyrus, Solomon, Zoroaster, Confucious,
Thothmes, and Buddha.  Its emblems and symbols have been found in the
Catacombs of Paris and Rome, on the stones of the Parthenon and the
Chinese Great Wall, among the temples of Karnak and Palmyra and in the
Egyptian Pyramids -- always by a Freemason.

\paragraph{FRIENDLESS, adj.}  Having no favors to bestow.  Destitute of fortune.
Addicted to utterance of truth and common sense.

\paragraph{FRIENDSHIP, n.}  A ship big enough to carry two in fair weather, but
only one in foul.

\begin{quote}   The sea was calm and the sky was blue; \\
  Merrily, merrily sailed we two. \\
      (High barometer maketh glad.) \\
  On the tipsy ship, with a dreadful shout, \\
  The tempest descended and we fell out. \\
      (O the walking is nasty bad!) \\
 \\
Armit Huff Bettle \end{quote}


\paragraph{FROG, n.}  A reptile with edible legs.  The first mention of frogs in
profane literature is in Homer's narrative of the war between them and
the mice.  Skeptical persons have doubted Homer's authorship of the
work, but the learned, ingenious and industrious Dr. Schliemann has
set the question forever at rest by uncovering the bones of the slain
frogs.  One of the forms of moral suasion by which Pharaoh was
besought to favor the Israelities was a plague of frogs, but Pharaoh,
who liked them {\em fricasees}, remarked, with truly oriental stoicism,
that he could stand it as long as the frogs and the Jews could; so the
programme was changed.  The frog is a diligent songster, having a good
voice but no ear.  The libretto of his favorite opera, as written by
Aristophanes, is brief, simple and effective -- "brekekex-koax"; the
music is apparently by that eminent composer, Richard Wagner.  Horses
have a frog in each hoof -- a thoughtful provision of nature, enabling
them to shine in a hurdle race.

\paragraph{FRYING-PAN, n.}  One part of the penal apparatus employed in that
punitive institution, a woman's kitchen.  The frying-pan was invented
by Calvin, and by him used in cooking span-long infants that had died
without baptism; and observing one day the horrible torment of a tramp
who had incautiously pulled a fried babe from the waste-dump and
devoured it, it occurred to the great divine to rob death of its
terrors by introducing the frying-pan into every household in Geneva.
Thence it spread to all corners of the world, and has been of
invaluable assistance in the propagation of his sombre faith.  The
following lines (said to be from the pen of his Grace Bishop Potter)
seem to imply that the usefulness of this utensil is not limited to
this world; but as the consequences of its employment in this life
reach over into the life to come, so also itself may be found on the
other side, rewarding its devotees:

\begin{quote}   Old Nick was summoned to the skies. \\
      Said Peter:  "Your intentions \\
  Are good, but you lack enterprise \\
      Concerning new inventions. \\
 \\
  "Now, broiling in an ancient plan \\
      Of torment, but I hear it \\
  Reported that the frying-pan \\
      Sears best the wicked spirit. \\
 \\
  "Go get one -- fill it up with fat -- \\
      Fry sinners brown and good in't." \\
  "I know a trick worth two o' that," \\
      Said Nick -- "I'll cook their food in't."  \end{quote}

\paragraph{FUNERAL, n.}  A pageant whereby we attest our respect for the dead by
enriching the undertaker, and strengthen our grief by an expenditure
that deepens our groans and doubles our tears.

\begin{quote}   The savage dies -- they sacrifice a horse \\
  To bear to happy hunting-grounds the corse. \\
  Our friends expire -- we make the money fly \\
  In hope their souls will chase it to the sky. \\
 \\
Jex Wopley \end{quote}


\paragraph{FUTURE, n.}  That period of time in which our affairs prosper, our
friends are true and our happiness is assured.



\section*{G}



\paragraph{GALLOWS, n.}  A stage for the performance of miracle plays, in which
the leading actor is translated to heaven.  In this country the
gallows is chiefly remarkable for the number of persons who escape it.

\begin{quote}   Whether on the gallows high \\
      Or where blood flows the reddest, \\
  The noblest place for man to die -- \\
      Is where he died the deadest. \\
 \\
(Old play) \end{quote}


\paragraph{GARGOYLE, n.}  A rain-spout projecting from the eaves of mediaeval
buildings, commonly fashioned into a grotesque caricature of some
personal enemy of the architect or owner of the building.  This was
especially the case in churches and ecclesiastical structures
generally, in which the gargoyles presented a perfect rogues' gallery
of local heretics and controversialists.  Sometimes when a new dean
and chapter were installed the old gargoyles were removed and others
substituted having a closer relation to the private animosities of the
new incumbents.

\paragraph{GARTHER, n.}  An elastic band intended to keep a woman from coming out
of her stockings and desolating the country.

\paragraph{GENEROUS, adj.}  Originally this word meant noble by birth and was
rightly applied to a great multitude of persons.  It now means noble
by nature and is taking a bit of a rest.

\paragraph{GENEALOGY, n.}  An account of one's descent from an ancestor who did
not particularly care to trace his own.

\paragraph{GENTEEL, adj.}  Refined, after the fashion of a gent.

\begin{quote}   Observe with care, my son, the distinction I reveal: \\
  A gentleman is gentle and a gent genteel. \\
  Heed not the definitions your "Unabridged" presents, \\
  For dictionary makers are generally gents. \\
 \\
G.J. \end{quote}


\paragraph{GEOGRAPHER, n.}  A chap who can tell you offhand the difference between
the outside of the world and the inside.

\begin{quote}   Habeam, geographer of wide reknown, \\
  Native of Abu-Keber's ancient town, \\
  In passing thence along the river Zam \\
  To the adjacent village of Xelam, \\
  Bewildered by the multitude of roads, \\
  Got lost, lived long on migratory toads, \\
  Then from exposure miserably died, \\
  And grateful travelers bewailed their guide. \\
 \\
Henry Haukhorn \end{quote}


\paragraph{GEOLOGY, n.}  The science of the earth's crust -- to which, doubtless,
will be added that of its interior whenever a man shall come up
garrulous out of a well.  The geological formations of the globe
already noted are catalogued thus:  The Primary, or lower one,
consists of rocks, bones or mired mules, gas-pipes, miners' tools,
antique statues minus the nose, Spanish doubloons and ancestors.  The
Secondary is largely made up of red worms and moles.  The Tertiary
comprises railway tracks, patent pavements, grass, snakes, mouldy
boots, beer bottles, tomato cans, intoxicated citizens, garbage,
anarchists, snap-dogs and fools.

\paragraph{GHOST, n.}  The outward and visible sign of an inward fear.

\begin{quote}           He saw a ghost. \\
  It occupied -- that dismal thing! -- \\
  The path that he was following. \\
  Before he'd time to stop and fly, \\
  An earthquake trifled with the eye \\
          That saw a ghost. \\
  He fell as fall the early good; \\
  Unmoved that awful vision stood. \\
  The stars that danced before his ken \\
  He wildly brushed away, and then \\
          He saw a post. \\
 \\
Jared Macphester \end{quote}


\subparagraph{}   Accounting for the uncommon behavior of ghosts, Heine mentions
somebody's ingenious theory to the effect that they are as much 
afraid of us as we of them.  Not quite, if I may judge from such
tables of comparative speed as I am able to compile from memories of
my own experience.
\subparagraph{}   There is one insuperable obstacle to a belief in ghosts.  A ghost
never comes naked:  he appears either in a winding-sheet or "in his 
habit as he lived."  To believe in him, then, is to believe that not
only have the dead the power to make themselves visible after there is
nothing left of them, but that the same power inheres in textile
fabrics.  Supposing the products of the loom to have this ability,
what object would they have in exercising it?  And why does not the
apparition of a suit of clothes sometimes walk abroad without a ghost
in it?  These be riddles of significance.  They reach away down and
get a convulsive grip on the very tap-root of this flourishing faith.

\paragraph{GHOUL, n.}  A demon addicted to the reprehensible habit of devouring
the dead.  The existence of ghouls has been disputed by that class of
controversialists who are more concerned to deprive the world of
comforting beliefs than to give it anything good in their place.  In
1640 Father Secchi saw one in a cemetery near Florence and frightened
it away with the sign of the cross.  He describes it as gifted with
many heads an an uncommon allowance of limbs, and he saw it in more
than one place at a time.  The good man was coming away from dinner at
the time and explains that if he had not been "heavy with eating" he
would have seized the demon at all hazards.  Atholston relates that a
ghoul was caught by some sturdy peasants in a churchyard at Sudbury
and ducked in a horsepond.  (He appears to think that so distinguished
a criminal should have been ducked in a tank of rosewater.)  The water
turned at once to blood "and so contynues unto ys daye."  The pond has
since been bled with a ditch.  As late as the beginning of the
fourteenth century a ghoul was cornered in the crypt of the cathedral
at Amiens and the whole population surrounded the place.  Twenty armed
men with a priest at their head, bearing a crucifix, entered and
captured the ghoul, which, thinking to escape by the stratagem, had
transformed itself to the semblance of a well known citizen, but was
nevertheless hanged, drawn and quartered in the midst of hideous
popular orgies.  The citizen whose shape the demon had assumed was so
affected by the sinister occurrence that he never again showed himself
in Amiens and his fate remains a mystery.

\paragraph{GLUTTON, n.}  A person who escapes the evils of moderation by
committing dyspepsia.

\paragraph{GNOME, n.}  In North-European mythology, a dwarfish imp inhabiting the
interior parts of the earth and having special custody of mineral
treasures.  Bjorsen, who died in 1765, says gnomes were common enough
in the southern parts of Sweden in his boyhood, and he frequently saw
them scampering on the hills in the evening twilight.  Ludwig
Binkerhoof saw three as recently as 1792, in the Black Forest, and
Sneddeker avers that in 1803 they drove a party of miners out of a
Silesian mine.  Basing our computations upon data supplied by these
statements, we find that the gnomes were probably extinct as early as
1764.

\paragraph{GNOSTICS, n.}  A sect of philosophers who tried to engineer a fusion
between the early Christians and the Platonists.  The former would not
go into the caucus and the combination failed, greatly to the chagrin
of the fusion managers.

\paragraph{GNU, n.}  An animal of South Africa, which in its domesticated state
resembles a horse, a buffalo and a stag.  In its wild condition it is
something like a thunderbolt, an earthquake and a cyclone.

\begin{quote}   A hunter from Kew caught a distant view \\
      Of a peacefully meditative gnu, \\
  And he said:  "I'll pursue, and my hands imbrue \\
      In its blood at a closer interview." \\
  But that beast did ensue and the hunter it threw \\
      O'er the top of a palm that adjacent grew; \\
  And he said as he flew:  "It is well I withdrew \\
      Ere, losing my temper, I wickedly slew \\
      That really meritorious gnu." \\
 \\
Jarn Leffer \end{quote}


\paragraph{GOOD, adj.}  Sensible, madam, to the worth of this present writer.
Alive, sir, to the advantages of letting him alone.

\paragraph{GOOSE, n.}  A bird that supplies quills for writing.  These, by some
occult process of nature, are penetrated and suffused with various
degrees of the bird's intellectual energies and emotional character,
so that when inked and drawn mechanically across paper by a person
called an "author," there results a very fair and accurate transcript
of the fowl's thought and feeling.  The difference in geese, as
discovered by this ingenious method, is considerable:  many are found
to have only trivial and insignificant powers, but some are seen to be
very great geese indeed.

\paragraph{GORGON, n.}

\begin{quote}   The Gorgon was a maiden bold \\
  Who turned to stone the Greeks of old \\
  That looked upon her awful brow. \\
  We dig them out of ruins now, \\
  And swear that workmanship so bad \\
  Proves all the ancient sculptors mad.  \end{quote}

\paragraph{GOUT, n.}  A physician's name for the rheumatism of a rich patient.

\paragraph{GRACES, n.}  Three beautiful goddesses, Aglaia, Thalia and Euphrosyne,
who attended upon Venus, serving without salary.  They were at no
expense for board and clothing, for they ate nothing to speak of and
dressed according to the weather, wearing whatever breeze happened to
be blowing.

\paragraph{GRAMMAR, n.}  A system of pitfalls thoughtfully prepared for the feet
for the self-made man, along the path by which he advances to
distinction.

\paragraph{GRAPE, n.}

\begin{quote}   Hail noble fruit! -- by Homer sung, \\
      Anacreon and Khayyam; \\
  Thy praise is ever on the tongue \\
      Of better men than I am. \\
 \\
  The lyre in my hand has never swept, \\
      The song I cannot offer: \\
  My humbler service pray accept -- \\
      I'll help to kill the scoffer. \\
  The water-drinkers and the cranks \\
      Who load their skins with liquor -- \\
  I'll gladly bear their belly-tanks \\
      And tap them with my sticker. \\
 \\
  Fill up, fill up, for wisdom cools \\
      When e'er we let the wine rest. \\
  Here's death to Prohibition's fools, \\
      And every kind of vine-pest! \\
 \\
Jamrach Holobom \end{quote}


\paragraph{GRAPESHOT, n.}  An argument which the future is preparing in answer to
the demands of American Socialism.

\paragraph{GRAVE, n.}  A place in which the dead are laid to await the coming of
the medical student.

\begin{quote}   Beside a lonely grave I stood -- \\
      With brambles 'twas encumbered; \\
  The winds were moaning in the wood, \\
      Unheard by him who slumbered, \\
 \\
  A rustic standing near, I said: \\
      "He cannot hear it blowing!" \\
  "'Course not," said he:  "the feller's dead -- \\
      He can't hear nowt [sic] that's going." \\
 \\
  "Too true," I said; "alas, too true -- \\
      No sound his sense can quicken!" \\
  "Well, mister, wot is that to you? -- \\
      The deadster ain't a-kickin'." \\
 \\
  I knelt and prayed:  "O Father, smile \\
      On him, and mercy show him!" \\
  That countryman looked on the while, \\
      And said:  "Ye didn't know him." \\
 \\
Pobeter Dunko \end{quote}


\paragraph{GRAVITATION, n.}  The tendency of all bodies to approach one another
with a strength proportion to the quantity of matter they contain --
the quantity of matter they contain being ascertained by the strength
of their tendency to approach one another.  This is a lovely and
edifying illustration of how science, having made A the proof of B,
makes B the proof of A.

\paragraph{GREAT, adj.}

\begin{quote}   "I'm great," the Lion said -- "I reign \\
  The monarch of the wood and plain!" \\
 \\
  The Elephant replied:  "I'm great -- \\
  No quadruped can match my weight!" \\
 \\
  "I'm great -- no animal has half \\
  So long a neck!" said the Giraffe. \\
 \\
  "I'm great," the Kangaroo said -- "see \\
  My femoral muscularity!" \\
 \\
  The 'Possum said:  "I'm great -- behold, \\
  My tail is lithe and bald and cold!" \\
 \\
  An Oyster fried was understood \\
  To say:  "I'm great because I'm good!" \\
 \\
  Each reckons greatness to consist \\
  In that in which he heads the list, \\
 \\
  And Vierick thinks he tops his class \\
  Because he is the greatest ass. \\
 \\
Arion Spurl Doke \end{quote}


\paragraph{GUILLOTINE, n.}  A machine which makes a Frenchman shrug his shoulders
with good reason.
\subparagraph{}   In his great work on {\em Divergent Lines of Racial Evolution}, the
learned Professor Brayfugle argues from the prevalence of this gesture
-- the shrug -- among Frenchmen, that they are descended from turtles
and it is simply a survival of the habit of retracing the head inside
the shell.  It is with reluctance that I differ with so eminent an
authority, but in my judgment (as more elaborately set forth and
enforced in my work entitled {\em Hereditary Emotions}~-- lib.~II, c.~XI)
the shrug is a poor foundation upon which to build so important a
theory, for previously to the Revolution the gesture was unknown.  I
have not a doubt that it is directly referable to the terror inspired
by the guillotine during the period of that instrument's activity.

\paragraph{GUNPOWDER, n.}  An agency employed by civilized nations for the
settlement of disputes which might become troublesome if left
unadjusted.  By most writers the invention of gunpowder is ascribed to
the Chinese, but not upon very convincing evidence.  Milton says it
was invented by the devil to dispel angels with, and this opinion
seems to derive some support from the scarcity of angels.  Moreover,
it has the hearty concurrence of the Hon. James Wilson, Secretary of
Agriculture.
\subparagraph{}   Secretary Wilson became interested in gunpowder through an event
that occurred on the Government experimental farm in the District of 
Columbia.  One day, several years ago, a rogue imperfectly reverent of
the Secretary's profound attainments and personal character presented
him with a sack of gunpowder, representing it as the sed of the
{\em Flashawful flabbergastor}, a Patagonian cereal of great commercial
value, admirably adapted to this climate.  The good Secretary was
instructed to spill it along in a furrow and afterward inhume it with
soil.  This he at once proceeded to do, and had made a continuous line
of it all the way across a ten-acre field, when he was made to look
backward by a shout from the generous donor, who at once dropped a
lighted match into the furrow at the starting-point.  Contact with the
earth had somewhat dampened the powder, but the startled functionary
saw himself pursued by a tall moving pillar of fire and smoke and
fierce evolution.  He stood for a moment paralyzed and speechless,
then he recollected an engagement and, dropping all, absented himself
thence with such surprising celerity that to the eyes of spectators
along the route selected he appeared like a long, dim streak
prolonging itself with inconceivable rapidity through seven villages,
and audibly refusing to be comforted.  "Great Scott! what is that?"
cried a surveyor's chainman, shading his eyes and gazing at the fading
line of agriculturist which bisected his visible horizon.  "That,"
said the surveyor, carelessly glancing at the phenomenon and again
centering his attention upon his instrument, "is the Meridian of
Washington."



\section*{H}



\paragraph{HABEAS CORPUS.}  A writ by which a man may be taken out of jail when
confined for the wrong crime.

\paragraph{HABIT, n.}  A shackle for the free.

\paragraph{HADES, n.}  The lower world; the residence of departed spirits; the
place where the dead live.
\subparagraph{}   Among the ancients the idea of Hades was not synonymous with our 
Hell, many of the most respectable men of antiquity residing there in 
a very comfortable kind of way.  Indeed, the Elysian Fields themselves
were a part of Hades, though they have since been removed to Paris.
When the Jacobean version of the New Testament was in process of
evolution the pious and learned men engaged in the work insisted by a
majority vote on translating the Greek word "Aides" as "Hell"; but a
conscientious minority member secretly possessed himself of the record
and struck out the objectional word wherever he could find it.  At the
next meeting, the Bishop of Salisbury, looking over the work, suddenly
sprang to his feet and said with considerable excitement:  "Gentlemen,
somebody has been razing 'Hell' here!"  Years afterward the good
prelate's death was made sweet by the reflection that he had been the
means (under Providence) of making an important, serviceable and
immortal addition to the phraseology of the English tongue.

\paragraph{HAG, n.}  An elderly lady whom you do not happen to like; sometimes
called, also, a hen, or cat.  Old witches, sorceresses, etc., were
called hags from the belief that their heads were surrounded by a kind
of baleful lumination or nimbus -- hag being the popular name of that
peculiar electrical light sometimes observed in the hair.  At one time
hag was not a word of reproach:  Drayton speaks of a "beautiful hag,
all smiles," much as Shakespeare said, "sweet wench."  It would not
now be proper to call your sweetheart a hag -- that compliment is
reserved for the use of her grandchildren.

\paragraph{HALF, n.}  One of two equal parts into which a thing may be divided, or
considered as divided.  In the fourteenth century a heated discussion
arose among theologists and philosophers as to whether Omniscience
could part an object into three halves; and the pious Father
Aldrovinus publicly prayed in the cathedral at Rouen that God would
demonstrate the affirmative of the proposition in some signal and
unmistakable way, and particularly (if it should please Him) upon the
body of that hardy blasphemer, Manutius Procinus, who maintained the
negative.  Procinus, however, was spared to die of the bite of a
viper.

\paragraph{HALO, n.}  Properly, a luminous ring encircling an astronomical body,
but not infrequently confounded with "aureola," or "nimbus," a
somewhat similar phenomenon worn as a head-dress by divinities and
saints.  The halo is a purely optical illusion, produced by moisture
in the air, in the manner of a rainbow; but the aureola is conferred
as a sign of superior sanctity, in the same way as a bishop's mitre,
or the Pope's tiara.  In the painting of the Nativity, by Szedgkin, a
pious artist of Pesth, not only do the Virgin and the Child wear the
nimbus, but an ass nibbling hay from the sacred manger is similarly
decorated and, to his lasting honor be it said, appears to bear his
unaccustomed dignity with a truly saintly grace.

\paragraph{HAND, n.}  A singular instrument worn at the end of the human arm and
commonly thrust into somebody's pocket.

\paragraph{HANDKERCHIEF, n.}  A small square of silk or linen, used in various
ignoble offices about the face and especially serviceable at funerals
to conceal the lack of tears.  The handkerchief is of recent
invention; our ancestors knew nothing of it and intrusted its duties
to the sleeve.  Shakespeare's introducing it into the play of
"Othello" is an anachronism:  Desdemona dried her nose with her skirt,
as Dr. Mary Walker and other reformers have done with their coattails
in our own day -- an evidence that revolutions sometimes go backward.

\paragraph{HANGMAN, n.}  An officer of the law charged with duties of the highest
dignity and utmost gravity, and held in hereditary disesteem by a
populace having a criminal ancestry.  In some of the American States
his functions are now performed by an electrician, as in New Jersey,
where executions by electricity have recently been ordered -- the
first instance known to this lexicographer of anybody questioning the
expediency of hanging Jerseymen.

\paragraph{HAPPINESS, n.}  An agreeable sensation arising from contemplating the
misery of another.

\paragraph{HARANGUE, n.}  A speech by an opponent, who is known as an harrangue-
outang.

\paragraph{HARBOR, n.}  A place where ships taking shelter from stores are exposed
to the fury of the customs.

\paragraph{HARMONISTS, n.}  A sect of Protestants, now extinct, who came from
Europe in the beginning of the last century and were distinguished for
the bitterness of their internal controversies and dissensions.

\paragraph{HASH, x.}  There is no definition for this word -- nobody knows what
hash is.

\paragraph{HATCHET, n.}  A young axe, known among Indians as a Thomashawk.

\begin{quote}   "O bury the hatchet, irascible Red, \\
  For peace is a blessing," the White Man said. \\
      The Savage concurred, and that weapon interred, \\
  With imposing rites, in the White Man's head. \\
 \\
John Lukkus \end{quote}


\paragraph{HATRED, n.}  A sentiment appropriate to the occasion of another's
superiority.

\paragraph{HEAD-MONEY, n.}  A capitation tax, or poll-tax.

\begin{quote}   In ancient times there lived a king \\
  Whose tax-collectors could not wring \\
  From all his subjects gold enough \\
  To make the royal way less rough. \\
  For pleasure's highway, like the dames \\
  Whose premises adjoin it, claims \\
  Perpetual repairing.  So \\
  The tax-collectors in a row \\
  Appeared before the throne to pray \\
  Their master to devise some way \\
  To swell the revenue.  "So great," \\
  Said they, "are the demands of state \\
  A tithe of all that we collect \\
  Will scarcely meet them.  Pray reflect: \\
  How, if one-tenth we must resign, \\
  Can we exist on t'other nine?" \\
  The monarch asked them in reply: \\
  "Has it occurred to you to try \\
  The advantage of economy?" \\
  "It has," the spokesman said:  "we sold \\
  All of our gray garrotes of gold; \\
  With plated-ware we now compress \\
  The necks of those whom we assess. \\
  Plain iron forceps we employ \\
  To mitigate the miser's joy \\
  Who hoards, with greed that never tires, \\
  That which your Majesty requires." \\
  Deep lines of thought were seen to plow \\
  Their way across the royal brow. \\
  "Your state is desperate, no question; \\
  Pray favor me with a suggestion." \\
  "O King of Men," the spokesman said, \\
  "If you'll impose upon each head \\
  A tax, the augmented revenue \\
  We'll cheerfully divide with you." \\
  As flashes of the sun illume \\
  The parted storm-cloud's sullen gloom, \\
  The king smiled grimly.  "I decree \\
  That it be so -- and, not to be \\
  In generosity outdone, \\
  Declare you, each and every one, \\
  Exempted from the operation \\
  Of this new law of capitation. \\
  But lest the people censure me \\
  Because they're bound and you are free, \\
  'Twere well some clever scheme were laid \\
  By you this poll-tax to evade. \\
  I'll leave you now while you confer \\
  With my most trusted minister." \\
  The monarch from the throne-room walked \\
  And straightway in among them stalked \\
  A silent man, with brow concealed, \\
  Bare-armed -- his gleaming axe revealed! \\
 \\
G.J. \end{quote}


\paragraph{HEARSE, n.}  Death's baby-carriage.

\paragraph{HEART, n.}  An automatic, muscular blood-pump.  Figuratively, this
useful organ is said to be the seat of emotions and sentiments -- a
very pretty fancy which, however, is nothing but a survival of a once
universal belief.  It is now known that the sentiments and emotions
reside in the stomach, being evolved from food by chemical action of
the gastric fluid.  The exact process by which a beefsteak becomes a
feeling -- tender or not, according to the age of the animal from
which it was cut; the successive stages of elaboration through which a
caviar sandwich is transmuted to a quaint fancy and reappears as a
pungent epigram; the marvelous functional methods of converting a
hard-boiled egg into religious contrition, or a cream-puff into a sigh
of sensibility -- these things have been patiently ascertained by M.
Pasteur, and by him expounded with convincing lucidity.  (See, also,
my monograph, {\em The Essential Identity of the Spiritual Affections and
Certain Intestinal Gases Freed in Digestion} -- 4to, 687 pp.)  In a
scientific work entitled, I believe, {\em Delectatio Demonorum} (John
Camden Hotton, London, 1873) this view of the sentiments receives a
striking illustration; and for further light consult Professor Dam's
famous treatise on {\em Love as a Product of Alimentary Maceration}.

\paragraph{HEAT, n.}

\begin{quote}   Heat, says Professor Tyndall, is a mode \\
      Of motion, but I know now how he's proving \\
  His point; but this I know -- hot words bestowed \\
      With skill will set the human fist a-moving, \\
  And where it stops the stars burn free and wild. \\
  {\em Crede expertum} -- I have seen them, child. \\
 \\
Gorton Swope \end{quote}


\paragraph{HEATHEN, n.}  A benighted creature who has the folly to worship
something that he can see and feel.  According to Professor Howison,
of the California State University, Hebrews are heathens.

\begin{quote}   "The Hebrews are heathens!" says Howison.  He's \\
      A Christian philosopher.  I'm \\
  A scurril agnostical chap, if you please, \\
      Addicted too much to the crime \\
      Of religious discussion in my rhyme. \\
 \\
  Though Hebrew and Howison cannot agree \\
      On a {\em modus vivendi} -- not they! -- \\
  Yet Heaven has had the designing of me, \\
      And I haven't been reared in a way \\
      To joy in the thick of the fray. \\
 \\
  For this of my creed is the soul and the gist, \\
      And the truth of it I aver: \\
  Who differs from me in his faith is an 'ist, \\
      And 'ite, an 'ie, or an 'er -- \\
      And I'm down upon him or her! \\
 \\
  Let Howison urge with perfunctory chin \\
      Toleration -- that's all very well, \\
  But a roast is "nuts" to his nostril thin, \\
      And he's running -- I know by the smell -- \\
      A secret and personal Hell! \\
 \\
Bissell Gip \end{quote}


\paragraph{HEAVEN, n.}  A place where the wicked cease from troubling you with
talk of their personal affairs, and the good listen with attention
while you expound your own.

\paragraph{HEBREW, n.}  A male Jew, as distinguished from the Shebrew, an
altogether superior creation.

\paragraph{HELPMATE, n.}  A wife, or bitter half.

\begin{quote}   "Now, why is yer wife called a helpmate, Pat?" \\
      Says the priest.  "Since the time 'o yer wooin' \\
  She's niver [sic] assisted in what ye were at -- \\
      For it's naught ye are ever doin'." \\
 \\
  "That's true of yer Riverence [sic]," Patrick replies, \\
      And no sign of contrition envices; \\
  "But, bedad, it's a fact which the word implies, \\
      For she helps to mate the expinses [sic]!" \\
 \\
Marley Wottel \end{quote}


\paragraph{HEMP, n.}  A plant from whose fibrous bark is made an article of
neckwear which is frequently put on after public speaking in the open
air and prevents the wearer from taking cold.

\paragraph{HERMIT, n.}  A person whose vices and follies are not sociable.

\paragraph{HERS, pron.}  His.

\paragraph{HIBERNATE, v.i.}  To pass the winter season in domestic seclusion.
There have been many singular popular notions about the hibernation of
various animals.  Many believe that the bear hibernates during the
whole winter and subsists by mechanically sucking its paws.  It is
admitted that it comes out of its retirement in the spring so lean
that it had to try twice before it can cast a shadow.  Three or four
centuries ago, in England, no fact was better attested than that
swallows passed the winter months in the mud at the bottom of their
brooks, clinging together in globular masses.  They have apparently
been compelled to give up the custom and account of the foulness of
the brooks.  Sotus Ecobius discovered in Central Asia a whole nation
of people who hibernate.  By some investigators, the fasting of Lent
is supposed to have been originally a modified form of hibernation, to
which the Church gave a religious significance; but this view was
strenuously opposed by that eminent authority, Bishop Kip, who did not
wish any honors denied to the memory of the Founder of his family.

\paragraph{HIPPOGRIFF, n.}  An animal (now extinct) which was half horse and half
griffin.  The griffin was itself a compound creature, half lion and
half eagle.  The hippogriff was actually, therefore, a one-quarter
eagle, which is two dollars and fifty cents in gold.  The study of
zoology is full of surprises.

\paragraph{HISTORIAN, n.}  A broad-gauge gossip.

\paragraph{HISTORY, n.}  An account mostly false, of events mostly unimportant,
which are brought about by rulers mostly knaves, and soldiers mostly
fools.

\begin{quote}   Of Roman history, great Niebuhr's shown \\
  'Tis nine-tenths lying.  Faith, I wish 'twere known, \\
  Ere we accept great Niebuhr as a guide, \\
  Wherein he blundered and how much he lied. \\
 \\
Salder Bupp \end{quote}


\paragraph{HOG, n.}  A bird remarkable for the catholicity of its appetite and
serving to illustrate that of ours.  Among the Mahometans and Jews,
the hog is not in favor as an article of diet, but is respected for
the delicacy and the melody of its voice.  It is chiefly as a songster
that the fowl is esteemed; the cage of him in full chorus has been
known to draw tears from two persons at once.  The scientific name of
this dicky-bird is {\em Porcus Rockefelleri}.  Mr. Rockefeller did not
discover the hog, but it is considered his by right of resemblance.

\paragraph{HOMOEOPATHIST, n.}  The humorist of the medical profession.

\paragraph{HOMOEOPATHY, n.}  A school of medicine midway between Allopathy and
Christian Science.  To the last both the others are distinctly
inferior, for Christian Science will cure imaginary diseases, and they
can not.

\paragraph{HOMICIDE, n.}  The slaying of one human being by another.  There are
four kinds of homocide:  felonious, excusable, justifiable, and
praiseworthy, but it makes no great difference to the person slain
whether he fell by one kind or another -- the classification is for
advantage of the lawyers.

\paragraph{HOMILETICS, n.}  The science of adapting sermons to the spiritual
needs, capacities and conditions of the congregation.

\begin{quote}   So skilled the parson was in homiletics \\
  That all his normal purges and emetics \\
  To medicine the spirit were compounded \\
  With a most just discrimination founded \\
  Upon a rigorous examination \\
  Of tongue and pulse and heart and respiration. \\
  Then, having diagnosed each one's condition, \\
  His scriptural specifics this physician \\
  Administered -- his pills so efficacious \\
  And pukes of disposition so vivacious \\
  That souls afflicted with ten kinds of Adam \\
  Were convalescent ere they knew they had 'em. \\
  But Slander's tongue -- itself all coated -- uttered \\
  Her bilious mind and scandalously muttered \\
  That in the case of patients having money \\
  The pills were sugar and the pukes were honey. \\
 \\
{\em Biography of Bishop Potter} \end{quote}


\paragraph{HONORABLE, adj.}  Afflicted with an impediment in one's reach.  In
legislative bodies it is customary to mention all members as
honorable; as, "the honorable gentleman is a scurvy cur."

\paragraph{HOPE, n.}  Desire and expectation rolled into one.

\begin{quote}   Delicious Hope! when naught to man it left -- \\
  Of fortune destitute, of friends bereft; \\
  When even his dog deserts him, and his goat \\
  With tranquil disaffection chews his coat \\
  While yet it hangs upon his back; then thou, \\
  The star far-flaming on thine angel brow, \\
  Descendest, radiant, from the skies to hint \\
  The promise of a clerkship in the Mint. \\
 \\
Fogarty Weffing \end{quote}


\paragraph{HOSPITALITY, n.}  The virtue which induces us to feed and lodge certain
persons who are not in need of food and lodging.

\paragraph{HOSTILITY, n.}  A peculiarly sharp and specially applied sense of the
earth's overpopulation.  Hostility is classified as active and
passive; as (respectively) the feeling of a woman for her female
friends, and that which she entertains for all the rest of her sex.

\paragraph{HOURI, n.}  A comely female inhabiting the Mohammedan Paradise to make
things cheery for the good Mussulman, whose belief in her existence
marks a noble discontent with his earthly spouse, whom he denies a
soul.  By that good lady the Houris are said to be held in deficient
esteem.

\paragraph{HOUSE, n.}  A hollow edifice erected for the habitation of man, rat,
mouse, beetle, cockroach, fly, mosquito, flea, bacillus and microbe.
{\em House of Correction}, a place of reward for political and personal
service, and for the detention of offenders and appropriations.
{\em House of God}, a building with a steeple and a mortgage on it.
{\em House-dog}, a pestilent beast kept on domestic premises to insult
persons passing by and appal the hardy visitor.  {\em House-maid}, a
youngerly person of the opposing sex employed to be variously
disagreeable and ingeniously unclean in the station in which it has
pleased God to place her.

\paragraph{HOUSELESS, adj.}  Having paid all taxes on household goods.

\paragraph{HOVEL, n.}  The fruit of a flower called the Palace.

\begin{quote}       Twaddle had a hovel, \\
          Twiddle had a palace; \\
      Twaddle said:  "I'll grovel \\
          Or he'll think I bear him malice" -- \\
  A sentiment as novel \\
      As a castor on a chalice. \\
 \\
      Down upon the middle \\
          Of his legs fell Twaddle \\
      And astonished Mr. Twiddle, \\
          Who began to lift his noddle. \\
      Feed upon the fiddle- \\
          Faddle flummery, unswaddle \\
  A new-born self-sufficiency and think himself a [mockery.] \\
 \\
G.J. \end{quote}


\paragraph{HUMANITY, n.}  The human race, collectively, exclusive of the
anthropoid poets.

\paragraph{HUMORIST, n.}  A plague that would have softened down the hoar
austerity of Pharaoh's heart and persuaded him to dismiss Israel with
his best wishes, cat-quick.

\begin{quote}   Lo! the poor humorist, whose tortured mind \\
  See jokes in crowds, though still to gloom inclined -- \\
  Whose simple appetite, untaught to stray, \\
  His brains, renewed by night, consumes by day. \\
  He thinks, admitted to an equal sty, \\
  A graceful hog would bear his company. \\
 \\
Alexander Poke \end{quote}


\paragraph{HURRICANE, n.}  An atmospheric demonstration once very common but now
generally abandoned for the tornado and cyclone.  The hurricane is
still in popular use in the West Indies and is preferred by certain
old-fashioned sea-captains.  It is also used in the construction of
the upper decks of steamboats, but generally speaking, the hurricane's
usefulness has outlasted it.

\paragraph{HURRY, n.}  The dispatch of bunglers.

\paragraph{HUSBAND, n.}  One who, having dined, is charged with the care of the
plate.

\paragraph{HYBRID, n.}  A pooled issue.

\paragraph{HYDRA, n.}  A kind of animal that the ancients catalogued under many
heads.

\paragraph{HYENA, n.}  A beast held in reverence by some oriental nations from its
habit of frequenting at night the burial-places of the dead.  But the
medical student does that.

\paragraph{HYPOCHONDRIASIS, n.}  Depression of one's own spirits.

\begin{quote}   Some heaps of trash upon a vacant lot \\
  Where long the village rubbish had been shot \\
  Displayed a sign among the stuff and stumps -- \\
  "Hypochondriasis."  It meant The Dumps. \\
 \\
Bogul S. Purvy \end{quote}


\paragraph{HYPOCRITE, n.}  One who, profession virtues that he does not respect
secures the advantage of seeming to be what he despises.



\section*{I}



I is the first letter of the alphabet, the first word of the language,
the first thought of the mind, the first object of affection.  In
grammar it is a pronoun of the first person and singular number.  Its
plural is said to be {\em We}, but how there can be more than one myself
is doubtless clearer the grammarians than it is to the author of this
incomparable dictionary.  Conception of two myselfs is difficult, but
fine.  The frank yet graceful use of "I" distinguishes a good writer
from a bad; the latter carries it with the manner of a thief trying to
cloak his loot.

\paragraph{ICHOR, n.}  A fluid that serves the gods and goddesses in place of
blood.

\begin{quote}   Fair Venus, speared by Diomed, \\
  Restrained the raging chief and said: \\
  "Behold, rash mortal, whom you've bled -- \\
  Your soul's stained white with ichorshed!" \\
 \\
Mary Doke \end{quote}


\paragraph{ICONOCLAST, n.}  A breaker of idols, the worshipers whereof are
imperfectly gratified by the performance, and most strenuously protest
that he unbuildeth but doth not reedify, that he pulleth down but
pileth not up.  For the poor things would have other idols in place of
those he thwacketh upon the mazzard and dispelleth.  But the
iconoclast saith:  "Ye shall have none at all, for ye need them not;
and if the rebuilder fooleth round hereabout, behold I will depress
the head of him and sit thereon till he squawk it."

\paragraph{IDIOT, n.}  A member of a large and powerful tribe whose influence in
human affairs has always been dominant and controlling.  The Idiot's
activity is not confined to any special field of thought or action,
but "pervades and regulates the whole."  He has the last word in
everything; his decision is unappealable.  He sets the fashions and
opinion of taste, dictates the limitations of speech and circumscribes
conduct with a dead-line.

\paragraph{IDLENESS, n.}  A model farm where the devil experiments with seeds of
new sins and promotes the growth of staple vices.

\paragraph{IGNORAMUS, n.}  A person unacquainted with certain kinds of knowledge
familiar to yourself, and having certain other kinds that you know
nothing about.

\begin{quote}   Dumble was an ignoramus, \\
  Mumble was for learning famous. \\
  Mumble said one day to Dumble: \\
  "Ignorance should be more humble. \\
  Not a spark have you of knowledge \\
  That was got in any college." \\
  Dumble said to Mumble:  "Truly \\
  You're self-satisfied unduly. \\
  Of things in college I'm denied \\
  A knowledge -- you of all beside." \\
 \\
Borelli \end{quote}


\paragraph{ILLUMINATI, n.}  A sect of Spanish heretics of the latter part of the
sixteenth century; so called because they were light weights --
{\em cunctationes illuminati}.

\paragraph{ILLUSTRIOUS, adj.}  Suitably placed for the shafts of malice, envy and
detraction.

\paragraph{IMAGINATION, n.}  A warehouse of facts, with poet and liar in joint
ownership.

\paragraph{IMBECILITY, n.}  A kind of divine inspiration, or sacred fire affecting
censorious critics of this dictionary.

\paragraph{IMMIGRANT, n.}  An unenlightened person who thinks one country better
than another.

\paragraph{IMMODEST, adj.}  Having a strong sense of one's own merit, coupled with
a feeble conception of worth in others.

\begin{quote}   There was once a man in Ispahan \\
      Ever and ever so long ago, \\
  And he had a head, the phrenologists said, \\
      That fitted him for a show. \\
 \\
  For his modesty's bump was so large a lump \\
      (Nature, they said, had taken a freak) \\
  That its summit stood far above the wood \\
      Of his hair, like a mountain peak. \\
 \\
  So modest a man in all Ispahan, \\
      Over and over again they swore -- \\
  So humble and meek, you would vainly seek; \\
      None ever was found before. \\
 \\
  Meantime the hump of that awful bump \\
      Into the heavens contrived to get \\
  To so great a height that they called the wight \\
      The man with the minaret. \\
 \\
  There wasn't a man in all Ispahan \\
      Prouder, or louder in praise of his chump: \\
  With a tireless tongue and a brazen lung \\
      He bragged of that beautiful bump \\
 \\
  Till the Shah in a rage sent a trusty page \\
      Bearing a sack and a bow-string too, \\
  And that gentle child explained as he smiled: \\
      "A little present for you." \\
 \\
  The saddest man in all Ispahan, \\
      Sniffed at the gift, yet accepted the same. \\
  "If I'd lived," said he, "my humility \\
      Had given me deathless fame!" \\
 \\
Sukker Uffro \end{quote}


\paragraph{IMMORAL, adj.}  Inexpedient.  Whatever in the long run and with regard
to the greater number of instances men find to be generally
inexpedient comes to be considered wrong, wicked, immoral.  If man's
notions of right and wrong have any other basis than this of
expediency; if they originated, or could have originated, in any other
way; if actions have in themselves a moral character apart from, and
nowise dependent on, their consequences -- then all philosophy is a
lie and reason a disorder of the mind.

\paragraph{IMMORTALITY, n.}

\begin{quote}   A toy which people cry for, \\
  And on their knees apply for, \\
  Dispute, contend and lie for, \\
      And if allowed \\
      Would be right proud \\
  Eternally to die for. \\
 \\
G.J. \end{quote}


\paragraph{IMPALE, v.t.}  In popular usage to pierce with any weapon which remains
fixed in the wound.  This, however, is inaccurate; to impale is,
properly, to put to death by thrusting an upright sharp stake into the
body, the victim being left in a sitting position.  This was a common
mode of punishment among many of the nations of antiquity, and is
still in high favor in China and other parts of Asia.  Down to the
beginning of the fifteenth century it was widely employed in
"churching" heretics and schismatics.  Wolecraft calls it the "stoole
of repentynge," and among the common people it was jocularly known as
"riding the one legged horse."  Ludwig Salzmann informs us that in
Thibet impalement is considered the most appropriate punishment for
crimes against religion; and although in China it is sometimes awarded
for secular offences, it is most frequently adjudged in cases of
sacrilege.  To the person in actual experience of impalement it must
be a matter of minor importance by what kind of civil or religious
dissent he was made acquainted with its discomforts; but doubtless he
would feel a certain satisfaction if able to contemplate himself in
the character of a weather-cock on the spire of the True Church.

\paragraph{IMPARTIAL, adj.}  Unable to perceive any promise of personal advantage
from espousing either side of a controversy or adopting either of two
conflicting opinions.

\paragraph{IMPENITENCE, n.}  A state of mind intermediate in point of time between
sin and punishment.

\paragraph{IMPIETY, n.}  Your irreverence toward my deity.

\paragraph{IMPOSITION, n.}  The act of blessing or consecrating by the laying on
of hands -- a ceremony common to many ecclesiastical systems, but
performed with the frankest sincerity by the sect known as Thieves.

\begin{quote}   "Lo! by the laying on of hands," \\
      Say parson, priest and dervise, \\
  "We consecrate your cash and lands \\
      To ecclesiastical service. \\
  No doubt you'll swear till all is blue \\
  At such an imposition.  Do." \\
 \\
Pollo Doncas \end{quote}


\paragraph{IMPOSTOR n.}  A rival aspirant to public honors.

\paragraph{IMPROBABILITY, n.}

\begin{quote}   His tale he told with a solemn face \\
  And a tender, melancholy grace. \\
      Improbable 'twas, no doubt, \\
      When you came to think it out, \\
      But the fascinated crowd \\
      Their deep surprise avowed \\
  And all with a single voice averred \\
  'Twas the most amazing thing they'd heard -- \\
  All save one who spake never a word, \\
      But sat as mum \\
      As if deaf and dumb, \\
  Serene, indifferent and unstirred. \\
      Then all the others turned to him \\
      And scrutinized him limb from limb -- \\
      Scanned him alive; \\
      But he seemed to thrive \\
      And tranquiler grow each minute, \\
      As if there were nothing in it. \\
  "What! what!" cried one, "are you not amazed \\
  At what our friend has told?"  He raised \\
  Soberly then his eyes and gazed \\
      In a natural way \\
      And proceeded to say, \\
  As he crossed his feet on the mantel-shelf: \\
  "O no -- not at all; I'm a liar myself."  \end{quote}

\paragraph{IMPROVIDENCE, n.}  Provision for the needs of to-day from the revenues
of to-morrow.

\paragraph{IMPUNITY, n.}  Wealth.

\paragraph{INADMISSIBLE, adj.}  Not competent to be considered.  Said of certain
kinds of testimony which juries are supposed to be unfit to be
entrusted with, and which judges, therefore, rule out, even of
proceedings before themselves alone.  Hearsay evidence is inadmissible
because the person quoted was unsworn and is not before the court for
examination; yet most momentous actions, military, political,
commercial and of every other kind, are daily undertaken on hearsay
evidence.  There is no religion in the world that has any other basis
than hearsay evidence.  Revelation is hearsay evidence; that the
Scriptures are the word of God we have only the testimony of men long
dead whose identity is not clearly established and who are not known
to have been sworn in any sense.  Under the rules of evidence as they
now exist in this country, no single assertion in the Bible has in its
support any evidence admissible in a court of law.  It cannot be
proved that the battle of Blenheim ever was fought, that there was
such as person as Julius Caesar, such an empire as Assyria.

But as records of courts of justice are admissible, it can easily
be proved that powerful and malevolent magicians once existed and were
a scourge to mankind.  The evidence (including confession) upon which
certain women were convicted of witchcraft and executed was without a
flaw; it is still unimpeachable.  The judges' decisions based on it
were sound in logic and in law.  Nothing in any existing court was
ever more thoroughly proved than the charges of witchcraft and sorcery
for which so many suffered death.  If there were no witches, human
testimony and human reason are alike destitute of value.

\paragraph{INAUSPICIOUSLY, adv.}  In an unpromising manner, the auspices being
unfavorable.  Among the Romans it was customary before undertaking any
important action or enterprise to obtain from the augurs, or state
prophets, some hint of its probable outcome; and one of their favorite
and most trustworthy modes of divination consisted in observing the
flight of birds -- the omens thence derived being called {\em auspices}.
Newspaper reporters and certain miscreant lexicographers have decided
that the word -- always in the plural -- shall mean "patronage" or
"management"; as, "The festivities were under the auspices of the
Ancient and Honorable Order of Body-Snatchers"; or, "The hilarities
were auspicated by the Knights of Hunger."

\begin{quote}   A Roman slave appeared one day \\
  Before the Augur.  "Tell me, pray, \\
  If --" here the Augur, smiling, made \\
  A checking gesture and displayed \\
  His open palm, which plainly itched, \\
  For visibly its surface twitched. \\
  A {\em denarius} (the Latin nickel) \\
  Successfully allayed the tickle, \\
  And then the slave proceeded:  "Please \\
  Inform me whether Fate decrees \\
  Success or failure in what I \\
  To-night (if it be dark) shall try. \\
  Its nature?  Never mind -- I think \\
  'Tis writ on this" -- and with a wink \\
  Which darkened half the earth, he drew \\
  Another denarius to view, \\
  Its shining face attentive scanned, \\
  Then slipped it into the good man's hand, \\
  Who with great gravity said:  "Wait \\
  While I retire to question Fate." \\
  That holy person then withdrew \\
  His scared clay and, passing through \\
  The temple's rearward gate, cried "Shoo!" \\
  Waving his robe of office.  Straight \\
  Each sacred peacock and its mate \\
  (Maintained for Juno's favor) fled \\
  With clamor from the trees o'erhead, \\
  Where they were perching for the night. \\
  The temple's roof received their flight, \\
  For thither they would always go, \\
  When danger threatened them below. \\
  Back to the slave the Augur went: \\
  "My son, forecasting the event \\
  By flight of birds, I must confess \\
  The auspices deny success." \\
  That slave retired, a sadder man, \\
  Abandoning his secret plan -- \\
  Which was (as well the craft seer \\
  Had from the first divined) to clear \\
  The wall and fraudulently seize \\
  On Juno's poultry in the trees. \\
 \\
G.J. \end{quote}


\paragraph{INCOME, n.}  The natural and rational gauge and measure of
respectability, the commonly accepted standards being artificial,
arbitrary and fallacious; for, as "Sir Sycophas Chrysolater" in the
play has justly remarked, "the true use and function of property (in
whatsoever it consisteth -- coins, or land, or houses, or merchant-
stuff, or anything which may be named as holden of right to one's own
subservience) as also of honors, titles, preferments and place, and
all favor and acquaintance of persons of quality or ableness, are but
to get money.  Hence it followeth that all things are truly to be
rated as of worth in measure of their serviceableness to that end; and
their possessors should take rank in agreement thereto, neither the
lord of an unproducing manor, howsoever broad and ancient, nor he who
bears an unremunerate dignity, nor yet the pauper favorite of a king,
being esteemed of level excellency with him whose riches are of daily
accretion; and hardly should they whose wealth is barren claim and
rightly take more honor than the poor and unworthy."

\paragraph{INCOMPATIBILITY, n.}  In matrimony a similarity of tastes, particularly
the taste for domination.  Incompatibility may, however, consist of a
meek-eyed matron living just around the corner.  It has even been
known to wear a moustache.

\paragraph{INCOMPOSSIBLE, adj.}  Unable to exist if something else exists.  Two
things are incompossible when the world of being has scope enough for
one of them, but not enough for both -- as Walt Whitman's poetry and
God's mercy to man.  Incompossibility, it will be seen, is only
incompatibility let loose.  Instead of such low language as "Go heel
yourself -- I mean to kill you on sight," the words, "Sir, we are
incompossible," would convey and equally significant intimation and in
stately courtesy are altogether superior.

\paragraph{INCUBUS, n.}  One of a race of highly improper demons who, though
probably not wholly extinct, may be said to have seen their best
nights.  For a complete account of {\em incubi} and {\em succubi}, including
{\em incubae} and {\em succubae}, see the {\em Liber Demonorum} of Protassus
(Paris, 1328), which contains much curious information that would be
out of place in a dictionary intended as a text-book for the public
schools.
\subparagraph{}   Victor Hugo relates that in the Channel Islands Satan himself --
tempted more than elsewhere by the beauty of the women, doubtless --
sometimes plays at {\em incubus}, greatly to the inconvenience and alarm
of the good dames who wish to be loyal to their marriage vows,
generally speaking.  A certain lady applied to the parish priest to
learn how they might, in the dark, distinguish the hardy intruder from
their husbands.  The holy man said they must feel his brown for horns;
but Hugo is ungallant enough to hint a doubt of the efficacy of the
test.

\paragraph{INCUMBENT, n.}  A person of the liveliest interest to the outcumbents.

\paragraph{INDECISION, n.}  The chief element of success; "for whereas," saith Sir
Thomas Brewbold, "there is but one way to do nothing and divers way to
do something, whereof, to a surety, only one is the right way, it
followeth that he who from indecision standeth still hath not so many
chances of going astray as he who pusheth forwards" -- a most clear
and satisfactory exposition on the matter.
\begin{quote}   "Your prompt decision to attack," said Genera Grant on a certain
occasion to General Gordon Granger, "was admirable; you had but five
minutes to make up your mind in." \\
    "Yes, sir," answered the victorious subordinate, "it is a great 
thing to be know exactly what to do in an emergency.  When in doubt
whether to attack or retreat I never hesitate a moment -- I toss us a
copper." \\
     "Do you mean to say that's what you did this time?" \\
  "Yes, General; but for Heaven's sake don't reprimand me:  I
disobeyed the coin."
\end{quote}

\paragraph{INDIFFERENT, adj.}  Imperfectly sensible to distinctions among things.

\begin{quote}   "You tiresome man!" cried Indolentio's wife, \\
  "You've grown indifferent to all in life." \\
  "Indifferent?" he drawled with a slow smile; \\
  "I would be, dear, but it is not worth while." \\
 \\
Apuleius M. Gokul \end{quote}


\paragraph{INDIGESTION, n.}  A disease which the patient and his friends
frequently mistake for deep religious conviction and concern for the
salvation of mankind.  As the simple Red Man of the western wild put
it, with, it must be confessed, a certain force:  "Plenty well, no
pray; big bellyache, heap God."

\paragraph{INDISCRETION, n.}  The guilt of woman.

\paragraph{INEXPEDIENT, adj.}  Not calculated to advance one's interests.

\paragraph{INFANCY, n.}  The period of our lives when, according to Wordsworth,
"Heaven lies about us."  The world begins lying about us pretty soon
afterward.

\paragraph{INFERIAE,n.}  [Latin]  Among the Greeks and Romans, sacrifices for
propitiation of the {\em Dii Manes}, or souls of the dead heroes; for the
pious ancients could not invent enough gods to satisfy their spiritual
needs, and had to have a number of makeshift deities, or, as a sailor
might say, jury-gods, which they made out of the most unpromising
materials.  It was while sacrificing a bullock to the spirit of
Agamemnon that Laiaides, a priest of Aulis, was favored with an
audience of that illustrious warrior's shade, who prophetically
recounted to him the birth of Christ and the triumph of Christianity,
giving him also a rapid but tolerably complete review of events down
to the reign of Saint Louis.  The narrative ended abruptly at the
point, owing to the inconsiderate crowing of a cock, which compelled
the ghosted King of Men to scamper back to Hades.  There is a fine
mediaeval flavor to this story, and as it has not been traced back
further than Pere Brateille, a pious but obscure writer at the court
of Saint Louis, we shall probably not err on the side of presumption
in considering it apocryphal, though Monsignor Capel's judgment of the
matter might be different; and to that I bow -- wow.

\paragraph{INFIDEL, n.}  In New York, one who does not believe in the Christian
religion; in Constantinople, one who does.  (See GIAOUR.)  A kind of
scoundrel imperfectly reverent of, and niggardly contributory to,
divines, ecclesiastics, popes, parsons, canons, monks, mollahs,
voodoos, presbyters, hierophants, prelates, obeah-men, abbes, nuns,
missionaries, exhorters, deacons, friars, hadjis, high-priests,
muezzins, brahmins, medicine-men, confessors, eminences, elders,
primates, prebendaries, pilgrims, prophets, imaums, beneficiaries,
clerks, vicars-choral, archbishops, bishops, abbots, priors,
preachers, padres, abbotesses, caloyers, palmers, curates, patriarchs,
bonezs, santons, beadsmen, canonesses, residentiaries, diocesans,
deans, subdeans, rural deans, abdals, charm-sellers, archdeacons,
hierarchs, class-leaders, incumbents, capitulars, sheiks, talapoins,
postulants, scribes, gooroos, precentors, beadles, fakeers, sextons,
reverences, revivalists, cenobites, perpetual curates, chaplains,
mudjoes, readers, novices, vicars, pastors, rabbis, ulemas, lamas,
sacristans, vergers, dervises, lectors, church wardens, cardinals,
prioresses, suffragans, acolytes, rectors, cures, sophis, mutifs and
pumpums.

\paragraph{INFLUENCE, n.}  In politics, a visionary {\em quo} given in exchange for a
substantial {\em quid}.

\paragraph{INFALAPSARIAN, n.}  One who ventures to believe that Adam need not have
sinned unless he had a mind to -- in opposition to the
Supralapsarians, who hold that that luckless person's fall was decreed
from the beginning.  Infralapsarians are sometimes called
Sublapsarians without material effect upon the importance and lucidity
of their views about Adam.

\begin{quote}   Two theologues once, as they wended their way \\
  To chapel, engaged in colloquial fray -- \\
  An earnest logomachy, bitter as gall, \\
  Concerning poor Adam and what made him fall. \\
  "'Twas Predestination," cried one -- "for the Lord \\
  Decreed he should fall of his own accord." \\
  "Not so -- 'twas Free will," the other maintained, \\
  "Which led him to choose what the Lord had ordained." \\
  So fierce and so fiery grew the debate \\
  That nothing but bloodshed their dudgeon could sate; \\
  So off flew their cassocks and caps to the ground \\
  And, moved by the spirit, their hands went round. \\
  Ere either had proved his theology right \\
  By winning, or even beginning, the fight, \\
  A gray old professor of Latin came by, \\
  A staff in his hand and a scowl in his eye, \\
  And learning the cause of their quarrel (for still \\
  As they clumsily sparred they disputed with skill \\
  Of foreordination freedom of will) \\
  Cried:  "Sirrahs! this reasonless warfare compose: \\
  Atwixt ye's no difference worthy of blows. \\
  The sects ye belong to -- I'm ready to swear \\
  Ye wrongly interpret the names that they bear. \\
  {\em You} -- Infralapsarian son of a clown! -- \\
  Should only contend that Adam slipped down; \\
  While {\em you} -- you Supralapsarian pup! -- \\
  Should nothing aver but that Adam slipped up. \\
  It's all the same whether up or down \\
  You slip on a peel of banana brown. \\
  Even Adam analyzed not his blunder, \\
  But thought he had slipped on a peal of thunder! \\
 \\
G.J. \end{quote}


\paragraph{INGRATE, n.}  One who receives a benefit from another, or is otherwise
an object of charity.

\begin{quote}   "All men are ingrates," sneered the cynic.  "Nay," \\
      The good philanthropist replied; \\
  "I did great service to a man one day \\
  Who never since has cursed me to repay, \\
              Nor vilified." \\
 \\
  "Ho!" cried the cynic, "lead me to him straight -- \\
      With veneration I am overcome, \\
  And fain would have his blessing."  "Sad your fate -- \\
  He cannot bless you, for AI grieve to state \\
              This man is dumb." \\
 \\
Ariel Selp \end{quote}


\paragraph{INJURY, n.}  An offense next in degree of enormity to a slight.

\paragraph{INJUSTICE, n.}  A burden which of all those that we load upon others
and carry ourselves is lightest in the hands and heaviest upon the
back.

\paragraph{INK, n.}  A villainous compound of tannogallate of iron, gum-arabic and
water, chiefly used to facilitate the infection of idiocy and promote
intellectual crime.  The properties of ink are peculiar and
contradictory:  it may be used to make reputations and unmake them; to
blacken them and to make them white; but it is most generally and
acceptably employed as a mortar to bind together the stones of an
edifice of fame, and as a whitewash to conceal afterward the rascal
quality of the material.  There are men called journalists who have
established ink baths which some persons pay money to get into, others
to get out of.  Not infrequently it occurs that a person who has paid
to get in pays twice as much to get out.

\paragraph{INNATE, adj.}  Natural, inherent -- as innate ideas, that is to say,
ideas that we are born with, having had them previously imparted to
us.  The doctrine of innate ideas is one of the most admirable faiths
of philosophy, being itself an innate idea and therefore inaccessible
to disproof, though Locke foolishly supposed himself to have given it
"a black eye."  Among innate ideas may be mentioned the belief in
one's ability to conduct a newspaper, in the greatness of one's
country, in the superiority of one's civilization, in the importance
of one's personal affairs and in the interesting nature of one's
diseases.

\paragraph{IN'ARDS, n.}  The stomach, heart, soul and other bowels.  Many eminent
investigators do not class the soul as an in'ard, but that acute
observer and renowned authority, Dr. Gunsaulus, is persuaded that the
mysterious organ known as the spleen is nothing less than our
important part.  To the contrary, Professor Garrett P. Servis holds
that man's soul is that prolongation of his spinal marrow which forms
the pith of his no tail; and for demonstration of his faith points
confidently to the fact that no tailed animals have no souls.
Concerning these two theories, it is best to suspend judgment by
believing both.

\paragraph{INSCRIPTION, n.}  Something written on another thing.  Inscriptions are
of many kinds, but mostly memorial, intended to commemorate the fame
of some illustrious person and hand down to distant ages the record of
his services and virtues.  To this class of inscriptions belongs the
name of John Smith, penciled on the Washington monument.  Following
are examples of memorial inscriptions on tombstones:  (See EPITAPH.)

\begin{quote}   "In the sky my soul is found, \\
  And my body in the ground. \\
  By and by my body'll rise \\
  To my spirit in the skies, \\
  Soaring up to Heaven's gate. \\
          1878." \\
 \\
  "Sacred to the memory of Jeremiah Tree.  Cut down May 9th, 1862, \\
aged 27 yrs. 4 mos. and 12 ds.  Indigenous." \end{quote}

\begin{quote}       "Affliction sore long time she boar, \\
          Phisicians was in vain, \\
      Till Deth released the dear deceased \\
          And left her a remain. \\
  Gone to join Ananias in the regions of bliss." \\
 \\
  "The clay that rests beneath this stone \\
  As Silas Wood was widely known. \\
  Now, lying here, I ask what good \\
  It was to let me be S. Wood. \\
  O Man, let not ambition trouble you, \\
  Is the advice of Silas W." \\
 \\
  "Richard Haymon, of Heaven.  Fell to Earth Jan. 20, 1807, and had \\
the dust brushed off him Oct. 3, 1874." \end{quote}

\paragraph{INSECTIVORA, n.}

\begin{quote}   "See," cries the chorus of admiring preachers, \\
  "How Providence provides for all His creatures!" \\
  "His care," the gnat said, "even the insects follows: \\
  For us He has provided wrens and swallows." \\
 \\
Sempen Railey \end{quote}


\paragraph{INSURANCE, n.}  An ingenious modern game of chance in which the player
is permitted to enjoy the comfortable conviction that he is beating
the man who keeps the table.

\begin{quote}   INSURANCE AGENT:  My dear sir, that is a fine house -- pray let me \\
      insure it. \\
  HOUSE OWNER:  With pleasure.  Please make the annual premium so \\
      low that by the time when, according to the tables of your \\
      actuary, it will probably be destroyed by fire I will have \\
      paid you considerably less than the face of the policy. \\
  INSURANCE AGENT:  O dear, no -- we could not afford to do that. \\
      We must fix the premium so that you will have paid more. \\
  HOUSE OWNER:  How, then, can {\em I} afford {\em that}? \\
  INSURANCE AGENT:  Why, your house may burn down at any time. \\
      There was Smith's house, for example, which -- \\
  HOUSE OWNER:  Spare me -- there were Brown's house, on the \\
      contrary, and Jones's house, and Robinson's house, which -- \\
  INSURANCE AGENT:  Spare {\em me}! \\
  HOUSE OWNER:  Let us understand each other.  You want me to pay \\
      you money on the supposition that something will occur \\
      previously to the time set by yourself for its occurrence.  In \\
      other words, you expect me to bet that my house will not last \\
      so long as you say that it will probably last. \\
  INSURANCE AGENT:  But if your house burns without insurance it \\
      will be a total loss. \\
  HOUSE OWNER:  Beg your pardon -- by your own actuary's tables I \\
      shall probably have saved, when it burns, all the premiums I \\
      would otherwise have paid to you -- amounting to more than the \\
      face of the policy they would have bought.  But suppose it to \\
      burn, uninsured, before the time upon which your figures are \\
      based.  If I could not afford that, how could you if it were \\
      insured? \\
  INSURANCE AGENT:  O, we should make ourselves whole from our \\
      luckier ventures with other clients.  Virtually, they pay your \\
      loss. \\
  HOUSE OWNER:  And virtually, then, don't I help to pay their \\
      losses?  Are not their houses as likely as mine to burn before \\
      they have paid you as much as you must pay them?  The case \\
      stands this way:  you expect to take more money from your \\
      clients than you pay to them, do you not? \\
  INSURANCE AGENT:  Certainly; if we did not -- \\
  HOUSE OWNER:  I would not trust you with my money.  Very well \\
      then.  If it is {\em certain}, with reference to the whole body of \\
      your clients, that they lose money on you it is {\em probable}, \\
      with reference to any one of them, that {\em he} will.  It is \\
      these individual probabilities that make the aggregate \\
      certainty. \\
  INSURANCE AGENT:  I will not deny it -- but look at the figures in \\
      this pamph -- \\
  HOUSE OWNER:  Heaven forbid! \\
  INSURANCE AGENT:  You spoke of saving the premiums which you would \\
      otherwise pay to me.  Will you not be more likely to squander \\
      them?  We offer you an incentive to thrift. \\
  HOUSE OWNER:  The willingness of A to take care of B's money is \\
      not peculiar to insurance, but as a charitable institution you \\
      command esteem.  Deign to accept its expression from a \\
      Deserving Object.  \end{quote}

\paragraph{INSURRECTION, n.}  An unsuccessful revolution.  Disaffection's failure
to substitute misrule for bad government.

\paragraph{INTENTION, n.}  The mind's sense of the prevalence of one set of
influences over another set; an effect whose cause is the imminence,
immediate or remote, of the performance of an involuntary act.

\paragraph{INTERPRETER, n.}  One who enables two persons of different languages to
understand each other by repeating to each what it would have been to
the interpreter's advantage for the other to have said.

\paragraph{INTERREGNUM, n.}  The period during which a monarchical country is
governed by a warm spot on the cushion of the throne.  The experiment
of letting the spot grow cold has commonly been attended by most
unhappy results from the zeal of many worthy persons to make it warm
again.

\paragraph{INTIMACY, n.}  A relation into which fools are providentially drawn for
their mutual destruction.

\begin{quote}   Two Seidlitz powders, one in blue \\
  And one in white, together drew \\
  And having each a pleasant sense \\
  Of t'other powder's excellence, \\
  Forsook their jackets for the snug \\
  Enjoyment of a common mug. \\
  So close their intimacy grew \\
  One paper would have held the two. \\
  To confidences straight they fell, \\
  Less anxious each to hear than tell; \\
  Then each remorsefully confessed \\
  To all the virtues he possessed, \\
  Acknowledging he had them in \\
  So high degree it was a sin. \\
  The more they said, the more they felt \\
  Their spirits with emotion melt, \\
  Till tears of sentiment expressed \\
  Their feelings.  Then they effervesced! \\
  So Nature executes her feats \\
  Of wrath on friends and sympathetes \\
  The good old rule who don't apply, \\
  That you are you and I am I.  \end{quote}

\paragraph{INTRODUCTION, n.}  A social ceremony invented by the devil for the
gratification of his servants and the plaguing of his enemies.  The
introduction attains its most malevolent development in this century,
being, indeed, closely related to our political system.  Every
American being the equal of every other American, it follows that
everybody has the right to know everybody else, which implies the
right to introduce without request or permission.  The Declaration of
Independence should have read thus:

\begin{quote}       "We hold these truths to be self-evident:  that all men are
  created equal; that they are endowed by their Creator with certain 
  inalienable rights; that among these are life, and the right to 
  make that of another miserable by thrusting upon him an 
  incalculable quantity of acquaintances; liberty, particularly the 
  liberty to introduce persons to one another without first 
  ascertaining if they are not already acquainted as enemies; and 
  the pursuit of another's happiness with a running pack of 
  strangers."  \end{quote}

\paragraph{INVENTOR, n.}  A person who makes an ingenious arrangement of wheels,
levers and springs, and believes it civilization.

\paragraph{IRRELIGION, n.}  The principal one of the great faiths of the world.

\paragraph{ITCH, n.}  The patriotism of a Scotchman.



\section*{J}



\paragraph{J} is a consonant in English, but some nations use it as a vowel --
than which nothing could be more absurd.  Its original form, which has
been but slightly modified, was that of the tail of a subdued dog, and
it was not a letter but a character, standing for a Latin verb,
{\em jacere}, "to throw," because when a stone is thrown at a dog the
dog's tail assumes that shape.  This is the origin of the letter, as
expounded by the renowned Dr. Jocolpus Bumer, of the University of
Belgrade, who established his conclusions on the subject in a work of
three quarto volumes and committed suicide on being reminded that the
j in the Roman alphabet had originally no curl.

\paragraph{JEALOUS, adj.}  Unduly concerned about the preservation of that which
can be lost only if not worth keeping.

\paragraph{JESTER, n.}  An officer formerly attached to a king's household, whose
business it was to amuse the court by ludicrous actions and
utterances, the absurdity being attested by his motley costume.  The
king himself being attired with dignity, it took the world some
centuries to discover that his own conduct and decrees were
sufficiently ridiculous for the amusement not only of his court but of
all mankind.  The jester was commonly called a fool, but the poets and
romancers have ever delighted to represent him as a singularly wise
and witty person.  In the circus of to-day the melancholy ghost of the
court fool effects the dejection of humbler audiences with the same
jests wherewith in life he gloomed the marble hall, panged the
patrician sense of humor and tapped the tank of royal tears.

\begin{quote}   The widow-queen of Portugal \\
      Had an audacious jester \\
  Who entered the confessional \\
      Disguised, and there confessed her. \\
 \\
  "Father," she said, "thine ear bend down -- \\
      My sins are more than scarlet: \\
  I love my fool -- blaspheming clown, \\
      And common, base-born varlet." \\
 \\
  "Daughter," the mimic priest replied, \\
      "That sin, indeed, is awful: \\
  The church's pardon is denied \\
      To love that is unlawful. \\
  "But since thy stubborn heart will be \\
      For him forever pleading, \\
  Thou'dst better make him, by decree, \\
      A man of birth and breeding." \\
 \\
  She made the fool a duke, in hope \\
      With Heaven's taboo to palter; \\
  Then told a priest, who told the Pope, \\
      Who damned her from the altar! \\
 \\
Barel Dort \end{quote}


\paragraph{JEWS-HARP, n.}  An unmusical instrument, played by holding it fast with
the teeth and trying to brush it away with the finger.

\paragraph{JOSS-STICKS, n.}  Small sticks burned by the Chinese in their pagan
tomfoolery, in imitation of certain sacred rites of our holy religion.

\paragraph{JUSTICE, n.}  A commodity which is a more or less adulterated condition
the State sells to the citizen as a reward for his allegiance, taxes
and personal service.



\section*{K}




\paragraph{K} is a consonant that we get from the Greeks, but it can be traced
away back beyond them to the Cerathians, a small commercial nation
inhabiting the peninsula of Smero.  In their tongue it was called
{\em Klatch}, which means "destroyed."  The form of the letter was
originally precisely that of our H, but the erudite Dr. Snedeker
explains that it was altered to its present shape to commemorate the
destruction of the great temple of Jarute by an earthquake, {\em circa}
730 B.C.  This building was famous for the two lofty columns of its
portico, one of which was broken in half by the catastrophe, the other
remaining intact.  As the earlier form of the letter is supposed to
have been suggested by these pillars, so, it is thought by the great
antiquary, its later was adopted as a simple and natural -- not to say
touching -- means of keeping the calamity ever in the national memory.
It is not known if the name of the letter was altered as an additional
mnemonic, or if the name was always {\em Klatch} and the destruction one
of nature's puns.  As each theory seems probable enough, I see no
objection to believing both -- and Dr. Snedeker arrayed himself on
that side of the question.

\paragraph{KEEP, v.t.}

\begin{quote}   He willed away his whole estate, \\
      And then in death he fell asleep, \\
  Murmuring:  "Well, at any rate, \\
      My name unblemished I shall keep." \\
  But when upon the tomb 'twas wrought \\
  Whose was it? -- for the dead keep naught. \\
 \\
Durang Gophel Arn \end{quote}


\paragraph{KILL, v.t.}  To create a vacancy without nominating a successor.

\paragraph{KILT, n.}  A costume sometimes worn by Scotchmen in America and
Americans in Scotland.

\paragraph{KINDNESS, n.}  A brief preface to ten volumes of exaction.

\paragraph{KING, n.}  A male person commonly known in America as a "crowned head,"
although he never wears a crown and has usually no head to speak of.

\begin{quote}   A king, in times long, long gone by, \\
      Said to his lazy jester: \\
  "If I were you and you were I \\
  My moments merrily would fly -- \\
      Nor care nor grief to pester." \\
 \\
  "The reason, Sire, that you would thrive," \\
      The fool said -- "if you'll hear it -- \\
  Is that of all the fools alive \\
  Who own you for their sovereign, I've \\
      The most forgiving spirit." \\
 \\
Oogum Bem \end{quote}


\paragraph{KING'S EVIL, n.}  A malady that was formerly cured by the touch of the
sovereign, but has now to be treated by the physicians.  Thus 'the
most pious Edward" of England used to lay his royal hand upon the
ailing subjects and make them whole~--

\begin{quote}                   a crowd of wretched souls \\
  That stay his cure:  their malady convinces \\
  The great essay of art; but at his touch, \\
  Such sanctity hath Heaven given his hand, \\
  They presently amend, 
\end{quote}
\subparagraph{} as the "Doctor" in {\em Macbeth} hath it.  This useful property of the
royal hand could, it appears, be transmitted along with other crown
properties; for according to "Malcolm,"

\begin{quote}                           'tis spoken \\
  To the succeeding royalty he leaves \\
  The healing benediction.
\end{quote}

\subparagraph{}  But the gift somewhere dropped out of the line of succession:  the
later sovereigns of England have not been tactual healers, and the
disease once honored with the name "king's evil" now bears the humbler
one of "scrofula," from {\em scrofa}, a sow.  The date and author of the
following epigram are known only to the author of this dictionary, but
it is old enough to show that the jest about Scotland's national
disorder is not a thing of yesterday.

\begin{quote}   Ye Kynge his evill in me laye, \\
  Wh. he of Scottlande charmed awaye. \\
  He layde his hand on mine and sayd: \\
  "Be gone!"  Ye ill no longer stayd. \\
  But O ye wofull plyght in wh. \\
  I'm now y-pight:  I have ye itche!
\end{quote}
  
\subparagraph{}  The superstition that maladies can be cured by royal taction is
dead, but like many a departed conviction it has left a monument of
custom to keep its memory green.  The practice of forming a line and
shaking the President's hand had no other origin, and when that great
dignitary bestows his healing salutation on

\begin{quote}                       strangely visited people, \\
  All swoln and ulcerous, pitiful to the eye, \\
  The mere despair of surgery,
\end{quote}

\subparagraph{} he and his patients are handing along an extinguished torch which once
was kindled at the altar-fire of a faith long held by all classes of
men.  It is a beautiful and edifying "survival" -- one which brings
the sainted past close home in our "business and bosoms."

\paragraph{KISS, n.}  A word invented by the poets as a rhyme for "bliss."  It is
supposed to signify, in a general way, some kind of rite or ceremony
appertaining to a good understanding; but the manner of its
performance is unknown to this lexicographer.

\paragraph{KLEPTOMANIAC, n.}  A rich thief.

\paragraph{KNIGHT, n.}

\begin{quote}   Once a warrior gentle of birth, \\
  Then a person of civic worth, \\
  Now a fellow to move our mirth. \\
  Warrior, person, and fellow -- no more: \\
  We must knight our dogs to get any lower. \\
  Brave Knights Kennelers then shall be, \\
  Noble Knights of the Golden Flea, \\
  Knights of the Order of St. Steboy, \\
  Knights of St. Gorge and Sir Knights Jawy. \\
  God speed the day when this knighting fad \\
  Shall go to the dogs and the dogs go mad.  \end{quote}

\paragraph{KORAN, n.}  A book which the Mohammedans foolishly believe to have been
written by divine inspiration, but which Christians know to be a
wicked imposture, contradictory to the Holy Scriptures.



\section*{L}



\paragraph{LABOR, n.}  One of the processes by which A acquires property for B.

\paragraph{LAND, n.}  A part of the earth's surface, considered as property.  The
theory that land is property subject to private ownership and control
is the foundation of modern society, and is eminently worthy of the
superstructure.  Carried to its logical conclusion, it means that some
have the right to prevent others from living; for the right to own
implies the right exclusively to occupy; and in fact laws of trespass
are enacted wherever property in land is recognized.  It follows that
if the whole area of {\em terra firma} is owned by A, B and C, there will
be no place for D, E, F and G to be born, or, born as trespassers, to
exist.

\begin{quote}   A life on the ocean wave, \\
      A home on the rolling deep, \\
  For the spark the nature gave \\
      I have there the right to keep. \\
 \\
  They give me the cat-o'-nine \\
      Whenever I go ashore. \\
  Then ho! for the flashing brine -- \\
      I'm a natural commodore! \\
 \\
Dodle \end{quote}


\paragraph{LANGUAGE, n.}  The music with which we charm the serpents guarding
another's treasure.

\paragraph{LAOCOON, n.}  A famous piece of antique scripture representing a priest
of that name and his two sons in the folds of two enormous serpents.
The skill and diligence with which the old man and lads support the
serpents and keep them up to their work have been justly regarded as
one of the noblest artistic illustrations of the mastery of human
intelligence over brute inertia.

\paragraph{LAP, n.}  One of the most important organs of the female system -- an
admirable provision of nature for the repose of infancy, but chiefly
useful in rural festivities to support plates of cold chicken and
heads of adult males.  The male of our species has a rudimentary lap,
imperfectly developed and in no way contributing to the animal's
substantial welfare.

\paragraph{LAST, n.}  A shoemaker's implement, named by a frowning Providence as
opportunity to the maker of puns.

\begin{quote}   Ah, punster, would my lot were cast, \\
      Where the cobbler is unknown, \\
  So that I might forget his last \\
      And hear your own. \\
 \\
Gargo Repsky \end{quote}


\paragraph{LAUGHTER, n.}  An interior convulsion, producing a distortion of the
features and accompanied by inarticulate noises.  It is infectious
and, though intermittent, incurable.  Liability to attacks of laughter
is one of the characteristics distinguishing man from the animals --
these being not only inaccessible to the provocation of his example,
but impregnable to the microbes having original jurisdiction in
bestowal of the disease.  Whether laughter could be imparted to
animals by inoculation from the human patient is a question that has
not been answered by experimentation.  Dr. Meir Witchell holds that
the infection character of laughter is due to the instantaneous
fermentation of {\em sputa} diffused in a spray.  From this peculiarity he
names the disorder {\em Convulsio spargens}.

\paragraph{LAUREATE, adj.}  Crowned with leaves of the laurel.  In England the
Poet Laureate is an officer of the sovereign's court, acting as
dancing skeleton at every royal feast and singing-mute at every royal
funeral.  Of all incumbents of that high office, Robert Southey had
the most notable knack at drugging the Samson of public joy and
cutting his hair to the quick; and he had an artistic color-sense
which enabled him so to blacken a public grief as to give it the
aspect of a national crime.

\paragraph{LAUREL, n.}  The {\em laurus}, a vegetable dedicated to Apollo, and
formerly defoliated to wreathe the brows of victors and such poets as
had influence at court.  ({\em Vide supra.})

\paragraph{LAW, n.}

\begin{quote}   Once Law was sitting on the bench, \\
      And Mercy knelt a-weeping. \\
  "Clear out!" he cried, "disordered wench! \\
      Nor come before me creeping. \\
  Upon your knees if you appear, \\
  'Tis plain your have no standing here." \\
 \\
  Then Justice came.  His Honor cried: \\
      "{\em Your} status? -- devil seize you!" \\
  "{\em Amica curiae,}" she replied -- \\
      "Friend of the court, so please you." \\
  "Begone!" he shouted -- "there's the door -- \\
  I never saw your face before!" \\
 \\
G.J. \end{quote}


\paragraph{LAWFUL, adj.}  Compatible with the will of a judge having jurisdiction.

\paragraph{LAWYER, n.}  One skilled in circumvention of the law.

\paragraph{LAZINESS, n.}  Unwarranted repose of manner in a person of low degree.

\paragraph{LEAD, n.}  A heavy blue-gray metal much used in giving stability to
light lovers -- particularly to those who love not wisely but other
men's wives.  Lead is also of great service as a counterpoise to an
argument of such weight that it turns the scale of debate the wrong
way.  An interesting fact in the chemistry of international
controversy is that at the point of contact of two patriotisms lead is
precipitated in great quantities.

\begin{quote}   Hail, holy Lead! -- of human feuds the great \\
      And universal arbiter; endowed \\
      With penetration to pierce any cloud \\
  Fogging the field of controversial hate, \\
  And with a sift, inevitable, straight, \\
      Searching precision find the unavowed \\
      But vital point.  Thy judgment, when allowed \\
  By the chirurgeon, settles the debate. \\
  O useful metal! -- were it not for thee \\
      We'd grapple one another's ears alway: \\
  But when we hear thee buzzing like a bee \\
      We, like old Muhlenberg, "care not to stay." \\
  And when the quick have run away like pellets \\
  Jack Satan smelts the dead to make new bullets.  \end{quote}

\paragraph{LEARNING, n.}  The kind of ignorance distinguishing the studious.

\paragraph{LECTURER, n.}  One with his hand in your pocket, his tongue in your ear
and his faith in your patience.

\paragraph{LEGACY, n.}  A gift from one who is legging it out of this vale of
tears.

\paragraph{LEONINE, adj.}  Unlike a menagerie lion.  Leonine verses are those in
which a word in the middle of a line rhymes with a word at the end, as
in this famous passage from Bella Peeler Silcox:

\begin{quote}   The electric light invades the dunnest deep of Hades. \\
  Cries Pluto, 'twixt his snores:  "O tempora! O mores!"
\end{quote}
  
\subparagraph{}  It should be explained that Mrs. Silcox does not undertake to
teach pronunciation of the Greek and Latin tongues.  Leonine verses
are so called in honor of a poet named Leo, whom prosodists appear to
find a pleasure in believing to have been the first to discover that a
rhyming couplet could be run into a single line.

\paragraph{LETTUCE, n.}  An herb of the genus {\em Lactuca}, "Wherewith," says that
pious gastronome, Hengist Pelly, "God has been pleased to reward the
good and punish the wicked.  For by his inner light the righteous man
has discerned a manner of compounding for it a dressing to the
appetency whereof a multitude of gustible condiments conspire, being
reconciled and ameliorated with profusion of oil, the entire
comestible making glad the heart of the godly and causing his face to
shine.  But the person of spiritual unworth is successfully tempted to
the Adversary to eat of lettuce with destitution of oil, mustard, egg,
salt and garlic, and with a rascal bath of vinegar polluted with
sugar.  Wherefore the person of spiritual unworth suffers an
intestinal pang of strange complexity and raises the song."

\paragraph{LEVIATHAN, n.}  An enormous aquatic animal mentioned by Job.  Some
suppose it to have been the whale, but that distinguished
ichthyologer, Dr. Jordan, of Stanford University, maintains with
considerable heat that it was a species of gigantic Tadpole ({\em Thaddeus
Polandensis}) or Polliwig -- {\em Maria pseudo-hirsuta}.  For an
exhaustive description and history of the Tadpole consult the famous
monograph of Jane Potter, {\em Thaddeus of Warsaw}.

\paragraph{LEXICOGRAPHER, n.}  A pestilent fellow who, under the pretense of
recording some particular stage in the development of a language, does
what he can to arrest its growth, stiffen its flexibility and
mechanize its methods.  For your lexicographer, having written his
dictionary, comes to be considered "as one having authority," whereas
his function is only to make a record, not to give a law.  The natural
servility of the human understanding having invested him with judicial
power, surrenders its right of reason and submits itself to a
chronicle as if it were a statue.  Let the dictionary (for example)
mark a good word as "obsolete" or "obsolescent" and few men
thereafter venture to use it, whatever their need of it and however
desirable its restoration to favor -- whereby the process of
impoverishment is accelerated and speech decays.  On the contrary,
recognizing the truth that language must grow by innovation if it grow
at all, makes new words and uses the old in an unfamiliar sense, has
no following and is tartly reminded that "it isn't in the dictionary"
-- although down to the time of the first lexicographer (Heaven
forgive him!) no author ever had used a word that {\em was} in the
dictionary.  In the golden prime and high noon of English speech; when
from the lips of the great Elizabethans fell words that made their own
meaning and carried it in their very sound; when a Shakespeare and a
Bacon were possible, and the language now rapidly perishing at one end
and slowly renewed at the other was in vigorous growth and hardy
preservation -- sweeter than honey and stronger than a lion -- the
lexicographer was a person unknown, the dictionary a creation which
his Creator had not created him to create.

\begin{quote}   God said:  "Let Spirit perish into Form," \\
  And lexicographers arose, a swarm! \\
  Thought fled and left her clothing, which they took, \\
  And catalogued each garment in a book. \\
  Now, from her leafy covert when she cries: \\
  "Give me my clothes and I'll return," they rise \\
  And scan the list, and say without compassion: \\
  "Excuse us -- they are mostly out of fashion." \\
 \\
Sigismund Smith \end{quote}


\paragraph{LIAR, n.}  A lawyer with a roving commission.

\paragraph{LIBERTY, n.}  One of Imagination's most precious possessions.

\begin{quote}   The rising People, hot and out of breath, \\
  Roared around the palace:  "Liberty or death!" \\
  "If death will do," the King said, "let me reign; \\
  You'll have, I'm sure, no reason to complain." \\
 \\
Martha Braymance \end{quote}


\paragraph{LICKSPITTLE, n.}  A useful functionary, not infrequently found editing
a newspaper.  In his character of editor he is closely allied to the
blackmailer by the tie of occasional identity; for in truth the
lickspittle is only the blackmailer under another aspect, although the
latter is frequently found as an independent species.  Lickspittling
is more detestable than blackmailing, precisely as the business of a
confidence man is more detestable than that of a highway robber; and
the parallel maintains itself throughout, for whereas few robbers will
cheat, every sneak will plunder if he dare.

\paragraph{LIFE, n.}  A spiritual pickle preserving the body from decay.  We live
in daily apprehension of its loss; yet when lost it is not missed.
The question, "Is life worth living?" has been much discussed;
particularly by those who think it is not, many of whom have written
at great length in support of their view and by careful observance of
the laws of health enjoyed for long terms of years the honors of
successful controversy.

\begin{quote}   "Life's not worth living, and that's the truth," \\
  Carelessly caroled the golden youth. \\
  In manhood still he maintained that view \\
  And held it more strongly the older he grew. \\
  When kicked by a jackass at eighty-three, \\
  "Go fetch me a surgeon at once!" cried he. \\
 \\
Han Soper \end{quote}


\paragraph{LIGHTHOUSE, n.}  A tall building on the seashore in which the
government maintains a lamp and the friend of a politician.

\paragraph{LIMB, n.}  The branch of a tree or the leg of an American woman.

\begin{quote}   'Twas a pair of boots that the lady bought, \\
      And the salesman laced them tight \\
      To a very remarkable height -- \\
  Higher, indeed, than I think he ought -- \\
      Higher than {\em can} be right. \\
  For the Bible declares -- but never mind: \\
      It is hardly fit \\
  To censure freely and fault to find \\
  With others for sins that I'm not inclined \\
      Myself to commit. \\
  Each has his weakness, and though my own \\
      Is freedom from every sin, \\
      It still were unfair to pitch in, \\
  Discharging the first censorious stone. \\
  Besides, the truth compels me to say, \\
  The boots in question were {\em made} that way. \\
  As he drew the lace she made a grimace, \\
      And blushingly said to him: \\
  "This boot, I'm sure, is too high to endure, \\
  It hurts my -- hurts my -- limb." \\
  The salesman smiled in a manner mild, \\
  Like an artless, undesigning child; \\
  Then, checking himself, to his face he gave \\
  A look as sorrowful as the grave, \\
      Though he didn't care two figs \\
  For her paints and throes, \\
  As he stroked her toes, \\
  Remarking with speech and manner just \\
  Befitting his calling:  "Madam, I trust \\
      That it doesn't hurt your twigs." \\
 \\
B. Percival Dike \end{quote}


\paragraph{LINEN, n.}  "A kind of cloth the making of which, when made of hemp,
entails a great waste of hemp." -- Calcraft the Hangman.

\paragraph{LITIGANT, n.}  A person about to give up his skin for the hope of
retaining his bones.

\paragraph{LITIGATION, n.}  A machine which you go into as a pig and come out of
as a sausage.

\paragraph{LIVER, n.}  A large red organ thoughtfully provided by nature to be
bilious with.  The sentiments and emotions which every literary
anatomist now knows to haunt the heart were anciently believed to
infest the liver; and even Gascoygne, speaking of the emotional side
of human nature, calls it "our hepaticall parte."  It was at one time
considered the seat of life; hence its name -- liver, the thing we
live with.  The liver is heaven's best gift to the goose; without it
that bird would be unable to supply us with the Strasbourg {\em pate}.

\paragraph{LL.D.}  Letters indicating the degree {\em Legumptionorum Doctor}, one
learned in laws, gifted with legal gumption.  Some suspicion is cast
upon this derivation by the fact that the title was formerly {\em LL.d.},
and conferred only upon gentlemen distinguished for their wealth.  At
the date of this writing Columbia University is considering the
expediency of making another degree for clergymen, in place of the old
D.D. -- {\em Damnator Diaboli}.  The new honor will be known as {\em Sanctorum
Custus}, and written {\em \$\$c}.  The name of the Rev. John Satan has been
suggested as a suitable recipient by a lover of consistency, who
points out that Professor Harry Thurston Peck has long enjoyed the
advantage of a degree.

\paragraph{LOCK-AND-KEY, n.}  The distinguishing device of civilization and
enlightenment.

\paragraph{LODGER, n.}  A less popular name for the Second Person of that
delectable newspaper Trinity, the Roomer, the Bedder, and the Mealer.

\paragraph{LOGIC, n.}  The art of thinking and reasoning in strict accordance with
the limitations and incapacities of the human misunderstanding.  The
basic of logic is the syllogism, consisting of a major and a minor
premise and a conclusion -- thus:
\begin{quote}   {\em Major Premise}:  Sixty men can do a piece of work sixty times as \\
quickly as one man. \end{quote}
\begin{quote}   {\em Minor Premise}:  One man can dig a posthole in sixty seconds; \\
therefore -- \end{quote}
\begin{quote}   {\em Conclusion}:  Sixty men can dig a posthole in one second. \\
  This may be called the syllogism arithmetical, in which, by \\
combining logic and mathematics, we obtain a double certainty and are \\
twice blessed.  \end{quote}

\paragraph{LOGOMACHY, n.}  A war in which the weapons are words and the wounds
punctures in the swim-bladder of self-esteem -- a kind of contest in
which, the vanquished being unconscious of defeat, the victor is
denied the reward of success.

\begin{quote}   'Tis said by divers of the scholar-men \\
  That poor Salmasius died of Milton's pen. \\
  Alas! we cannot know if this is true, \\
  For reading Milton's wit we perish too.  \end{quote}

\paragraph{LONGANIMITY, n.}  The disposition to endure injury with meek forbearance
while maturing a plan of revenge.

\paragraph{LONGEVITY, n.}  Uncommon extension of the fear of death.

\paragraph{LOOKING-GLASS, n.}  A vitreous plane upon which to display a fleeting
show for man's disillusion given.
\subparagraph{}   The King of Manchuria had a magic looking-glass, whereon whoso
looked saw, not his own image, but only that of the king.  A certain
courtier who had long enjoyed the king's favor and was thereby
enriched beyond any other subject of the realm, said to the king:
"Give me, I pray, thy wonderful mirror, so that when absent out of
thine august presence I may yet do homage before thy visible shadow,
prostrating myself night and morning in the glory of thy benign
countenance, as which nothing has so divine splendor, O Noonday Sun of
the Universe!"
\subparagraph{}   Please with the speech, the king commanded that the mirror be
conveyed to the courtier's palace; but after, having gone thither
without apprisal, he found it in an apartment where was naught but
idle lumber.  And the mirror was dimmed with dust and overlaced with
cobwebs.  This so angered him that he fisted it hard, shattering the
glass, and was sorely hurt.  Enraged all the more by this mischance,
he commanded that the ungrateful courtier be thrown into prison, and
that the glass be repaired and taken back to his own palace; and this
was done.  But when the king looked again on the mirror he saw not his
image as before, but only the figure of a crowned ass, having a bloody
bandage on one of its hinder hooves -- as the artificers and all who
had looked upon it had before discerned but feared to report.  Taught
wisdom and charity, the king restored his courtier to liberty, had the
mirror set into the back of the throne and reigned many years with
justice and humility; and one day when he fell asleep in death while
on the throne, the whole court saw in the mirror the luminous figure
of an angel, which remains to this day.

\paragraph{LOQUACITY, n.}  A disorder which renders the sufferer unable to curb
his tongue when you wish to talk.

\paragraph{LORD, n.}  In American society, an English tourist above the state of a
costermonger, as, lord 'Aberdasher, Lord Hartisan and so forth.  The
traveling Briton of lesser degree is addressed as "Sir," as, Sir 'Arry
Donkiboi, or 'Amstead 'Eath.  The word "Lord" is sometimes used, also,
as a title of the Supreme Being; but this is thought to be rather
flattery than true reverence.

\begin{quote}   Miss Sallie Ann Splurge, of her own accord, \\
  Wedded a wandering English lord -- \\
  Wedded and took him to dwell with her "paw," \\
  A parent who throve by the practice of Draw. \\
  Lord Cadde I don't hesitate to declare \\
  Unworthy the father-in-legal care \\
  Of that elderly sport, notwithstanding the truth \\
  That Cadde had renounced all the follies of youth; \\
  For, sad to relate, he'd arrived at the stage \\
  Of existence that's marked by the vices of age. \\
  Among them, cupidity caused him to urge \\
  Repeated demands on the pocket of Splurge, \\
  Till, wrecked in his fortune, that gentleman saw \\
  Inadequate aid in the practice of Draw, \\
  And took, as a means of augmenting his pelf, \\
  To the business of being a lord himself. \\
  His neat-fitting garments he wilfully shed \\
  And sacked himself strangely in checks instead; \\
  Denuded his chin, but retained at each ear \\
  A whisker that looked like a blasted career. \\
  He painted his neck an incarnadine hue \\
  Each morning and varnished it all that he knew. \\
  The moony monocular set in his eye \\
  Appeared to be scanning the Sweet Bye-and-Bye. \\
  His head was enroofed with a billycock hat, \\
  And his low-necked shoes were aduncous and flat. \\
  In speech he eschewed his American ways, \\
  Denying his nose to the use of his A's \\
  And dulling their edge till the delicate sense \\
  Of a babe at their temper could take no offence. \\
  His H's -- 'twas most inexpressibly sweet, \\
  The patter they made as they fell at his feet! \\
  Re-outfitted thus, Mr. Splurge without fear \\
  Began as Lord Splurge his recouping career. \\
  Alas, the Divinity shaping his end \\
  Entertained other views and decided to send \\
  His lordship in horror, despair and dismay \\
  From the land of the nobleman's natural prey. \\
  For, smit with his Old World ways, Lady Cadde \\
  Fell -- suffering Caesar! -- in love with her dad! \\
 \\
G.J. \end{quote}


\paragraph{LORE, n.}  Learning -- particularly that sort which is not derived from
a regular course of instruction but comes of the reading of occult
books, or by nature.  This latter is commonly designated as folk-lore
and embraces popularly myths and superstitions.  In Baring-Gould's
{\em Curious Myths of the Middle Ages} the reader will find many of these
traced backward, through various people son converging lines, toward a
common origin in remote antiquity.  Among these are the fables of
"Teddy the Giant Killer," "The Sleeping John Sharp Williams," "Little
Red Riding Hood and the Sugar Trust," "Beauty and the Brisbane," "The
Seven Aldermen of Ephesus," "Rip Van Fairbanks," and so forth.  The
fable with Goethe so affectingly relates under the title of "The Erl-
King" was known two thousand years ago in Greece as "The Demos and the
Infant Industry."  One of the most general and ancient of these myths
is that Arabian tale of "Ali Baba and the Forty Rockefellers."

\paragraph{LOSS, n.}  Privation of that which we had, or had not.  Thus, in the
latter sense, it is said of a defeated candidate that he "lost his
election"; and of that eminent man, the poet Gilder, that he has "lost
his mind."  It is in the former and more legitimate sense, that the
word is used in the famous epitaph:

\begin{quote}   Here Huntington's ashes long have lain \\
  Whose loss is our eternal gain, \\
  For while he exercised all his powers \\
  Whatever he gained, the loss was ours.  \end{quote}

\paragraph{LOVE, n.}  A temporary insanity curable by marriage or by removal of
the patient from the influences under which he incurred the disorder.
This disease, like {\em caries} and many other ailments, is prevalent only
among civilized races living under artificial conditions; barbarous
nations breathing pure air and eating simple food enjoy immunity from
its ravages.  It is sometimes fatal, but more frequently to the
physician than to the patient.

\paragraph{LOW-BRED, adj.}  "Raised" instead of brought up.

\paragraph{LUMINARY, n.}  One who throws light upon a subject; as an editor by not
writing about it.

\paragraph{LUNARIAN, n.}  An inhabitant of the moon, as distinguished from
Lunatic, one whom the moon inhabits.  The Lunarians have been
described by Lucian, Locke and other observers, but without much
agreement.  For example, Bragellos avers their anatomical identity
with Man, but Professor Newcomb says they are more like the hill
tribes of Vermont.

\paragraph{LYRE, n.}  An ancient instrument of torture.  The word is now used in a
figurative sense to denote the poetic faculty, as in the following
fiery lines of our great poet, Ella Wheeler Wilcox:

\begin{quote}   I sit astride Parnassus with my lyre, \\
  And pick with care the disobedient wire. \\
  That stupid shepherd lolling on his crook \\
  With deaf attention scarcely deigns to look. \\
  I bide my time, and it shall come at length, \\
  When, with a Titan's energy and strength, \\
  I'll grab a fistful of the strings, and O, \\
  The word shall suffer when I let them go! \\
 \\
Farquharson Harris \end{quote}




\section*{M}



\paragraph{MACE, n.}  A staff of office signifying authority.  Its form, that of a
heavy club, indicates its original purpose and use in dissuading from
dissent.

\paragraph{MACHINATION, n.}  The method employed by one's opponents in baffling
one's open and honorable efforts to do the right thing.

\begin{quote}   So plain the advantages of machination \\
  It constitutes a moral obligation, \\
  And honest wolves who think upon't with loathing \\
  Feel bound to don the sheep's deceptive clothing. \\
  So prospers still the diplomatic art, \\
  And Satan bows, with hand upon his heart. \\
 \\
R.S.K. \end{quote}


\paragraph{MACROBIAN, n.}  One forgotten of the gods and living to a great age.
History is abundantly supplied with examples, from Methuselah to Old
Parr, but some notable instances of longevity are less well known.  A
Calabrian peasant named Coloni, born in 1753, lived so long that he
had what he considered a glimpse of the dawn of universal peace.
Scanavius relates that he knew an archbishop who was so old that he
could remember a time when he did not deserve hanging.  In 1566 a
linen draper of Bristol, England, declared that he had lived five
hundred years, and that in all that time he had never told a lie.
There are instances of longevity ({\em macrobiosis}) in our own country.
Senator Chauncey Depew is old enough to know better.  The editor of
{\em The American}, a newspaper in New York City, has a memory that goes
back to the time when he was a rascal, but not to the fact.  The
President of the United States was born so long ago that many of the
friends of his youth have risen to high political and military
preferment without the assistance of personal merit.  The verses
following were written by a macrobian:

\begin{quote}   When I was young the world was fair \\
      And amiable and sunny. \\
  A brightness was in all the air, \\
      In all the waters, honey. \\
      The jokes were fine and funny, \\
  The statesmen honest in their views, \\
      And in their lives, as well, \\
  And when you heard a bit of news \\
      'Twas true enough to tell. \\
  Men were not ranting, shouting, reeking, \\
  Nor women "generally speaking." \\
 \\
  The Summer then was long indeed: \\
      It lasted one whole season! \\
  The sparkling Winter gave no heed \\
      When ordered by Unreason \\
      To bring the early peas on. \\
  Now, where the dickens is the sense \\
      In calling that a year \\
  Which does no more than just commence \\
      Before the end is near? \\
  When I was young the year extended \\
  From month to month until it ended. \\
  I know not why the world has changed \\
      To something dark and dreary, \\
  And everything is now arranged \\
      To make a fellow weary. \\
      The Weather Man -- I fear he \\
  Has much to do with it, for, sure, \\
      The air is not the same: \\
  It chokes you when it is impure, \\
      When pure it makes you lame. \\
  With windows closed you are asthmatic; \\
  Open, neuralgic or sciatic. \\
 \\
  Well, I suppose this new regime \\
      Of dun degeneration \\
  Seems eviler than it would seem \\
      To a better observation, \\
      And has for compensation \\
  Some blessings in a deep disguise \\
      Which mortal sight has failed \\
  To pierce, although to angels' eyes \\
      They're visible unveiled. \\
  If Age is such a boon, good land! \\
  He's costumed by a master hand! \\
 \\
Venable Strigg \end{quote}


\paragraph{MAD, adj.}  Affected with a high degree of intellectual independence;
not conforming to standards of thought, speech and action derived by
the conformants from study of themselves; at odds with the majority;
in short, unusual.  It is noteworthy that persons are pronounced mad
by officials destitute of evidence that themselves are sane.  For
illustration, this present (and illustrious) lexicographer is no
firmer in the faith of his own sanity than is any inmate of any
madhouse in the land; yet for aught he knows to the contrary, instead
of the lofty occupation that seems to him to be engaging his powers he
may really be beating his hands against the window bars of an asylum
and declaring himself Noah Webster, to the innocent delight of many
thoughtless spectators.

\paragraph{MAGDALENE, n.}  An inhabitant of Magdala.  Popularly, a woman found
out.  This definition of the word has the authority of ignorance, Mary
of Magdala being another person than the penitent woman mentioned by
St. Luke.  It has also the official sanction of the governments of
Great Britain and the United States.  In England the word is
pronounced Maudlin, whence maudlin, adjective, unpleasantly
sentimental.  With their Maudlin for Magdalene, and their Bedlam for
Bethlehem, the English may justly boast themselves the greatest of
revisers.

\paragraph{MAGIC, n.}  An art of converting superstition into coin.  There are
other arts serving the same high purpose, but the discreet
lexicographer does not name them.

\paragraph{MAGNET, n.}  Something acted upon by magnetism.

\paragraph{MAGNETISM, n.}  Something acting upon a magnet.
\subparagraph{}   The two definitions immediately foregoing are condensed from the
works of one thousand eminent scientists, who have illuminated the
subject with a great white light, to the inexpressible advancement of
human knowledge.

\paragraph{MAGNIFICENT, adj.}  Having a grandeur or splendor superior to that to
which the spectator is accustomed, as the ears of an ass, to a rabbit,
or the glory of a glowworm, to a maggot.

\paragraph{MAGNITUDE, n.}  Size.  Magnitude being purely relative, nothing is
large and nothing small.  If everything in the universe were increased
in bulk one thousand diameters nothing would be any larger than it was
before, but if one thing remain unchanged all the others would be
larger than they had been.  To an understanding familiar with the
relativity of magnitude and distance the spaces and masses of the
astronomer would be no more impressive than those of the microscopist.
For anything we know to the contrary, the visible universe may be a
small part of an atom, with its component ions, floating in the life-
fluid (luminiferous ether) of some animal.  Possibly the wee creatures
peopling the corpuscles of our own blood are overcome with the proper
emotion when contemplating the unthinkable distance from one of these
to another.

\paragraph{MAGPIE, n.}  A bird whose thievish disposition suggested to someone
that it might be taught to talk.

\paragraph{MAIDEN, n.}  A young person of the unfair sex addicted to clewless
conduct and views that madden to crime.  The genus has a wide
geographical distribution, being found wherever sought and deplored
wherever found.  The maiden is not altogether unpleasing to the eye,
nor (without her piano and her views) insupportable to the ear, though
in respect to comeliness distinctly inferior to the rainbow, and, with
regard to the part of her that is audible, bleating out of the field
by the canary -- which, also, is more portable.

\begin{quote}   A lovelorn maiden she sat and sang -- \\
      This quaint, sweet song sang she; \\
  "It's O for a youth with a football bang \\
      And a muscle fair to see! \\
              The Captain he \\
              Of a team to be! \\
  On the gridiron he shall shine, \\
  A monarch by right divine, \\
      And never to roast on it -- me!" \\
 \\
Opoline Jones \end{quote}


\paragraph{MAJESTY, n.}  The state and title of a king.  Regarded with a just
contempt by the Most Eminent Grand Masters, Grand Chancellors, Great
Incohonees and Imperial Potentates of the ancient and honorable orders
of republican America.

\paragraph{MALE, n.}  A member of the unconsidered, or negligible sex.  The male
of the human race is commonly known (to the female) as Mere Man.  The
genus has two varieties:  good providers and bad providers.

\paragraph{MALEFACTOR, n.}  The chief factor in the progress of the human race.

\paragraph{MALTHUSIAN, adj.}  Pertaining to Malthus and his doctrines.  Malthus
believed in artificially limiting population, but found that it could
not be done by talking.  One of the most practical exponents of the
Malthusian idea was Herod of Judea, though all the famous soldiers
have been of the same way of thinking.

\paragraph{MAMMALIA, n.pl.}  A family of vertebrate animals whose females in a
state of nature suckle their young, but when civilized and enlightened
put them out to nurse, or use the bottle.

\paragraph{MAMMON, n.}  The god of the world's leading religion.  The chief temple
is in the holy city of New York.

\begin{quote}   He swore that all other religions were gammon, \\
  And wore out his knees in the worship of Mammon. \\
 \\
Jared Oopf \end{quote}


\paragraph{MAN, n.}  An animal so lost in rapturous contemplation of what he
thinks he is as to overlook what he indubitably ought to be.  His
chief occupation is extermination of other animals and his own
species, which, however, multiplies with such insistent rapidity as to
infest the whole habitable earth and Canada.

\begin{quote}   When the world was young and Man was new, \\
      And everything was pleasant, \\
  Distinctions Nature never drew \\
      'Mongst kings and priest and peasant. \\
      We're not that way at present, \\
  Save here in this Republic, where \\
      We have that old regime, \\
  For all are kings, however bare \\
      Their backs, howe'er extreme \\
  Their hunger.  And, indeed, each has a voice \\
  To accept the tyrant of his party's choice. \\
 \\
  A citizen who would not vote, \\
      And, therefore, was detested, \\
  Was one day with a tarry coat \\
      (With feathers backed and breasted) \\
      By patriots invested. \\
  "It is your duty," cried the crowd, \\
      "Your ballot true to cast \\
  For the man o' your choice."  He humbly bowed, \\
      And explained his wicked past: \\
  "That's what I very gladly would have done, \\
  Dear patriots, but he has never run." \\
 \\
Apperton Duke \end{quote}


\paragraph{MANES, n.}  The immortal parts of dead Greeks and Romans.  They were in
a state of dull discomfort until the bodies from which they had
exhaled were buried and burned; and they seem not to have been
particularly happy afterward.

\paragraph{MANICHEISM, n.}  The ancient Persian doctrine of an incessant warfare
between Good and Evil.  When Good gave up the fight the Persians
joined the victorious Opposition.

\paragraph{MANNA, n.}  A food miraculously given to the Israelites in the
wilderness.  When it was no longer supplied to them they settled
down and tilled the soil, fertilizing it, as a rule, with the bodies
of the original occupants.

\paragraph{MARRIAGE, n.}  The state or condition of a community consisting of a
master, a mistress and two slaves, making in all, two.

\paragraph{MARTYR, n.}  One who moves along the line of least reluctance to a
desired death.

\paragraph{MATERIAL, adj.}  Having an actual existence, as distinguished from an
imaginary one.  Important.

\begin{quote}   Material things I know, or fell, or see; \\
  All else is immaterial to me. \\
 \\
Jamrach Holobom \end{quote}


\paragraph{MAUSOLEUM, n.}  The final and funniest folly of the rich.

\paragraph{MAYONNAISE, n.}  One of the sauces which serve the French in place of a
state religion.

\paragraph{ME, pro.}  The objectionable case of I.  The personal pronoun in
English has three cases, the dominative, the objectionable and the
oppressive.  Each is all three.

\paragraph{MEANDER, n.}  To proceed sinuously and aimlessly.  The word is the
ancient name of a river about one hundred and fifty miles south of
Troy, which turned and twisted in the effort to get out of hearing
when the Greeks and Trojans boasted of their prowess.

\paragraph{MEDAL, n.}  A small metal disk given as a reward for virtues,
attainments or services more or less authentic.
\subparagraph{}   It is related of Bismark, who had been awarded a medal for
gallantly rescuing a drowning person, that, being asked the meaning of
the medal, he replied:  "I save lives sometimes."  And sometimes he
didn't.

\paragraph{MEDICINE, n.}  A stone flung down the Bowery to kill a dog in Broadway.

\paragraph{MEEKNESS, n.}  Uncommon patience in planning a revenge that is worth
while.

\begin{quote}   M is for Moses, \\
      Who slew the Egyptian. \\
  As sweet as a rose is \\
  The meekness of Moses. \\
  No monument shows his \\
      Post-mortem inscription, \\
  But M is for Moses \\
      Who slew the Egyptian. \\
 \\
{\em The Biographical Alphabet} \end{quote}

\paragraph{MEERSCHAUM, n.}  (Literally, seafoam, and by many erroneously supposed
to be made of it.)  A fine white clay, which for convenience in
coloring it brown is made into tobacco pipes and smoked by the workmen
engaged in that industry.  The purpose of coloring it has not been
disclosed by the manufacturers.

\begin{quote}   There was a youth (you've heard before, \\
      This woeful tale, may be), \\
  Who bought a meerschaum pipe and swore \\
      That color it would he! \\
 \\
  He shut himself from the world away, \\
      Nor any soul he saw. \\
  He smoke by night, he smoked by day, \\
      As hard as he could draw. \\
 \\
  His dog died moaning in the wrath \\
      Of winds that blew aloof; \\
  The weeds were in the gravel path, \\
      The owl was on the roof. \\
 \\
  "He's gone afar, he'll come no more," \\
      The neighbors sadly say. \\
  And so they batter in the door \\
      To take his goods away. \\
 \\
  Dead, pipe in mouth, the youngster lay, \\
      Nut-brown in face and limb. \\
  "That pipe's a lovely white," they say, \\
      "But it has colored him!" \\
 \\
  The moral there's small need to sing -- \\
      'Tis plain as day to you: \\
  Don't play your game on any thing \\
      That is a gamester too. \\
 \\
Martin Bulstrode \end{quote}


\paragraph{MENDACIOUS, adj.}  Addicted to rhetoric.

\paragraph{MERCHANT, n.}  One engaged in a commercial pursuit.  A commercial
pursuit is one in which the thing pursued is a dollar.

\paragraph{MERCY, n.}  An attribute beloved of detected offenders.

\paragraph{MESMERISM, n.}  Hypnotism before it wore good clothes, kept a carriage
and asked Incredulity to dinner.

\paragraph{METROPOLIS, n.}  A stronghold of provincialism.

\paragraph{MILLENNIUM, n.}  The period of a thousand years when the lid is to be
screwed down, with all reformers on the under side.

\paragraph{MIND, n.}  A mysterious form of matter secreted by the brain.  Its
chief activity consists in the endeavor to ascertain its own nature,
the futility of the attempt being due to the fact that it has nothing
but itself to know itself with.  From the Latin {\em mens}, a fact unknown
to that honest shoe-seller, who, observing that his learned competitor
over the way had displayed the motto "{\em Mens conscia recti},"
emblazoned his own front with the words "Men's, women's and children's
conscia recti."

\paragraph{MINE, adj.}  Belonging to me if I can hold or seize it.

\paragraph{MINISTER, n.}  An agent of a higher power with a lower responsibility.
In diplomacy and officer sent into a foreign country as the visible
embodiment of his sovereign's hostility.  His principal qualification
is a degree of plausible inveracity next below that of an ambassador.

\paragraph{MINOR, adj.}  Less objectionable.

\paragraph{MINSTREL, adj.}  Formerly a poet, singer or musician; now a nigger with
a color less than skin deep and a humor more than flesh and blood can
bear.

\paragraph{MIRACLE, n.}  An act or event out of the order of nature and
unaccountable, as beating a normal hand of four kings and an ace with
four aces and a king.

\paragraph{MISCREANT, n.}  A person of the highest degree of unworth.
Etymologically, the word means unbeliever, and its present
signification may be regarded as theology's noblest contribution to
the development of our language.

\paragraph{MISDEMEANOR, n.}  An infraction of the law having less dignity than a
felony and constituting no claim to admittance into the best criminal
society.

\begin{quote}   By misdemeanors he essays to climb \\
  Into the aristocracy of crime. \\
  O, woe was him! -- with manner chill and grand \\
  "Captains of industry" refused his hand, \\
  "Kings of finance" denied him recognition \\
  And "railway magnates" jeered his low condition. \\
  He robbed a bank to make himself respected. \\
  They still rebuffed him, for he was detected. \\
 \\
S.V. Hanipur \end{quote}


\paragraph{MISERICORDE, n.}  A dagger which in mediaeval warfare was used by the
foot soldier to remind an unhorsed knight that he was mortal.

\paragraph{MISFORTUNE, n.}  The kind of fortune that never misses.

\paragraph{MISS, n.}  The title with which we brand unmarried women to indicate
that they are in the market.  Miss, Missis (Mrs.) and Mister (Mr.) are
the three most distinctly disagreeable words in the language, in sound
and sense.  Two are corruptions of Mistress, the other of Master.  In
the general abolition of social titles in this our country they
miraculously escaped to plague us.  If we must have them let us be
consistent and give one to the unmarried man.  I venture to suggest
Mush, abbreviated to Mh.

\paragraph{MOLECULE, n.}  The ultimate, indivisible unit of matter.  It is
distinguished from the corpuscle, also the ultimate, indivisible unit
of matter, by a closer resemblance to the atom, also the ultimate,
indivisible unit of matter.  Three great scientific theories of the
structure of the universe are the molecular, the corpuscular and the
atomic.  A fourth affirms, with Haeckel, the condensation of
precipitation of matter from ether -- whose existence is proved by the
condensation of precipitation.  The present trend of scientific
thought is toward the theory of ions.  The ion differs from the
molecule, the corpuscle and the atom in that it is an ion.  A fifth
theory is held by idiots, but it is doubtful if they know any more
about the matter than the others.

\paragraph{MONAD, n.}  The ultimate, indivisible unit of matter.  (See
{\em Molecule}.)  According to Leibnitz, as nearly as he seems willing to
be understood, the monad has body without bulk, and mind without
manifestation -- Leibnitz knows him by the innate power of
considering.  He has founded upon him a theory of the universe, which
the creature bears without resentment, for the monad is a gentleman.
Small as he is, the monad contains all the powers and possibilities
needful to his evolution into a German philosopher of the first class
-- altogether a very capable little fellow.  He is not to be
confounded with the microbe, or bacillus; by its inability to discern
him, a good microscope shows him to be of an entirely distinct
species.

\paragraph{MONARCH, n.}  A person engaged in reigning.  Formerly the monarch
ruled, as the derivation of the word attests, and as many subjects
have had occasion to learn.  In Russia and the Orient the monarch has
still a considerable influence in public affairs and in the
disposition of the human head, but in western Europe political
administration is mostly entrusted to his ministers, he being
somewhat preoccupied with reflections relating to the status of his
own head.

\paragraph{MONARCHICAL GOVERNMENT, n.}  Government.

\paragraph{MONDAY, n.}  In Christian countries, the day after the baseball game.

\paragraph{MONEY, n.}  A blessing that is of no advantage to us excepting when we
part with it.  An evidence of culture and a passport to polite
society.  Supportable property.

\paragraph{MONKEY, n.}  An arboreal animal which makes itself at home in
genealogical trees.

\paragraph{MONOSYLLABIC, adj.}  Composed of words of one syllable, for literary
babes who never tire of testifying their delight in the vapid compound
by appropriate googoogling.  The words are commonly Saxon -- that is
to say, words of a barbarous people destitute of ideas and incapable
of any but the most elementary sentiments and emotions.

\begin{quote}   The man who writes in Saxon \\
  Is the man to use an ax on \\
 \\
Judibras \end{quote}


\paragraph{MONSIGNOR, n.}  A high ecclesiastical title, of which the Founder of
our religion overlooked the advantages.

\paragraph{MONUMENT, n.}  A structure intended to commemorate something which
either needs no commemoration or cannot be commemorated.

\subparagraph{}   The bones of Agammemnon are a show,
  And ruined is his royal monument,
but Agammemnon's fame suffers no diminution in consequence.  The
monument custom has its {\em reductiones ad absurdum} in monuments "to the
unknown dead" -- that is to say, monuments to perpetuate the memory of
those who have left no memory.

\paragraph{MORAL, adj.}  Conforming to a local and mutable standard of right.
Having the quality of general expediency.

\subparagraph{}       It is sayd there be a raunge of mountaynes in the Easte, on
one syde of the which certayn conducts are immorall, yet on the other
syde they are holden in good esteeme; wherebye the mountayneer is much
conveenyenced, for it is given to him to goe downe eyther way and act
as it shall suite his moode, withouten offence.

{\em Gooke's Meditations}


\paragraph{MORE, adj.}  The comparative degree of too much.

\paragraph{MOUSE, n.}  An animal which strews its path with fainting women.  As in
Rome Christians were thrown to the lions, so centuries earlier in
Otumwee, the most ancient and famous city of the world, female
heretics were thrown to the mice.  Jakak-Zotp, the historian, the only
Otumwump whose writings have descended to us, says that these martyrs
met their death with little dignity and much exertion.  He even
attempts to exculpate the mice (such is the malice of bigotry) by
declaring that the unfortunate women perished, some from exhaustion,
some of broken necks from falling over their own feet, and some from
lack of restoratives.  The mice, he avers, enjoyed the pleasures of
the chase with composure.  But if "Roman history is nine-tenths
lying," we can hardly expect a smaller proportion of that rhetorical
figure in the annals of a people capable of so incredible cruelty to a
lovely women; for a hard heart has a false tongue.

\paragraph{MOUSQUETAIRE, n.}  A long glove covering a part of the arm.  Worn in
New Jersey.  But "mousquetaire" is a might poor way to spell
muskeeter.

\paragraph{MOUTH, n.}  In man, the gateway to the soul; in woman, the outlet of
the heart.

\paragraph{MUGWUMP, n.}  In politics one afflicted with self-respect and addicted
to the vice of independence.  A term of contempt.

\paragraph{MULATTO, n.}  A child of two races, ashamed of both.

\paragraph{MULTITUDE, n.}  A crowd; the source of political wisdom and virtue.  In
a republic, the object of the statesman's adoration.  "In a multitude
of counsellors there is wisdom," saith the proverb.  If many men of
equal individual wisdom are wiser than any one of them, it must be
that they acquire the excess of wisdom by the mere act of getting
together.  Whence comes it?  Obviously from nowhere -- as well say
that a range of mountains is higher than the single mountains
composing it.  A multitude is as wise as its wisest member if it obey
him; if not, it is no wiser than its most foolish.

\paragraph{MUMMY, n.}  An ancient Egyptian, formerly in universal use among modern
civilized nations as medicine, and now engaged in supplying art with
an excellent pigment.  He is handy, too, in museums in gratifying the
vulgar curiosity that serves to distinguish man from the lower
animals.

\begin{quote}   By means of the Mummy, mankind, it is said, \\
  Attests to the gods its respect for the dead. \\
  We plunder his tomb, be he sinner or saint, \\
  Distil him for physic and grind him for paint, \\
  Exhibit for money his poor, shrunken frame, \\
  And with levity flock to the scene of the shame. \\
  \\
  O, tell me, ye gods, for the use of my rhyme: \\
  For respecting the dead what's the limit of time? \\
 \\
Scopas Brune \end{quote}


\paragraph{MUSTANG, n.}  An indocile horse of the western plains.  In English
society, the American wife of an English nobleman.

\paragraph{MYRMIDON, n.}  A follower of Achilles -- particularly when he didn't
lead.

\paragraph{MYTHOLOGY, n.}  The body of a primitive people's beliefs concerning its
origin, early history, heroes, deities and so forth, as distinguished
from the true accounts which it invents later.



\section*{N}



\paragraph{NECTAR, n.}  A drink served at banquets of the Olympian deities.  The
secret of its preparation is lost, but the modern Kentuckians believe
that they come pretty near to a knowledge of its chief ingredient.

\begin{quote}   Juno drank a cup of nectar, \\
  But the draught did not affect her. \\
  Juno drank a cup of rye -- \\
  Then she bad herself good-bye. \\
 \\
J.G. \end{quote}


\paragraph{NEGRO, n.}  The {\em piece de resistance} in the American political
problem.  Representing him by the letter n, the Republicans begin to
build their equation thus:  "Let n = the white man."  This, however,
appears to give an unsatisfactory solution.

\paragraph{NEIGHBOR, n.}  One whom we are commanded to love as ourselves, and who
does all he knows how to make us disobedient.

\paragraph{NEPOTISM, n.}  Appointing your grandmother to office for the good of
the party.

\paragraph{NEWTONIAN, adj.}  Pertaining to a philosophy of the universe invented
by Newton, who discovered that an apple will fall to the ground, but
was unable to say why.  His successors and disciples have advanced so
far as to be able to say when.

\paragraph{NIHILIST, n.}  A Russian who denies the existence of anything but
Tolstoi.  The leader of the school is Tolstoi.

\paragraph{NIRVANA, n.}  In the Buddhist religion, a state of pleasurable
annihilation awarded to the wise, particularly to those wise enough to
understand it.

\paragraph{NOBLEMAN, n.}  Nature's provision for wealthy American minds ambitious
to incur social distinction and suffer high life.

\paragraph{NOISE, n.}  A stench in the ear.  Undomesticated music.  The chief
product and authenticating sign of civilization.

\paragraph{NOMINATE, v.}  To designate for the heaviest political assessment.  To
put forward a suitable person to incur the mudgobbling and deadcatting
of the opposition.

\paragraph{NOMINEE, n.}  A modest gentleman shrinking from the distinction of
private life and diligently seeking the honorable obscurity of public
office.

\paragraph{NON-COMBATANT, n.}  A dead Quaker.

\paragraph{NONSENSE, n.}  The objections that are urged against this excellent
dictionary.

\paragraph{NOSE, n.}  The extreme outpost of the face.  From the circumstance that
great conquerors have great noses, Getius, whose writings antedate the
age of humor, calls the nose the organ of quell.  It has been observed
that one's nose is never so happy as when thrust into the affairs of
others, from which some physiologists have drawn the inference that
the nose is devoid of the sense of smell.

\begin{quote}       There's a man with a Nose, \\
      And wherever he goes \\
  The people run from him and shout: \\
      "No cotton have we \\
      For our ears if so be \\
  He blow that interminous snout!" \\
 \\
      So the lawyers applied \\
      For injunction.  "Denied," \\
  Said the Judge:  "the defendant prefixion, \\
      Whate'er it portend, \\
      Appears to transcend \\
  The bounds of this court's jurisdiction." \\
 \\
Arpad Singiny \end{quote}


\paragraph{NOTORIETY, n.}  The fame of one's competitor for public honors.  The
kind of renown most accessible and acceptable to mediocrity.  A
Jacob's-ladder leading to the vaudeville stage, with angels ascending
and descending.

\paragraph{NOUMENON, n.}  That which exists, as distinguished from that which
merely seems to exist, the latter being a phenomenon.  The noumenon is
a bit difficult to locate; it can be apprehended only be a process of
reasoning -- which is a phenomenon.  Nevertheless, the discovery and
exposition of noumena offer a rich field for what Lewes calls "the
endless variety and excitement of philosophic thought."  Hurrah
(therefore) for the noumenon!

\paragraph{NOVEL, n.}  A short story padded.  A species of composition bearing the
same relation to literature that the panorama bears to art.  As it is
too long to be read at a sitting the impressions made by its
successive parts are successively effaced, as in the panorama.  Unity,
totality of effect, is impossible; for besides the few pages last read
all that is carried in mind is the mere plot of what has gone before.
To the romance the novel is what photography is to painting.  Its
distinguishing principle, probability, corresponds to the literal
actuality of the photograph and puts it distinctly into the category
of reporting; whereas the free wing of the romancer enables him to
mount to such altitudes of imagination as he may be fitted to attain;
and the first three essentials of the literary art are imagination,
imagination and imagination.  The art of writing novels, such as it
was, is long dead everywhere except in Russia, where it is new.  Peace
to its ashes -- some of which have a large sale.

\paragraph{NOVEMBER, n.}  The eleventh twelfth of a weariness.



\section*{O}



\paragraph{OATH, n.}  In law, a solemn appeal to the Deity, made binding upon the
conscience by a penalty for perjury.

\paragraph{OBLIVION, n.}  The state or condition in which the wicked cease from
struggling and the dreary are at rest.  Fame's eternal dumping ground.
Cold storage for high hopes.  A place where ambitious authors meet
their works without pride and their betters without envy.  A dormitory
without an alarm clock.

\paragraph{OBSERVATORY, n.}  A place where astronomers conjecture away the guesses
of their predecessors.

\paragraph{OBSESSED, p.p.}  Vexed by an evil spirit, like the Gadarene swine and
other critics.  Obsession was once more common than it is now.
Arasthus tells of a peasant who was occupied by a different devil for
every day in the week, and on Sundays by two.  They were frequently
seen, always walking in his shadow, when he had one, but were finally
driven away by the village notary, a holy man; but they took the
peasant with them, for he vanished utterly.  A devil thrown out of a
woman by the Archbishop of Rheims ran through the trees, pursued by a
hundred persons, until the open country was reached, where by a leap
higher than a church spire he escaped into a bird.  A chaplain in
Cromwell's army exorcised a soldier's obsessing devil by throwing the
soldier into the water, when the devil came to the surface.  The
soldier, unfortunately, did not.

\paragraph{OBSOLETE, adj.}  No longer used by the timid.  Said chiefly of words.
A word which some lexicographer has marked obsolete is ever thereafter
an object of dread and loathing to the fool writer, but if it is a
good word and has no exact modern equivalent equally good, it is good
enough for the good writer.  Indeed, a writer's attitude toward
"obsolete" words is as true a measure of his literary ability as
anything except the character of his work.  A dictionary of obsolete
and obsolescent words would not only be singularly rich in strong and
sweet parts of speech; it would add large possessions to the
vocabulary of every competent writer who might not happen to be a
competent reader.

\paragraph{OBSTINATE, adj.}  Inaccessible to the truth as it is manifest in the
splendor and stress of our advocacy.
\begin{quote}   The popular type and exponent of obstinacy is the mule, a most \\
intelligent animal. \end{quote}

\paragraph{OCCASIONAL, adj.}  Afflicting us with greater or less frequency.  That,
however, is not the sense in which the word is used in the phrase
"occasional verses," which are verses written for an "occasion," such
as an anniversary, a celebration or other event.  True, they afflict
us a little worse than other sorts of verse, but their name has no
reference to irregular recurrence.

\paragraph{OCCIDENT, n.}  The part of the world lying west (or east) of the
Orient.  It is largely inhabited by Christians, a powerful subtribe of
the Hypocrites, whose principal industries are murder and cheating,
which they are pleased to call "war" and "commerce."  These, also, are
the principal industries of the Orient.

\paragraph{OCEAN, n.}  A body of water occupying about two-thirds of a world made
for man -- who has no gills.

\paragraph{OFFENSIVE, adj.}  Generating disagreeable emotions or sensations, as
the advance of an army against its enemy.
\subparagraph{}   "Were the enemy's tactics offensive?" the king asked.  "I should
say so!" replied the unsuccessful general.  "The blackguard wouldn't
come out of his works!"

\paragraph{OLD, adj.}  In that stage of usefulness which is not inconsistent with
general inefficiency, as an {\em old man}.  Discredited by lapse of time
and offensive to the popular taste, as an {\em old} book.

\begin{quote}   "Old books?  The devil take them!" Goby said. \\
  "Fresh every day must be my books and bread." \\
  Nature herself approves the Goby rule \\
  And gives us every moment a fresh fool. \\
 \\
Harley Shum \end{quote}


\paragraph{OLEAGINOUS, adj.}  Oily, smooth, sleek.
\subparagraph{}   Disraeli once described the manner of Bishop Wilberforce as
"unctuous, oleaginous, saponaceous."  And the good prelate was ever
afterward known as Soapy Sam.  For every man there is something in the
vocabulary that would stick to him like a second skin.  His enemies
have only to find it.

\paragraph{OLYMPIAN, adj.}  Relating to a mountain in Thessaly, once inhabited by
gods, now a repository of yellowing newspapers, beer bottles and
mutilated sardine cans, attesting the presence of the tourist and his
appetite.

\begin{quote}   His name the smirking tourist scrawls \\
  Upon Minerva's temple walls, \\
  Where thundered once Olympian Zeus, \\
  And marks his appetite's abuse. \\
 \\
Averil Joop \end{quote}


\paragraph{OMEN, n.}  A sign that something will happen if nothing happens.

\paragraph{ONCE, adv.}  Enough.

\paragraph{OPERA, n.}  A play representing life in another world, whose
inhabitants have no speech but song, no motions but gestures and no
postures but attitudes.  All acting is simulation, and the word
{\em simulation} is from {\em simia}, an ape; but in opera the actor takes for
his model {\em Simia audibilis} (or {\em Pithecanthropos stentor}) -- the ape
that howls.

\begin{quote}   The actor apes a man -- at least in shape; \\
  The opera performer apes and ape.  \end{quote}

\paragraph{OPIATE, n.}  An unlocked door in the prison of Identity.  It leads into
the jail yard.

\paragraph{OPPORTUNITY, n.}  A favorable occasion for grasping a disappointment.

\paragraph{OPPOSE, v.}  To assist with obstructions and objections.

\begin{quote}   How lonely he who thinks to vex \\
  With bandinage the Solemn Sex! \\
  Of levity, Mere Man, beware; \\
  None but the Grave deserve the Unfair. \\
 \\
Percy P. Orminder \end{quote}


\paragraph{OPPOSITION, n.}  In politics the party that prevents the Government from
running amuck by hamstringing it.
\subparagraph{}   The King of Ghargaroo, who had been abroad to study the science of
government, appointed one hundred of his fattest subjects as members
of a parliament to make laws for the collection of revenue.  Forty of
these he named the Party of Opposition and had his Prime Minister
carefully instruct them in their duty of opposing every royal measure.
Nevertheless, the first one that was submitted passed unanimously.
Greatly displeased, the King vetoed it, informing the Opposition that
if they did that again they would pay for their obstinacy with their
heads.  The entire forty promptly disemboweled themselves.
\subparagraph{}   "What shall we do now?" the King asked.  "Liberal institutions
cannot be maintained without a party of Opposition."
\subparagraph{}   "Splendor of the universe," replied the Prime Minister, "it is
true these dogs of darkness have no longer their credentials, but all
is not lost.  Leave the matter to this worm of the dust."
\subparagraph{}   So the Minister had the bodies of his Majesty's Opposition 
embalmed and stuffed with straw, put back into the seats of power and 
nailed there.  Forty votes were recorded against every bill and the
nation prospered.  But one day a bill imposing a tax on warts was
defeated -- the members of the Government party had not been nailed to
their seats!  This so enraged the King that the Prime Minister was put
to death, the parliament was dissolved with a battery of artillery,
and government of the people, by the people, for the people perished
from Ghargaroo.

\paragraph{OPTIMISM, n.}  The doctrine, or belief, that everything is beautiful,
including what is ugly, everything good, especially the bad, and
everything right that is wrong.  It is held with greatest tenacity by
those most accustomed to the mischance of falling into adversity, and
is most acceptably expounded with the grin that apes a smile.  Being a
blind faith, it is inaccessible to the light of disproof -- an
intellectual disorder, yielding to no treatment but death.  It is
hereditary, but fortunately not contagious.

\paragraph{OPTIMIST, n.}  A proponent of the doctrine that black is white.
\begin{quote}   A pessimist applied to God for relief.\\
  \\
  "Ah, you wish me to restore your hope and cheerfulness," said God.\\
  "No," replied the petitioner, "I wish you to create something that
would justify them." \\
   "The world is all created," said God, "but you have overlooked
something -- the mortality of the optimist."
\end{quote}

\paragraph{ORATORY, n.}  A conspiracy between speech and action to cheat the
understanding.  A tyranny tempered by stenography.

\paragraph{ORPHAN, n.}  A living person whom death has deprived of the power of
filial ingratitude -- a privation appealing with a particular
eloquence to all that is sympathetic in human nature.  When young the
orphan is commonly sent to an asylum, where by careful cultivation of
its rudimentary sense of locality it is taught to know its place.  It
is then instructed in the arts of dependence and servitude and
eventually turned loose to prey upon the world as a bootblack or
scullery maid.

\paragraph{ORTHODOX, n.}  An ox wearing the popular religious joke.

\paragraph{ORTHOGRAPHY, n.}  The science of spelling by the eye instead of the
ear.  Advocated with more heat than light by the outmates of every
asylum for the insane.  They have had to concede a few things since
the time of Chaucer, but are none the less hot in defence of those to
be conceded hereafter.

\begin{quote}   A spelling reformer indicted \\
  For fudge was before the court cicted. \\
      The judge said:  "Enough -- \\
      His candle we'll snough, \\
  And his sepulchre shall not be whicted."
\end{quote}
\paragraph{OSTRICH, n.}  A large bird to which (for its sins, doubtless) nature
has denied that hinder toe in which so many pious naturalists have
seen a conspicuous evidence of design.  The absence of a good working
pair of wings is no defect, for, as has been ingeniously pointed out,
the ostrich does not fly.

\paragraph{OTHERWISE, adv.}  No better.

\paragraph{OUTCOME, n.}  A particular type of disappointment.  By the kind of
intelligence that sees in an exception a proof of the rule the wisdom
of an act is judged by the outcome, the result.  This is immortal
nonsense; the wisdom of an act is to be juded by the light that the
doer had when he performed it.

\paragraph{OUTDO, v.t.}  To make an enemy.

\paragraph{OUT-OF-DOORS, n.}  That part of one's environment upon which no
government has been able to collect taxes.  Chiefly useful to inspire
poets.

\begin{quote}   I climbed to the top of a mountain one day \\
      To see the sun setting in glory, \\
  And I thought, as I looked at his vanishing ray, \\
      Of a perfectly splendid story. \\
 \\
  'Twas about an old man and the ass he bestrode \\
      Till the strength of the beast was o'ertested; \\
  Then the man would carry him miles on the road \\
      Till Neddy was pretty well rested. \\
 \\
  The moon rising solemnly over the crest \\
      Of the hills to the east of my station \\
  Displayed her broad disk to the darkening west \\
      Like a visible new creation. \\
 \\
  And I thought of a joke (and I laughed till I cried) \\
      Of an idle young woman who tarried \\
  About a church-door for a look at the bride, \\
      Although 'twas herself that was married. \\
 \\
  To poets all Nature is pregnant with grand \\
      Ideas -- with thought and emotion. \\
  I pity the dunces who don't understand \\
      The speech of earth, heaven and ocean. \\
 \\
Stromboli Smith \end{quote}


\paragraph{OVATION, n.}  n ancient Rome, a definite, formal pageant in honor of
one who had been disserviceable to the enemies of the nation.  A
lesser "triumph."  In modern English the word is improperly used to
signify any loose and spontaneous expression of popular homage to the
hero of the hour and place.

\begin{quote}   "I had an ovation!" the actor man said, \\
      But I thought it uncommonly queer, \\
  That people and critics by him had been led \\
          By the ear. \\
 \\
  The Latin lexicon makes his absurd \\
      Assertion as plain as a peg; \\
  In "ovum" we find the true root of the word. \\
          It means egg. \\
 \\
Dudley Spink \end{quote}


\paragraph{OVEREAT, v.}  To dine.

\begin{quote}   Hail, Gastronome, Apostle of Excess, \\
  Well skilled to overeat without distress! \\
  Thy great invention, the unfatal feast, \\
  Shows Man's superiority to Beast. \\
 \\
John Boop \end{quote}


\paragraph{OVERWORK, n.}  A dangerous disorder affecting high public functionaries
who want to go fishing.

\paragraph{OWE, v.}  To have (and to hold) a debt.  The word formerly signified
not indebtedness, but possession; it meant "own," and in the minds of
debtors there is still a good deal of confusion between assets and
liabilities.

\paragraph{OYSTER, n.}  A slimy, gobby shellfish which civilization gives men the
hardihood to eat without removing its entrails!  The shells are
sometimes given to the poor.



\section*{P}



\paragraph{PAIN, n.}  An uncomfortable frame of mind that may have a physical
basis in something that is being done to the body, or may be purely
mental, caused by the good fortune of another.

\paragraph{PAINTING, n.}  The art of protecting flat surfaces from the weather and
exposing them to the critic.
\subparagraph{}   Formerly, painting and sculpture were combined in the same work:
the ancients painted their statues.  The only present alliance between
the two arts is that the modern painter chisels his patrons.

\paragraph{PALACE, n.}  A fine and costly residence, particularly that of a great
official.  The residence of a high dignitary of the Christian Church
is called a palace; that of the Founder of his religion was known as a
field, or wayside.  There is progress.

\paragraph{PALM, n.}  A species of tree having several varieties, of which the
familiar "itching palm" ({\em Palma hominis}) is most widely distributed
and sedulously cultivated.  This noble vegetable exudes a kind of
invisible gum, which may be detected by applying to the bark a piece
of gold or silver.  The metal will adhere with remarkable tenacity.
The fruit of the itching palm is so bitter and unsatisfying that a
considerable percentage of it is sometimes given away in what are known
as "benefactions."

\paragraph{PALMISTRY, n.}  The 947th method (according to Mimbleshaw's
classification) of obtaining money by false pretences.  It consists in
"reading character" in the wrinkles made by closing the hand.  The
pretence is not altogether false; character can really be read very
accurately in this way, for the wrinkles in every hand submitted
plainly spell the word "dupe."  The imposture consists in not reading
it aloud.

\paragraph{PANDEMONIUM, n.}  Literally, the Place of All the Demons.  Most of them
have escaped into politics and finance, and the place is now used as a
lecture hall by the Audible Reformer.  When disturbed by his voice the
ancient echoes clamor appropriate responses most gratifying to his
pride of distinction.

\paragraph{PANTALOONS, n.}  A nether habiliment of the adult civilized male.  The
garment is tubular and unprovided with hinges at the points of
flexion.  Supposed to have been invented by a humorist.  Called
"trousers" by the enlightened and "pants" by the unworthy.

\paragraph{PANTHEISM, n.}  The doctrine that everything is God, in
contradistinction to the doctrine that God is everything.

\paragraph{PANTOMIME, n.}  A play in which the story is told without violence to
the language.  The least disagreeable form of dramatic action.

\paragraph{PARDON, v.}  To remit a penalty and restore to the life of crime.  To
add to the lure of crime the temptation of ingratitude.

\paragraph{PASSPORT, n.}  A document treacherously inflicted upon a citizen going
abroad, exposing him as an alien and pointing him out for special
reprobation and outrage.

\paragraph{PAST, n.}  That part of Eternity with some small fraction of which we
have a slight and regrettable acquaintance.  A moving line called the
Present parts it from an imaginary period known as the Future.  These
two grand divisions of Eternity, of which the one is continually
effacing the other, are entirely unlike.  The one is dark with sorrow
and disappointment, the other bright with prosperity and joy.  The
Past is the region of sobs, the Future is the realm of song.  In the
one crouches Memory, clad in sackcloth and ashes, mumbling penitential
prayer; in the sunshine of the other Hope flies with a free wing,
beckoning to temples of success and bowers of ease.  Yet the Past is
the Future of yesterday, the Future is the Past of to-morrow.  They
are one -- the knowledge and the dream.

\paragraph{PASTIME, n.}  A device for promoting dejection.  Gentle exercise for
intellectual debility.

\paragraph{PATIENCE, n.}  A minor form of despair, disguised as a virtue.

\paragraph{PATRIOT, n.}  One to whom the interests of a part seem superior to
those of the whole.  The dupe of statesmen and the tool of conquerors.

\paragraph{PATRIOTISM, n.}  Combustible rubbish read to the torch of any one
ambitious to illuminate his name.
\subparagraph{}   In Dr. Johnson's famous dictionary patriotism is defined as the
last resort of a scoundrel.  With all due respect to an enlightened
but inferior lexicographer I beg to submit that it is the first.

\paragraph{PEACE, n.}  In international affairs, a period of cheating between two
periods of fighting.

\begin{quote}  O, what's the loud uproar assailing \\
       Mine ears without cease? \\
  'Tis the voice of the hopeful, all-hailing \\
      The horrors of peace. \\
 \\
  Ah, Peace Universal; they woo it -- \\
      Would marry it, too. \\
  If only they knew how to do it \\
      'Twere easy to do. \\
 \\
  They're working by night and by day \\
      On their problem, like moles. \\
  Have mercy, O Heaven, I pray, \\
      On their meddlesome souls! \\
 \\
Ro Amil \end{quote}


\paragraph{PEDESTRIAN, n.}  The variable (an audible) part of the roadway for an
automobile.

\paragraph{PEDIGREE, n.}  The known part of the route from an arboreal ancestor
with a swim bladder to an urban descendant with a cigarette.

\paragraph{PENITENT, adj.}  Undergoing or awaiting punishment.

\paragraph{PERFECTION, n.}  An imaginary state of quality distinguished from the
actual by an element known as excellence; an attribute of the critic.
\subparagraph{}   The editor of an English magazine having received a letter
pointing out the erroneous nature of his views and style, and signed
"Perfection," promptly wrote at the foot of the letter:  "I don't
agree with you," and mailed it to Matthew Arnold.

\paragraph{PERIPATETIC, adj.}  Walking about.  Relating to the philosophy of
Aristotle, who, while expounding it, moved from place to place in
order to avoid his pupil's objections.  A needless precaution -- they
knew no more of the matter than he.

\paragraph{PERORATION, n.}  The explosion of an oratorical rocket.  It dazzles,
but to an observer having the wrong kind of nose its most conspicuous
peculiarity is the smell of the several kinds of powder used in
preparing it.

\paragraph{PERSEVERANCE, n.}  A lowly virtue whereby mediocrity achieves an
inglorious success.

\begin{quote}   "Persevere, persevere!" cry the homilists all, \\
  Themselves, day and night, persevering to bawl. \\
  "Remember the fable of tortoise and hare -- \\
  The one at the goal while the other is -- where?" \\
  Why, back there in Dreamland, renewing his lease \\
  Of life, all his muscles preserving the peace, \\
  The goal and the rival forgotten alike, \\
  And the long fatigue of the needless hike. \\
  His spirit a-squat in the grass and the dew \\
  Of the dogless Land beyond the Stew, \\
  He sleeps, like a saint in a holy place, \\
  A winner of all that is good in a race. \\
 \\
Sukker Uffro \end{quote}


\paragraph{PESSIMISM, n.}  A philosophy forced upon the convictions of the
observer by the disheartening prevalence of the optimist with his
scarecrow hope and his unsightly smile.

\paragraph{PHILANTHROPIST, n.}  A rich (and usually bald) old gentleman who has
trained himself to grin while his conscience is picking his pocket.

\paragraph{PHILISTINE, n.}  One whose mind is the creature of its environment,
following the fashion in thought, feeling and sentiment.  He is
sometimes learned, frequently prosperous, commonly clean and always
solemn.

\paragraph{PHILOSOPHY, n.}  A route of many roads leading from nowhere to nothing.

\paragraph{PHOENIX, n.}  The classical prototype of the modern "small hot bird."

\paragraph{PHONOGRAPH, n.}  An irritating toy that restores life to dead noises.

\paragraph{PHOTOGRAPH, n.}  A picture painted by the sun without instruction in
art.  It is a little better than the work of an Apache, but not quite
so good as that of a Cheyenne.

\paragraph{PHRENOLOGY, n.}  The science of picking the pocket through the scalp.
It consists in locating and exploiting the organ that one is a dupe
with.

\paragraph{PHYSICIAN, n.}  One upon whom we set our hopes when ill and our dogs
when well.

\paragraph{PHYSIOGNOMY, n.}  The art of determining the character of another by
the resemblances and differences between his face and our own, which
is the standard of excellence.

\begin{quote}   "There is no art," says Shakespeare, foolish man, \\
      "To read the mind's construction in the face." \\
  The physiognomists his portrait scan, \\
      And say:  "How little wisdom here we trace! \\
  He knew his face disclosed his mind and heart, \\
  So, in his own defence, denied our art." \\
 \\
Lavatar Shunk \end{quote}


\paragraph{PIANO, n.}  A parlor utensil for subduing the impenitent visitor.  It
is operated by pressing the keys of the machine and the spirits of the
audience.

\paragraph{PICKANINNY, n.}  The young of the {\em Procyanthropos}, or {\em Americanus
dominans}.  It is small, black and charged with political fatalities.

\paragraph{PICTURE, n.}  A representation in two dimensions of something wearisome
in three.

\begin{quote}   "Behold great Daubert's picture here on view -- \\
  Taken from Life."  If that description's true, \\
  Grant, heavenly Powers, that I be taken, too. \\
 \\
Jali Hane \end{quote}


\paragraph{PIE, n.}  An advance agent of the reaper whose name is Indigestion.

\begin{quote}   Cold pie was highly esteemed by the remains. \\
 \\
Rev. Dr. Mucker\\
  \\
(in a funeral sermon over a British nobleman) \\
\\
   Cold pie is a detestable \\
  American comestible. \\
  That's why I'm done -- or undone -- \\
  So far from that dear London. \\
 \\
(from the headstone of a British nobleman in Kalamazoo) \end{quote}


\paragraph{PIETY, n.}  Reverence for the Supreme Being, based upon His supposed
resemblance to man.

\begin{quote}   The pig is taught by sermons and epistles \\
  To think the God of Swine has snout and bristles. \\
 \\
Judibras \end{quote}


\paragraph{PIG, n.}  An animal ({\em Porcus omnivorus}) closely allied to the human
race by the splendor and vivacity of its appetite, which, however, is
inferior in scope, for it sticks at pig.

\paragraph{PIGMY, n.}  One of a tribe of very small men found by ancient travelers
in many parts of the world, but by modern in Central Africa only.  The
Pigmies are so called to distinguish them from the bulkier Caucasians
-- who are Hogmies.

\paragraph{PILGRIM, n.}  A traveler that is taken seriously.  A Pilgrim Father was
one who, leaving Europe in 1620 because not permitted to sing psalms
through his nose, followed it to Massachusetts, where he could
personate God according to the dictates of his conscience.

\paragraph{PILLORY, n.}  A mechanical device for inflicting personal distinction
-- prototype of the modern newspaper conducted by persons of austere
virtues and blameless lives.

\paragraph{PIRACY, n.}  Commerce without its folly-swaddles, just as God made it.

\paragraph{PITIFUL, adj.}  The state of an enemy of opponent after an imaginary
encounter with oneself.

\paragraph{PITY, n.}  A failing sense of exemption, inspired by contrast.

\paragraph{PLAGIARISM, n.}  A literary coincidence compounded of a discreditable
priority and an honorable subsequence.

\paragraph{PLAGIARIZE, v.}  To take the thought or style of another writer whom
one has never, never read.

\paragraph{PLAGUE, n.}  In ancient times a general punishment of the innocent for
admonition of their ruler, as in the familiar instance of Pharaoh the
Immune.  The plague as we of to-day have the happiness to know it is
merely Nature's fortuitous manifestation of her purposeless
objectionableness.

\paragraph{PLAN, v.t.}  To bother about the best method of accomplishing an
accidental result.

\paragraph{PLATITUDE, n.}  The fundamental element and special glory of popular
literature. A thought that snores in words that smoke.  The wisdom of
a million fools in the diction of a dullard.  A fossil sentiment in
artificial rock.  A moral without the fable.  All that is mortal of a
departed truth.  A demi-tasse of milk-and-mortality.  The Pope's-nose
of a featherless peacock.  A jelly-fish withering on the shore of the
sea of thought.  The cackle surviving the egg.  A desiccated epigram.

\paragraph{PLATONIC, adj.}  Pertaining to the philosophy of Socrates.  Platonic
Love is a fool's name for the affection between a disability and a
frost.

\paragraph{PLAUDITS, n.}  Coins with which the populace pays those who tickle and
devour it.

\paragraph{PLEASE, v.}  To lay the foundation for a superstructure of imposition.

\paragraph{PLEASURE, n.}  The least hateful form of dejection.

\paragraph{PLEBEIAN, n.}  An ancient Roman who in the blood of his country stained
nothing but his hands.  Distinguished from the Patrician, who was a
saturated solution.

\paragraph{PLEBISCITE, n.}  A popular vote to ascertain the will of the sovereign.

\paragraph{PLENIPOTENTIARY, adj.}  Having full power.  A Minister Plenipotentiary
is a diplomatist possessing absolute authority on condition that he
never exert it.

\paragraph{PLEONASM, n.}  An army of words escorting a corporal of thought.

\paragraph{PLOW, n.}  An implement that cries aloud for hands accustomed to the
pen.

\paragraph{PLUNDER, v.}  To take the property of another without observing the
decent and customary reticences of theft.  To effect a change of
ownership with the candid concomitance of a brass band.  To wrest the
wealth of A from B and leave C lamenting a vanishing opportunity.

\paragraph{POCKET, n.}  The cradle of motive and the grave of conscience.  In
woman this organ is lacking; so she acts without motive, and her
conscience, denied burial, remains ever alive, confessing the sins of
others.

\paragraph{POETRY, n.}  A form of expression peculiar to the Land beyond the
Magazines.

\paragraph{POKER, n.}  A game said to be played with cards for some purpose to
this lexicographer unknown.

\paragraph{POLICE, n.}  An armed force for protection and participation.

\paragraph{POLITENESS, n.}  The most acceptable hypocrisy.

\paragraph{POLITICS, n.}  A strife of interests masquerading as a contest of
principles.  The conduct of public affairs for private advantage.

\paragraph{POLITICIAN, n.}  An eel in the fundamental mud upon which the
superstructure of organized society is reared.  When we wriggles he
mistakes the agitation of his tail for the trembling of the edifice.
As compared with the statesman, he suffers the disadvantage of being
alive.

\paragraph{POLYGAMY, n.}  A house of atonement, or expiatory chapel, fitted with
several stools of repentance, as distinguished from monogamy, which
has but one.

\paragraph{POPULIST, n.}  A fossil patriot of the early agricultural period, found
in the old red soapstone underlying Kansas; characterized by an
uncommon spread of ear, which some naturalists contend gave him the
power of flight, though Professors Morse and Whitney, pursuing
independent lines of thought, have ingeniously pointed out that had he
possessed it he would have gone elsewhere.  In the picturesque speech
of his period, some fragments of which have come down to us, he was
known as "The Matter with Kansas."

\paragraph{PORTABLE, adj.}  Exposed to a mutable ownership through vicissitudes of
possession.

\begin{quote}   His light estate, if neither he did make it \\
  Nor yet its former guardian forsake it, \\
  Is portable improperly, I take it. \\
 \\
Worgum Slupsky \end{quote}


\paragraph{PORTUGUESE, n.pl.}  A species of geese indigenous to Portugal.  They
are mostly without feathers and imperfectly edible, even when stuffed
with garlic.

\paragraph{POSITIVE, adj.}  Mistaken at the top of one's voice.

\paragraph{POSITIVISM, n.}  A philosophy that denies our knowledge of the Real and
affirms our ignorance of the Apparent.  Its longest exponent is Comte,
its broadest Mill and its thickest Spencer.

\paragraph{POSTERITY, n.}  An appellate court which reverses the judgment of a
popular author's contemporaries, the appellant being his obscure
competitor.

\paragraph{POTABLE, n.}  Suitable for drinking.  Water is said to be potable;
indeed, some declare it our natural beverage, although even they find
it palatable only when suffering from the recurrent disorder known as
thirst, for which it is a medicine.  Upon nothing has so great and
diligent ingenuity been brought to bear in all ages and in all
countries, except the most uncivilized, as upon the invention of
substitutes for water.  To hold that this general aversion to that
liquid has no basis in the preservative instinct of the race is to be
unscientific -- and without science we are as the snakes and toads.

\paragraph{POVERTY, n.}  A file provided for the teeth of the rats of reform.  The
number of plans for its abolition equals that of the reformers who
suffer from it, plus that of the philosophers who know nothing about
it.  Its victims are distinguished by possession of all the virtues
and by their faith in leaders seeking to conduct them into a
prosperity where they believe these to be unknown.

\paragraph{PRAY, v.}  To ask that the laws of the universe be annulled in behalf
of a single petitioner confessedly unworthy.

\paragraph{PRE-ADAMITE, n.}  One of an experimental and apparently unsatisfactory
race of antedated Creation and lived under conditions not easily
conceived.  Melsius believed them to have inhabited "the Void" and to
have been something intermediate between fishes and birds.  Little its
known of them beyond the fact that they supplied Cain with a wife and
theologians with a controversy.

\paragraph{PRECEDENT, n.}  In Law, a previous decision, rule or practice which, in
the absence of a definite statute, has whatever force and authority a
Judge may choose to give it, thereby greatly simplifying his task of
doing as he pleases.  As there are precedents for everything, he has
only to ignore those that make against his interest and accentuate
those in the line of his desire.  Invention of the precedent elevates
the trial-at-law from the low estate of a fortuitous ordeal to the
noble attitude of a dirigible arbitrament.

\paragraph{PRECIPITATE, adj.}  Anteprandial.

\begin{quote}   Precipitate in all, this sinner \\
  Took action first, and then his dinner. \\
 \\
Judibras \end{quote}


\paragraph{PREDESTINATION, n.}  The doctrine that all things occur according to
programme.  This doctrine should not be confused with that of
foreordination, which means that all things are programmed, but does
not affirm their occurrence, that being only an implication from other
doctrines by which this is entailed.  The difference is great enough
to have deluged Christendom with ink, to say nothing of the gore.
With the distinction of the two doctrines kept well in mind, and a
reverent belief in both, one may hope to escape perdition if spared.

\paragraph{PREDICAMENT, n.}  The wage of consistency.

\paragraph{PREDILECTION, n.}  The preparatory stage of disillusion.

\paragraph{PRE-EXISTENCE, n.}  An unnoted factor in creation.

\paragraph{PREFERENCE, n.}  A sentiment, or frame of mind, induced by the
erroneous belief that one thing is better than another.
\subparagraph{}   An ancient philosopher, expounding his conviction that life is no
better than death, was asked by a disciple why, then, he did not die.
"Because," he replied, "death is no better than life." 

   It is longer.
 
\paragraph{PREHISTORIC, adj.}  Belonging to an early period and a museum.
Antedating the art and practice of perpetuating falsehood.

\begin{quote}   He lived in a period prehistoric, \\
  When all was absurd and phantasmagoric. \\
  Born later, when Clio, celestial recorded, \\
  Set down great events in succession and order, \\
  He surely had seen nothing droll or fortuitous \\
  In anything here but the lies that she threw at us. \\
 \\
Orpheus Bowen \end{quote}


\paragraph{PREJUDICE, n.}  A vagrant opinion without visible means of support.

\paragraph{PRELATE, n.}  A church officer having a superior degree of holiness and
a fat preferment.  One of Heaven's aristocracy.  A gentleman of God.

\paragraph{PREROGATIVE, n.}  A sovereign's right to do wrong.

\paragraph{PRESBYTERIAN, n.}  One who holds the conviction that the government
authorities of the Church should be called presbyters.

\paragraph{PRESCRIPTION, n.}  A physician's guess at what will best prolong the
situation with least harm to the patient.

\paragraph{PRESENT, n.}  That part of eternity dividing the domain of
disappointment from the realm of hope.

\paragraph{PRESENTABLE, adj.}  Hideously appareled after the manner of the time
and place.
\subparagraph{}   In Boorioboola-Gha a man is presentable on occasions of ceremony
if he have his abdomen painted a bright blue and wear a cow's tail; in
New York he may, if it please him, omit the paint, but after sunset he
must wear two tails made of the wool of a sheep and dyed black.

\paragraph{PRESIDE, v.}  To guide the action of a deliberative body to a desirable
result.  In Journalese, to perform upon a musical instrument; as, "He
presided at the piccolo."

\begin{quote}   The Headliner, holding the copy in hand, \\
      Read with a solemn face: \\
  "The music was very uncommonly grand -- \\
          The best that was every provided, \\
          For our townsman Brown presided \\
      At the organ with skill and grace." \\
  The Headliner discontinued to read, \\
      And, spread the paper down \\
  On the desk, he dashed in at the top of the screed: \\
      "Great playing by President Brown." \\
 \\
Orpheus Bowen \end{quote}


\paragraph{PRESIDENCY, n.}  The greased pig in the field game of American
politics.

\paragraph{PRESIDENT, n.}  The leading figure in a small group of men of whom --
and of whom only -- it is positively known that immense numbers of
their countrymen did not want any of them for President.

\begin{quote}   If that's an honor surely 'tis a greater \\
  To have been a simple and undamned spectator. \\
  Behold in me a man of mark and note \\
  Whom no elector e'er denied a vote! -- \\
  An undiscredited, unhooted gent \\
  Who might, for all we know, be President \\
  By acclimation.  Cheer, ye varlets, cheer -- \\
  I'm passing with a wide and open ear! \\
 \\
Jonathan Fomry \end{quote}


\paragraph{PREVARICATOR, n.}  A liar in the caterpillar estate.

\paragraph{PRICE, n.}  Value, plus a reasonable sum for the wear and tear of
conscience in demanding it.

\paragraph{PRIMATE, n.}  The head of a church, especially a State church supported
by involuntary contributions.  The Primate of England is the
Archbishop of Canterbury, an amiable old gentleman, who occupies
Lambeth Palace when living and Westminster Abbey when dead.  He is
commonly dead.

\paragraph{PRISON, n.}  A place of punishments and rewards.  The poet assures us
that --

\begin{quote}   "Stone walls do not a prison make,"  \end{quote}

but a combination of the stone wall, the political parasite and the
moral instructor is no garden of sweets.

\paragraph{PRIVATE, n.}  A military gentleman with a field-marshal's baton in his
knapsack and an impediment in his hope.

\paragraph{PROBOSCIS, n.}  The rudimentary organ of an elephant which serves him
in place of the knife-and-fork that Evolution has as yet denied him.
For purposes of humor it is popularly called a trunk.
\subparagraph{}   Asked how he knew that an elephant was going on a journey, the
illustrious Jo. Miller cast a reproachful look upon his tormentor, and
answered, absently:  "When it is ajar," and threw himself from a high
promontory into the sea.  Thus perished in his pride the most famous
humorist of antiquity, leaving to mankind a heritage of woe!  No
successor worthy of the title has appeared, though Mr. Edward Bok, of
{\em The Ladies' Home Journal}, is much respected for the purity and
sweetness of his personal character.

\paragraph{PROJECTILE, n.}  The final arbiter in international disputes.  Formerly
these disputes were settled by physical contact of the disputants,
with such simple arguments as the rudimentary logic of the times could
supply -- the sword, the spear, and so forth.  With the growth of
prudence in military affairs the projectile came more and more into
favor, and is now held in high esteem by the most courageous.  Its
capital defect is that it requires personal attendance at the point of
propulsion.

\paragraph{PROOF, n.}  Evidence having a shade more of plausibility than of
unlikelihood.  The testimony of two credible witnesses as opposed to
that of only one.

\paragraph{PROOF-READER, n.}  A malefactor who atones for making your writing
nonsense by permitting the compositor to make it unintelligible.

\paragraph{PROPERTY, n.}  Any material thing, having no particular value, that may
be held by A against the cupidity of B.  Whatever gratifies the
passion for possession in one and disappoints it in all others.  The
object of man's brief rapacity and long indifference.

\paragraph{PROPHECY, n.}  The art and practice of selling one's credibility for
future delivery.

\paragraph{PROSPECT, n.}  An outlook, usually forbidding.  An expectation, usually
forbidden.

\begin{quote}   Blow, blow, ye spicy breezes -- \\
      O'er Ceylon blow your breath, \\
  Where every prospect pleases, \\
      Save only that of death. \\
 \\
Bishop Sheber \end{quote}


\paragraph{PROVIDENTIAL, adj.}  Unexpectedly and conspicuously beneficial to the
person so describing it.

\paragraph{PRUDE, n.}  A bawd hiding behind the back of her demeanor.

\paragraph{PUBLISH, n.}  In literary affairs, to become the fundamental element in
a cone of critics.

\paragraph{PUSH, n.}  One of the two things mainly conducive to success,
especially in politics.  The other is Pull.

\paragraph{PYRRHONISM, n.}  An ancient philosophy, named for its inventor.  It
consisted of an absolute disbelief in everything but Pyrrhonism.  Its
modern professors have added that.



\section*{Q}



\paragraph{QUEEN, n.}  A woman by whom the realm is ruled when there is a king,
and through whom it is ruled when there is not.

\paragraph{QUILL, n.}  An implement of torture yielded by a goose and commonly
wielded by an ass.  This use of the quill is now obsolete, but its
modern equivalent, the steel pen, is wielded by the same everlasting
Presence.

\paragraph{QUIVER, n.}  A portable sheath in which the ancient statesman and the
aboriginal lawyer carried their lighter arguments.

\begin{quote}   He extracted from his quiver, \\
      Did the controversial Roman, \\
  An argument well fitted \\
  To the question as submitted, \\
  Then addressed it to the liver, \\
      Of the unpersuaded foeman. \\
 \\
Oglum P. Boomp \end{quote}


\paragraph{QUIXOTIC, adj.}  Absurdly chivalric, like Don Quixote.  An insight into
the beauty and excellence of this incomparable adjective is unhappily
denied to him who has the misfortune to know that the gentleman's name
is pronounced Ke-ho-tay.

\begin{quote}   When ignorance from out of our lives can banish \\
  Philology, 'tis folly to know Spanish. \\
 \\
Juan Smith \end{quote}


\paragraph{QUORUM, n.}  A sufficient number of members of a deliberative body to
have their own way and their own way of having it.  In the United
States Senate a quorum consists of the chairman of the Committee on
Finance and a messenger from the White House; in the House of
Representatives, of the Speaker and the devil.

\paragraph{QUOTATION, n.}  The act of repeating erroneously the words of another.
The words erroneously repeated.

\begin{quote}   Intent on making his quotation truer, \\
  He sought the page infallible of Brewer, \\
  Then made a solemn vow that we would be \\
  Condemned eternally.  Ah, me, ah, me! \\
 \\
Stumpo Gaker \end{quote}


\paragraph{QUOTIENT, n.}  A number showing how many times a sum of money belonging
to one person is contained in the pocket of another -- usually about
as many times as it can be got there.



\section*{R}



\paragraph{RABBLE, n.}  In a republic, those who exercise a supreme authority
tempered by fraudulent elections.  The rabble is like the sacred
Simurgh, of Arabian fable -- omnipotent on condition that it do
nothing.  (The word is Aristocratese, and has no exact equivalent in
our tongue, but means, as nearly as may be, "soaring swine.")

\paragraph{RACK, n.}  An argumentative implement formerly much used in persuading
devotees of a false faith to embrace the living truth.  As a call to
the unconverted the rack never had any particular efficacy, and is now
held in light popular esteem.

\paragraph{RANK, n.}  Relative elevation in the scale of human worth.

\begin{quote}   He held at court a rank so high \\
  That other noblemen asked why. \\
  "Because," 'twas answered, "others lack \\
  His skill to scratch the royal back." \\
 \\
Aramis Jukes \end{quote}


\paragraph{RANSOM, n.}  The purchase of that which neither belongs to the seller,
nor can belong to the buyer.  The most unprofitable of investments.

\paragraph{RAPACITY, n.}  Providence without industry.  The thrift of power.

\paragraph{RAREBIT, n.}  A Welsh rabbit, in the speech of the humorless, who point
out that it is not a rabbit.  To whom it may be solemnly explained
that the comestible known as toad-in-a-hole is really not a toad, and
that {\em riz-de-veau a la financiere} is not the smile of a calf prepared
after the recipe of a she banker.

\paragraph{RASCAL, n.}  A fool considered under another aspect.

\paragraph{RASCALITY, n.}  Stupidity militant.  The activity of a clouded
intellect.

\paragraph{RASH, adj.}  Insensible to the value of our advice.

\begin{quote}   "Now lay your bet with mine, nor let \\
      These gamblers take your cash." \\
  "Nay, this child makes no bet."  "Great snakes! \\
      How can you be so rash?" \\
 \\
Bootle P. Gish \end{quote}


\paragraph{RATIONAL, adj.}  Devoid of all delusions save those of observation,
experience and reflection.

\paragraph{RATTLESNAKE, n.}  Our prostrate brother, {\em Homo ventrambulans}.

\paragraph{RAZOR, n.}  An instrument used by the Caucasian to enhance his beauty,
by the Mongolian to make a guy of himself, and by the Afro-American to
affirm his worth.

\paragraph{REACH, n.}  The radius of action of the human hand.  The area within
which it is possible (and customary) to gratify directly the
propensity to provide.

\begin{quote}   This is a truth, as old as the hills, \\
      That life and experience teach: \\
  The poor man suffers that keenest of ills, \\
      An impediment of his reach. \\
 \\
G.J. \end{quote}


\paragraph{READING, n.}  The general body of what one reads.  In our country it
consists, as a rule, of Indiana novels, short stories in "dialect" and
humor in slang.

\begin{quote}   We know by one's reading \\
  His learning and breeding; \\
  By what draws his laughter \\
  We know his Hereafter. \\
  Read nothing, laugh never -- \\
  The Sphinx was less clever! \\
 \\
Jupiter Muke \end{quote}


\paragraph{RADICALISM, n.}  The conservatism of to-morrow injected into the
affairs of to-day.

\paragraph{RADIUM, n.}  A mineral that gives off heat and stimulates the organ
that a scientist is a fool with.

\paragraph{RAILROAD, n.}  The chief of many mechanical devices enabling us to get
away from where we are to where we are no better off.  For this purpose
the railroad is held in highest favor by the optimist, for it permits
him to make the transit with great expedition.

\paragraph{RAMSHACKLE, adj.}  Pertaining to a certain order of architecture,
otherwise known as the Normal American.  Most of the public buildings
of the United States are of the Ramshackle order, though some of our
earlier architects preferred the Ironic.  Recent additions to the
White House in Washington are Theo-Doric, the ecclesiastic order of
the Dorians.  They are exceedingly fine and cost one hundred dollars a
brick.

\paragraph{REALISM, n.}  The art of depicting nature as it is seem by toads.  The
charm suffusing a landscape painted by a mole, or a story written by a
measuring-worm.

\paragraph{REALITY, n.}  The dream of a mad philosopher.  That which would remain
in the cupel if one should assay a phantom.  The nucleus of a vacuum.

\paragraph{REALLY, adv.}  Apparently.

\paragraph{REAR, n.}  In American military matters, that exposed part of the army
that is nearest to Congress.

\paragraph{REASON, v.i.}  To weight probabilities in the scales of desire.

\paragraph{REASON, n.}  Propensitate of prejudice.

\paragraph{REASONABLE, adj.}  Accessible to the infection of our own opinions.
Hospitable to persuasion, dissuasion and evasion.

\paragraph{REBEL, n.}  A proponent of a new misrule who has failed to establish
it.

\paragraph{RECOLLECT, v.}  To recall with additions something not previously
known.

\paragraph{RECONCILIATION, n.}  A suspension of hostilities.  An armed truce for
the purpose of digging up the dead.

\paragraph{RECONSIDER, v.}  To seek a justification for a decision already made.

\paragraph{RECOUNT, n.}  In American politics, another throw of the dice, accorded
to the player against whom they are loaded.

\paragraph{RECREATION, n.}  A particular kind of dejection to relieve a general
fatigue.

\paragraph{RECRUIT, n.}  A person distinguishable from a civilian by his uniform
and from a soldier by his gait.

\begin{quote}   Fresh from the farm or factory or street, \\
  His marching, in pursuit or in retreat, \\
      Were an impressive martial spectacle \\
  Except for two impediments -- his feet. \\
 \\
Thompson Johnson \end{quote}


\paragraph{RECTOR, n.}  In the Church of England, the Third Person of the
parochial Trinity, the Cruate and the Vicar being the other two.

\paragraph{REDEMPTION, n.}  Deliverance of sinners from the penalty of their sin,
through their murder of the deity against whom they sinned.  The
doctrine of Redemption is the fundamental mystery of our holy
religion, and whoso believeth in it shall not perish, but have
everlasting life in which to try to understand it.

\begin{quote}   We must awake Man's spirit from his sin, \\
      And take some special measure for redeeming it; \\
  Though hard indeed the task to get it in \\
      Among the angels any way but teaming it, \\
      Or purify it otherwise than steaming it. \\
  I'm awkward at Redemption -- a beginner: \\
  My method is to crucify the sinner. \\
 \\
Golgo Brone \end{quote}


\paragraph{REDRESS, n.}  Reparation without satisfaction.
\subparagraph{}   Among the Anglo-Saxon a subject conceiving himself wronged by the
king was permitted, on proving his injury, to beat a brazen image of
the royal offender with a switch that was afterward applied to his own
naked back.  The latter rite was performed by the public hangman, and
it assured moderation in the plaintiff's choice of a switch.

\paragraph{RED-SKIN, n.}  A North American Indian, whose skin is not red -- at
least not on the outside.

\paragraph{REDUNDANT, adj.}  Superfluous; needless; {\em de trop}.

\begin{quote}   The Sultan said:  "There's evidence abundant \\
  To prove this unbelieving dog redundant." \\
  To whom the Grand Vizier, with mien impressive, \\
  Replied:  "His head, at least, appears excessive." \\
 \\
Habeeb Suleiman \end{quote}


\begin{quote}   Mr. Debs is a redundant citizen. \\
 \\
Theodore Roosevelt \end{quote}


\paragraph{REFERENDUM, n.}  A law for submission of proposed legislation to a
popular vote to learn the nonsensus of public opinion.

\paragraph{REFLECTION, n.}  An action of the mind whereby we obtain a clearer view
of our relation to the things of yesterday and are able to avoid the
perils that we shall not again encounter.

\paragraph{REFORM, v.}  A thing that mostly satisfies reformers opposed to
reformation.

\paragraph{REFUGE, n.}  Anything assuring protection to one in peril.  Moses and
Joshua provided six cities of refuge -- Bezer, Golan, Ramoth, Kadesh,
Schekem and Hebron -- to which one who had taken life inadvertently
could flee when hunted by relatives of the deceased.  This admirable
expedient supplied him with wholesome exercise and enabled them to
enjoy the pleasures of the chase; whereby the soul of the dead man was
appropriately honored by observations akin to the funeral games of
early Greece.

\paragraph{REFUSAL, n.}  Denial of something desired; as an elderly maiden's hand
in marriage, to a rich and handsome suitor; a valuable franchise to a
rich corporation, by an alderman; absolution to an impenitent king, by
a priest, and so forth.  Refusals are graded in a descending scale of
finality thus:  the refusal absolute, the refusal condition, the
refusal tentative and the refusal feminine.  The last is called by
some casuists the refusal assentive.

\paragraph{REGALIA, n.}  Distinguishing insignia, jewels and costume of such
ancient and honorable orders as Knights of Adam; Visionaries of
Detectable Bosh; the Ancient Order of Modern Troglodytes; the League
of Holy Humbug; the Golden Phalanx of Phalangers; the Genteel Society
of Expurgated Hoodlums; the Mystic Alliances of Georgeous Regalians;
Knights and Ladies of the Yellow Dog; the Oriental Order of Sons of
the West; the Blatherhood of Insufferable Stuff; Warriors of the Long
Bow; Guardians of the Great Horn Spoon; the Band of Brutes; the
Impenitent Order of Wife-Beaters; the Sublime Legion of Flamboyant
Conspicuants; Worshipers at the Electroplated Shrine; Shining
Inaccessibles; Fee-Faw-Fummers of the inimitable Grip; Jannissaries of
the Broad-Blown Peacock; Plumed Increscencies of the Magic Temple; the
Grand Cabal of Able-Bodied Sedentarians; Associated Deities of the
Butter Trade; the Garden of Galoots; the Affectionate Fraternity of
Men Similarly Warted; the Flashing Astonishers; Ladies of Horror;
Cooperative Association for Breaking into the Spotlight; Dukes of Eden;
Disciples Militant of the Hidden Faith; Knights-Champions of the
Domestic Dog; the Holy Gregarians; the Resolute Optimists; the Ancient
Sodality of Inhospitable Hogs; Associated Sovereigns of Mendacity;
Dukes-Guardian of the Mystic Cess-Pool; the Society for Prevention of
Prevalence; Kings of Drink; Polite Federation of Gents-Consequential;
the Mysterious Order of the Undecipherable Scroll; Uniformed Rank of
Lousy Cats; Monarchs of Worth and Hunger; Sons of the South Star;
Prelates of the Tub-and-Sword.

\paragraph{RELIGION, n.}  A daughter of Hope and Fear, explaining to Ignorance the
nature of the Unknowable.
\begin{quote}   "What is your religion my son?" inquired the Archbishop of Rheims. \\
  "Pardon, monseigneur," replied Rochebriant; "I am ashamed of it." \\
  "Then why do you not become an atheist?" \\
  "Impossible!  I should be ashamed of atheism." \\
  "In that case, monsieur, you should join the Protestants."  \end{quote}

\paragraph{RELIQUARY, n.}  A receptacle for such sacred objects as pieces of the
true cross, short-ribs of the saints, the ears of Balaam's ass, the
lung of the cock that called Peter to repentance and so forth.
Reliquaries are commonly of metal, and provided with a lock to prevent
the contents from coming out and performing miracles at unseasonable
times.  A feather from the wing of the Angel of the Annunciation once
escaped during a sermon in Saint Peter's and so tickled the noses of
the congregation that they woke and sneezed with great vehemence three
times each.  It is related in the "Gesta Sanctorum" that a sacristan
in the Canterbury cathedral surprised the head of Saint Dennis in the
library.  Reprimanded by its stern custodian, it explained that it was
seeking a body of doctrine.  This unseemly levity so raged the
diocesan that the offender was publicly anathematized, thrown into the
Stour and replaced by another head of Saint Dennis, brought from Rome.

\paragraph{RENOWN, n.}  A degree of distinction between notoriety and fame -- a
little more supportable than the one and a little more intolerable
than the other.  Sometimes it is conferred by an unfriendly and
inconsiderate hand.

\begin{quote}   I touched the harp in every key, \\
      But found no heeding ear; \\
  And then Ithuriel touched me \\
      With a revealing spear. \\
 \\
  Not all my genius, great as 'tis, \\
      Could urge me out of night. \\
  I felt the faint appulse of his, \\
      And leapt into the light! \\
 \\
W.J. Candleton \end{quote}


\paragraph{REPARATION, n.}  Satisfaction that is made for a wrong and deducted
from the satisfaction felt in committing it.

\paragraph{REPARTEE, n.}  Prudent insult in retort.  Practiced by gentlemen with a
constitutional aversion to violence, but a strong disposition to
offend.  In a war of words, the tactics of the North American Indian.

\paragraph{REPENTANCE, n.}  The faithful attendant and follower of Punishment.  It
is usually manifest in a degree of reformation that is not
inconsistent with continuity of sin.

\begin{quote}   Desirous to avoid the pains of Hell, \\
  You will repent and join the Church, Parnell? \\
  How needless! -- Nick will keep you off the coals \\
  And add you to the woes of other souls. \\
 \\
Jomater Abemy \end{quote}


\paragraph{REPLICA, n.}  A reproduction of a work of art, by the artist that made
the original.  It is so called to distinguish it from a "copy," which
is made by another artist.  When the two are mae with equal skill the
replica is the more valuable, for it is supposed to be more beautiful
than it looks.

\paragraph{REPORTER, n.}  A writer who guesses his way to the truth and dispels it
with a tempest of words.

\begin{quote}   "More dear than all my bosom knows, O thou \\
  Whose 'lips are sealed' and will not disavow!" \\
  So sang the blithe reporter-man as grew \\
  Beneath his hand the leg-long "interview." \\
 \\
Barson Maith \end{quote}


\paragraph{REPOSE, v.i.}  To cease from troubling.

\paragraph{REPRESENTATIVE, n.}  In national politics, a member of the Lower House
in this world, and without discernible hope of promotion in the next.

\paragraph{REPROBATION, n.}  In theology, the state of a luckless mortal
prenatally damned.  The doctrine of reprobation was taught by Calvin,
whose joy in it was somewhat marred by the sad sincerity of his
conviction that although some are foredoomed to perdition, others are
predestined to salvation.

\paragraph{REPUBLIC, n.}  A nation in which, the thing governing and the thing
governed being the same, there is only a permitted authority to
enforce an optional obedience.  In a republic, the foundation of
public order is the ever lessening habit of submission inherited from
ancestors who, being truly governed, submitted because they had to.
There are as many kinds of republics as there are graduations between
the despotism whence they came and the anarchy whither they lead.

\paragraph{REQUIEM, n.}  A mass for the dead which the minor poets assure us the
winds sing o'er the graves of their favorites.  Sometimes, by way of
providing a varied entertainment, they sing a dirge.

\paragraph{RESIDENT, adj.}  Unable to leave.

\paragraph{RESIGN, v.t.}  To renounce an honor for an advantage.  To renounce an
advantage for a greater advantage.

\begin{quote}   'Twas rumored Leonard Wood had signed \\
      A true renunciation \\
  Of title, rank and every kind \\
      Of military station -- \\
      Each honorable station. \\
 \\
  By his example fired -- inclined \\
      To noble emulation, \\
  The country humbly was resigned \\
      To Leonard's resignation -- \\
      His Christian resignation. \\
 \\
Politian Greame \end{quote}


\paragraph{RESOLUTE, adj.}  Obstinate in a course that we approve.

\paragraph{RESPECTABILITY, n.}  The offspring of a {\em liaison} between a bald head
and a bank account.

\paragraph{RESPIRATOR, n.}  An apparatus fitted over the nose and mouth of an
inhabitant of London, whereby to filter the visible universe in its
passage to the lungs.

\paragraph{RESPITE, n.}  A suspension of hostilities against a sentenced assassin,
to enable the Executive to determine whether the murder may not have
been done by the prosecuting attorney.  Any break in the continuity of
a disagreeable expectation.

\begin{quote}   Altgeld upon his incandescent bed \\
  Lay, an attendant demon at his head. \\
 \\
  "O cruel cook, pray grant me some relief -- \\
  Some respite from the roast, however brief." \\
 \\
  "Remember how on earth I pardoned all \\
  Your friends in Illinois when held in thrall." \\
 \\
  "Unhappy soul! for that alone you squirm \\
  O'er fire unquenched, a never-dying worm. \\
 \\
  "Yet, for I pity your uneasy state, \\
  Your doom I'll mollify and pains abate. \\
 \\
  "Naught, for a season, shall your comfort mar, \\
  Not even the memory of who you are." \\
 \\
  Throughout eternal space dread silence fell; \\
  Heaven trembled as Compassion entered Hell. \\
 \\
  "As long, sweet demon, let my respite be \\
  As, governing down here, I'd respite thee." \\
 \\
  "As long, poor soul, as any of the pack \\
  You thrust from jail consumed in getting back." \\
 \\
  A genial chill affected Altgeld's hide \\
  While they were turning him on t'other side. \\
 \\
Joel Spate Woop \end{quote}


\paragraph{RESPLENDENT, adj.}  Like a simple American citizen beduking himself in
his lodge, or affirming his consequence in the Scheme of Things as an
elemental unit of a parade.

\begin{quote}       The Knights of Dominion were so resplendent in their velvet- \\
  and-gold that their masters would hardly have known them. \\
 \\
"Chronicles of the Classes" \end{quote}


\paragraph{RESPOND, v.i.}  To make answer, or disclose otherwise a consciousness
of having inspired an interest in what Herbert Spencer calls "external
coexistences," as Satan "squat like a toad" at the ear of Eve,
responded to the touch of the angel's spear.  To respond in damages is
to contribute to the maintenance of the plaintiff's attorney and,
incidentally, to the gratification of the plaintiff.

\paragraph{RESPONSIBILITY, n.}  A detachable burden easily shifted to the
shoulders of God, Fate, Fortune, Luck or one's neighbor.  In the days
of astrology it was customary to unload it upon a star.

\begin{quote}   Alas, things ain't what we should see \\
  If Eve had let that apple be; \\
  And many a feller which had ought \\
  To set with monarchses of thought, \\
  Or play some rosy little game \\
  With battle-chaps on fields of fame, \\
  Is downed by his unlucky star \\
  And hollers:  "Peanuts! -- here you are!" \\
 \\
"The Sturdy Beggar" \end{quote}


\paragraph{RESTITUTIONS, n.}  The founding or endowing of universities and public
libraries by gift or bequest.

\paragraph{RESTITUTOR, n.}  Benefactor; philanthropist.

\paragraph{RETALIATION, n.}  The natural rock upon which is reared the Temple of
Law.

\paragraph{RETRIBUTION, n.}  A rain of fire-and-brimstone that falls alike upon
the just and such of the unjust as have not procured shelter by
evicting them.
\subparagraph{}   In the lines following, addressed to an Emperor in exile by Father
Gassalasca Jape, the reverend poet appears to hint his sense of the
improduence of turning about to face Retribution when it is talking
exercise:

\begin{quote}   What, what! Dom Pedro, you desire to go \\
      Back to Brazil to end your days in quiet? \\
  Why, what assurance have you 'twould be so? \\
      'Tis not so long since you were in a riot, \\
      And your dear subjects showed a will to fly at \\
  Your throat and shake you like a rat.  You know \\
  That empires are ungrateful; are you certain \\
  Republics are less handy to get hurt in?  \end{quote}

\paragraph{REVEILLE, n.}  A signal to sleeping soldiers to dream of battlefields
no more, but get up and have their blue noses counted.  In the
American army it is ingeniously called "rev-e-lee," and to that
pronunciation our countrymen have pledged their lives, their
misfortunes and their sacred dishonor.

\paragraph{REVELATION, n.}  A famous book in which St. John the Divine concealed
all that he knew.  The revealing is done by the commentators, who know
nothing.

\paragraph{REVERENCE, n.}  The spiritual attitude of a man to a god and a dog to a
man.

\paragraph{REVIEW, v.t.}

\begin{quote}   To set your wisdom (holding not a doubt of it, \\
      Although in truth there's neither bone nor skin to it) \\
  At work upon a book, and so read out of it \\
      The qualities that you have first read into it.
\end{quote}
\paragraph{REVOLUTION, n.}  In politics, an abrupt change in the form of
misgovernment.  Specifically, in American history, the substitution of
the rule of an Administration for that of a Ministry, whereby the
welfare and happiness of the people were advanced a full half-inch.
Revolutions are usually accompanied by a considerable effusion of
blood, but are accounted worth it -- this appraisement being made by
beneficiaries whose blood had not the mischance to be shed.  The
French revolution is of incalculable value to the Socialist of to-day;
when he pulls the string actuating its bones its gestures are
inexpressibly terrifying to gory tyrants suspected of fomenting law
and order.

\paragraph{RHADOMANCER, n.}  One who uses a divining-rod in prospecting for
precious metals in the pocket of a fool.

\paragraph{RIBALDRY, n.}  Censorious language by another concerning oneself.

\paragraph{RIBROASTER, n.}  Censorious language by oneself concerning another.
The word is of classical refinement, and is even said to have been
used in a fable by Georgius Coadjutor, one of the most fastidious
writers of the fifteenth century -- commonly, indeed, regarded as the
founder of the Fastidiotic School.

\paragraph{RICE-WATER, n.}  A mystic beverage secretly used by our most popular
novelists and poets to regulate the imagination and narcotize the
conscience.  It is said to be rich in both obtundite and lethargine,
and is brewed in a midnight fog by a fat which of the Dismal Swamp.

\paragraph{RICH, adj.}  Holding in trust and subject to an accounting the property
of the indolent, the incompetent, the unthrifty, the envious and the
luckless.  That is the view that prevails in the underworld, where the
Brotherhood of Man finds its most logical development and candid
advocacy.  To denizens of the midworld the word means good and wise.

\paragraph{RICHES, n.}

\begin{quote}       A gift from Heaven signifying, "This is my beloved son, in \\
  whom I am well pleased." \\
 \\
John D. Rockefeller \end{quote}


\begin{quote}       The reward of toil and virtue. \\
 \\
J.P. Morgan \end{quote}


\begin{quote}       The sayings of many in the hands of one. \\
 \\
Eugene Debs \end{quote}


\subparagraph{}   To these excellent definitions the inspired lexicographer feels
that he can add nothing of value.

\paragraph{RIDICULE, n.}  Words designed to show that the person of whom they are
uttered is devoid of the dignity of character distinguishing him who
utters them.  It may be graphic, mimetic or merely rident.
Shaftesbury is quoted as having pronounced it the test of truth -- a
ridiculous assertion, for many a solemn fallacy has undergone
centuries of ridicule with no abatement of its popular acceptance.
What, for example, has been more valorously derided than the doctrine
of Infant Respectability?

\paragraph{RIGHT, n.}  Legitimate authority to be, to do or to have; as the right
to be a king, the right to do one's neighbor, the right to have
measles, and the like.  The first of these rights was once universally
believed to be derived directly from the will of God; and this is
still sometimes affirmed {\em in partibus infidelium} outside the
enlightened realms of Democracy; as the well known lines of Sir
Abednego Bink, following:

\begin{quote}       By what right, then, do royal rulers rule? \\
          Whose is the sanction of their state and pow'r? \\
      He surely were as stubborn as a mule \\
          Who, God unwilling, could maintain an hour \\
  His uninvited session on the throne, or air \\
  His pride securely in the Presidential chair. \\
 \\
      Whatever is is so by Right Divine; \\
          Whate'er occurs, God wills it so.  Good land! \\
      It were a wondrous thing if His design \\
          A fool could baffle or a rogue withstand! \\
  If so, then God, I say (intending no offence) \\
  Is guilty of contributory negligence.  \end{quote}

\paragraph{RIGHTEOUSNESS, n.}  A sturdy virtue that was once found among the
Pantidoodles inhabiting the lower part of the peninsula of Oque.  Some
feeble attempts were made by returned missionaries to introduce it
into several European countries, but it appears to have been
imperfectly expounded.  An example of this faulty exposition is found
in the only extant sermon of the pious Bishop Rowley, a characteristic
passage from which is here given:

\subparagraph{}       "Now righteousness consisteth not merely in a holy state of
  mind, nor yet in performance of religious rites and obedience to
  the letter of the law.  It is not enough that one be pious and
  just:  one must see to it that others also are in the same state;
  and to this end compulsion is a proper means.  Forasmuch as my
  injustice may work ill to another, so by his injustice may evil be
  wrought upon still another, the which it is as manifestly my duty
  to estop as to forestall mine own tort.  Wherefore if I would be
  righteous I am bound to restrain my neighbor, by force if needful,
  in all those injurious enterprises from which, through a better
  disposition and by the help of Heaven, I do myself restrain."

\paragraph{RIME, n.}  Agreeing sounds in the terminals of verse, mostly bad.  The
verses themselves, as distinguished from prose, mostly dull.  Usually
(and wickedly) spelled "rhyme."

\paragraph{RIMER, n.}  A poet regarded with indifference or disesteem.

\begin{quote}   The rimer quenches his unheeded fires, \\
  The sound surceases and the sense expires. \\
  Then the domestic dog, to east and west, \\
  Expounds the passions burning in his breast. \\
  The rising moon o'er that enchanted land \\
  Pauses to hear and yearns to understand. \\
 \\
Mowbray Myles \end{quote}


\paragraph{RIOT, n.}  A popular entertainment given to the military by innocent
bystanders.

\paragraph{R.I.P.}  A careless abbreviation of {\em requiescat in pace}, attesting to
indolent goodwill to the dead.  According to the learned Dr. Drigge,
however, the letters originally meant nothing more than {\em reductus in
pulvis}.

\paragraph{RITE, n.}  A religious or semi-religious ceremony fixed by law, precept
or custom, with the essential oil of sincerity carefully squeezed out
of it.

\paragraph{RITUALISM, n.}  A Dutch Garden of God where He may walk in rectilinear
freedom, keeping off the grass.

\paragraph{ROAD, n.}  A strip of land along which one may pass from where it is
too tiresome to be to where it is futile to go.

\begin{quote}   All roads, howsoe'er they diverge, lead to Rome, \\
  Whence, thank the good Lord, at least one leads back home. \\
 \\
Borey the Bald \end{quote}


\paragraph{ROBBER, n.}  A candid man of affairs.
\subparagraph{}   It is related of Voltaire that one night he and some traveling
companion lodged at a wayside inn.  The surroundings were suggestive,
and after supper they agreed to tell robber stories in turn.  "Once
there was a Farmer-General of the Revenues."  Saying nothing more, he
was encouraged to continue.  "That," he said, "is the story."

\paragraph{ROMANCE, n.}  Fiction that owes no allegiance to the God of Things as
They Are.  In the novel the writer's thought is tethered to
probability, as a domestic horse to the hitching-post, but in romance
it ranges at will over the entire region of the imagination -- free,
lawless, immune to bit and rein.  Your novelist is a poor creature, as
Carlyle might say -- a mere reporter.  He may invent his characters
and plot, but he must not imagine anything taking place that might not
occur, albeit his entire narrative is candidly a lie.  Why he imposes
this hard condition on himself, and "drags at each remove a
lengthening chain" of his own forging he can explain in ten thick
volumes without illuminating by so much as a candle's ray the black
profound of his own ignorance of the matter.  There are great novels,
for great writers have "laid waste their powers" to write them, but it
remains true that far and away the most fascinating fiction that we
have is "The Thousand and One Nights."

\paragraph{ROPE, n.}  An obsolescent appliance for reminding assassins that they
too are mortal.  It is put about the neck and remains in place one's
whole life long.  It has been largely superseded by a more complex
electrical device worn upon another part of the person; and this is
rapidly giving place to an apparatus known as the preachment.

\paragraph{ROSTRUM, n.}  In Latin, the beak of a bird or the prow of a ship.  In
America, a place from which a candidate for office energetically
expounds the wisdom, virtue and power of the rabble.

\paragraph{ROUNDHEAD, n.}  A member of the Parliamentarian party in the English
civil war -- so called from his habit of wearing his hair short,
whereas his enemy, the Cavalier, wore his long.  There were other
points of difference between them, but the fashion in hair was the
fundamental cause of quarrel.  The Cavaliers were royalists because
the king, an indolent fellow, found it more convenient to let his hair
grow than to wash his neck.  This the Roundheads, who were mostly
barbers and soap-boilers, deemed an injury to trade, and the royal
neck was therefore the object of their particular indignation.
Descendants of the belligerents now wear their hair all alike, but the
fires of animosity enkindled in that ancient strife smoulder to this
day beneath the snows of British civility.

\paragraph{RUBBISH, n.}  Worthless matter, such as the religions, philosophies,
literatures, arts and sciences of the tribes infesting the regions
lying due south from Boreaplas.

\paragraph{RUIN, v.}  To destroy.  Specifically, to destroy a maid's belief in the
virtue of maids.

\paragraph{RUM, n.}  Generically, fiery liquors that produce madness in total
abstainers.

\paragraph{RUMOR, n.}  A favorite weapon of the assassins of character.

\begin{quote}   Sharp, irresistible by mail or shield, \\
      By guard unparried as by flight unstayed, \\
  O serviceable Rumor, let me wield \\
      Against my enemy no other blade. \\
  His be the terror of a foe unseen, \\
      His the inutile hand upon the hilt, \\
  And mine the deadly tongue, long, slender, keen, \\
      Hinting a rumor of some ancient guilt. \\
  So shall I slay the wretch without a blow, \\
  Spare me to celebrate his overthrow, \\
  And nurse my valor for another foe. \\
 \\
Joel Buxter \end{quote}


\paragraph{RUSSIAN, n.}  A person with a Caucasian body and a Mongolian soul.  A
Tartar Emetic.



\section*{S}



\paragraph{SABBATH, n.}  A weekly festival having its origin in the fact that God
made the world in six days and was arrested on the seventh.  Among the
Jews observance of the day was enforced by a Commandment of which this
is the Christian version:  "Remember the seventh day to make thy
neighbor keep it wholly."  To the Creator it seemed fit and expedient
that the Sabbath should be the last day of the week, but the Early
Fathers of the Church held other views.  So great is the sanctity of
the day that even where the Lord holds a doubtful and precarious
jurisdiction over those who go down to (and down into) the sea it is
reverently recognized, as is manifest in the following deep-water
version of the Fourth Commandment:

\begin{quote}   Six days shalt thou labor and do all thou art able, \\
  And on the seventh holystone the deck and scrape the cable. \end{quote}

\subparagraph{} Decks are no longer holystoned, but the cable still supplies the
captain with opportunity to attest a pious respect for the divine
ordinance.

\paragraph{SACERDOTALIST, n.}  One who holds the belief that a clergyman is a
priest.  Denial of this momentous doctrine is the hardest challenge
that is now flung into the teeth of the Episcopalian church by the
Neo-Dictionarians.

\paragraph{SACRAMENT, n.}  A solemn religious ceremony to which several degrees of
authority and significance are attached.  Rome has seven sacraments,
but the Protestant churches, being less prosperous, feel that they can
afford only two, and these of inferior sanctity.  Some of the smaller
sects have no sacraments at all -- for which mean economy they will
indubitable be damned.

\paragraph{SACRED, adj.}  Dedicated to some religious purpose; having a divine
character; inspiring solemn thoughts or emotions; as, the Dalai Lama
of Thibet; the Moogum of M'bwango; the temple of Apes in Ceylon; the
Cow in India; the Crocodile, the Cat and the Onion of ancient Egypt;
the Mufti of Moosh; the hair of the dog that bit Noah, etc.

\begin{quote}   All things are either sacred or profane. \\
  The former to ecclesiasts bring gain; \\
  The latter to the devil appertain. \\
 \\
Dumbo Omohundro \end{quote}


\paragraph{SANDLOTTER, n.}  A vertebrate mammal holding the political views of
Denis Kearney, a notorious demagogue of San Francisco, whose audiences
gathered in the open spaces (sandlots) of the town.  True to the
traditions of his species, this leader of the proletariat was finally
bought off by his law-and-order enemies, living prosperously silent
and dying impenitently rich.  But before his treason he imposed upon
California a constitution that was a confection of sin in a diction of
solecisms.  The similarity between the words "sandlotter" and
"sansculotte" is problematically significant, but indubitably
suggestive.

\paragraph{SAFETY-CLUTCH, n.}  A mechanical device acting automatically to prevent
the fall of an elevator, or cage, in case of an accident to the
hoisting apparatus.

\begin{quote}   Once I seen a human ruin \\
      In an elevator-well, \\
  And his members was bestrewin' \\
      All the place where he had fell. \\
 \\
  And I says, apostrophisin' \\
      That uncommon woful wreck: \\
  "Your position's so surprisin' \\
      That I tremble for your neck!" \\
 \\
  Then that ruin, smilin' sadly \\
      And impressive, up and spoke: \\
  "Well, I wouldn't tremble badly, \\
      For it's been a fortnight broke." \\
 \\
  Then, for further comprehension \\
      Of his attitude, he begs \\
  I will focus my attention \\
      On his various arms and legs -- \\
 \\
  How they all are contumacious; \\
      Where they each, respective, lie; \\
  How one trotter proves ungracious, \\
      T'other one an {\em alibi}. \\
 \\
  These particulars is mentioned \\
      For to show his dismal state, \\
  Which I wasn't first intentioned \\
      To specifical relate. \\
 \\
  None is worser to be dreaded \\
      That I ever have heard tell \\
  Than the gent's who there was spreaded \\
      In that elevator-well. \\
 \\
  Now this tale is allegoric -- \\
      It is figurative all, \\
  For the well is metaphoric \\
      And the feller didn't fall. \\
 \\
  I opine it isn't moral \\
      For a writer-man to cheat, \\
  And despise to wear a laurel \\
      As was gotten by deceit. \\
 \\
  For 'tis Politics intended \\
      By the elevator, mind, \\
  It will boost a person splendid \\
      If his talent is the kind. \\
 \\
  Col. Bryan had the talent \\
      (For the busted man is him) \\
  And it shot him up right gallant \\
      Till his head begun to swim. \\
 \\
  Then the rope it broke above him \\
      And he painful come to earth \\
  Where there's nobody to love him \\
      For his detrimented worth. \\
 \\
  Though he's livin' none would know him, \\
      Or at leastwise not as such. \\
  Moral of this woful poem: \\
      Frequent oil your safety-clutch. \\
 \\
Porfer Poog \end{quote}


\paragraph{SAINT, n.}  A dead sinner revised and edited.
\subparagraph{}   The Duchess of Orleans relates that the irreverent old
calumniator, Marshal Villeroi, who in his youth had known St. Francis
de Sales, said, on hearing him called saint:  "I am delighted to hear
that Monsieur de Sales is a saint.  He was fond of saying indelicate
things, and used to cheat at cards.  In other respects he was a
perfect gentleman, though a fool."

\paragraph{SALACITY, n.}  A certain literary quality frequently observed in
popular novels, especially in those written by women and young girls,
who give it another name and think that in introducing it they are
occupying a neglected field of letters and reaping an overlooked
harvest.  If they have the misfortune to live long enough they are
tormented with a desire to burn their sheaves.

\paragraph{SALAMANDER, n.}  Originally a reptile inhabiting fire; later, an
anthropomorphous immortal, but still a pyrophile.  Salamanders are now
believed to be extinct, the last one of which we have an account
having been seen in Carcassonne by the Abbe Belloc, who exorcised it
with a bucket of holy water.

\paragraph{SARCOPHAGUS, n.}  Among the Greeks a coffin which being made of a
certain kind of carnivorous stone, had the peculiar property of
devouring the body placed in it.  The sarcophagus known to modern
obsequiographers is commonly a product of the carpenter's art.

\paragraph{SATAN, n.}  One of the Creator's lamentable mistakes, repented in
sashcloth and axes.  Being instated as an archangel, Satan made
himself multifariously objectionable and was finally expelled from
Heaven.  Halfway in his descent he paused, bent his head in thought a
moment and at last went back.  "There is one favor that I should like
to ask," said he.
\begin{quote}   "Name it." \\
  "Man, I understand, is about to be created.  He will need laws." \\
  "What, wretch! you his appointed adversary, charged from the dawn
of eternity with hatred of his soul -- you ask for the right to make
his laws?"\\
\\
"Pardon; what I have to ask is that he be permitted to make them
himself."
\end{quote}
\subparagraph{}   It was so ordered.

\paragraph{SATIETY, n.}  The feeling that one has for the plate after he has eaten
its contents, madam.

\paragraph{SATIRE, n.}  An obsolete kind of literary composition in which the
vices and follies of the author's enemies were expounded with
imperfect tenderness.  In this country satire never had more than a
sickly and uncertain existence, for the soul of it is wit, wherein we
are dolefully deficient, the humor that we mistake for it, like all
humor, being tolerant and sympathetic.  Moreover, although Americans
are "endowed by their Creator" with abundant vice and folly, it is not
generally known that these are reprehensible qualities, wherefore the
satirist is popularly regarded as a soul-spirited knave, and his ever
victim's outcry for codefendants evokes a national assent.

\begin{quote}   Hail Satire! be thy praises ever sung \\
  In the dead language of a mummy's tongue, \\
  For thou thyself art dead, and damned as well -- \\
  Thy spirit (usefully employed) in Hell. \\
  Had it been such as consecrates the Bible \\
  Thou hadst not perished by the law of libel. \\
 \\
Barney Stims \end{quote}


\paragraph{SATYR, n.}  One of the few characters of the Grecian mythology accorded
recognition in the Hebrew.  (Leviticus, xvii, 7.)  The satyr was at
first a member of the dissolute community acknowledging a loose
allegiance with Dionysius, but underwent many transformations and
improvements.  Not infrequently he is confounded with the faun, a
later and decenter creation of the Romans, who was less like a man and
more like a goat.

\paragraph{SAUCE, n.}  The one infallible sign of civilization and enlightenment.
A people with no sauces has one thousand vices; a people with one
sauce has only nine hundred and ninety-nine.  For every sauce invented
and accepted a vice is renounced and forgiven.

\paragraph{SAW, n.}  A trite popular saying, or proverb.  (Figurative and
colloquial.)  So called because it makes its way into a wooden head.
Following are examples of old saws fitted with new teeth.

\begin{quote}       A penny saved is a penny to squander. \\
 \\
      A man is known by the company that he organizes. \\
      A bad workman quarrels with the man who calls him that. \\
 \\
      A bird in the hand is worth what it will bring. \\
 \\
      Better late than before anybody has invited you. \\
 \\
      Example is better than following it. \\
 \\
      Half a loaf is better than a whole one if there is much else. \\
 \\
      Think twice before you speak to a friend in need. \\
 \\
      What is worth doing is worth the trouble of asking somebody to do it. \\
 \\
      Least said is soonest disavowed. \\
 \\
      He laughs best who laughs least. \\
 \\
      Speak of the Devil and he will hear about it. \\
 \\
      Of two evils choose to be the least. \\
 \\
      Strike while your employer has a big contract. \\
 \\
      Where there's a will there's a won't.  \end{quote}

\paragraph{SCARABAEUS, n.}  The sacred beetle of the ancient Egyptians, allied to
our familiar "tumble-bug."  It was supposed to symbolize immortality,
the fact that God knew why giving it its peculiar sanctity.  Its habit
of incubating its eggs in a ball of ordure may also have commended it
to the favor of the priesthood, and may some day assure it an equal
reverence among ourselves.  True, the American beetle is an inferior
beetle, but the American priest is an inferior priest.

\paragraph{SCARABEE, n.}  The same as scarabaeus.

\begin{quote}               He fell by his own hand \\
                  Beneath the great oak tree. \\
              He'd traveled in a foreign land. \\
              He tried to make her understand \\
              The dance that's called the Saraband, \\
                  But he called it Scarabee. \\
  He had called it so through an afternoon, \\
      And she, the light of his harem if so might be, \\
      Had smiled and said naught.  O the body was fair to see, \\
  All frosted there in the shine o' the moon -- \\
                      Dead for a Scarabee \\
  And a recollection that came too late. \\
                          O Fate! \\
                  They buried him where he lay, \\
                  He sleeps awaiting the Day, \\
                          In state, \\
  And two Possible Puns, moon-eyed and wan, \\
  Gloom over the grave and then move on. \\
                      Dead for a Scarabee! \\
                                                     Fernando Tapple  \end{quote}

\paragraph{SCARIFICATION, n.}  A form of penance practised by the mediaeval pious.
The rite was performed, sometimes with a knife, sometimes with a hot
iron, but always, says Arsenius Asceticus, acceptably if the penitent
spared himself no pain nor harmless disfigurement.  Scarification,
with other crude penances, has now been superseded by benefaction.
The founding of a library or endowment of a university is said to
yield to the penitent a sharper and more lasting pain than is
conferred by the knife or iron, and is therefore a surer means of
grace.  There are, however, two grave objections to it as a
penitential method:  the good that it does and the taint of justice.

\paragraph{SCEPTER, n.}  A king's staff of office, the sign and symbol of his
authority.  It was originally a mace with which the sovereign
admonished his jester and vetoed ministerial measures by breaking the
bones of their proponents.

\paragraph{SCIMETAR, n.}  A curved sword of exceeding keenness, in the conduct of
which certain Orientals attain a surprising proficiency, as the
incident here related will serve to show.  The account is translated
from the Japanese by Shusi Itama, a famous writer of the thirteenth
century.

\subparagraph{} When the great Gichi-Kuktai was Mikado he condemned to
  decapitation Jijiji Ri, a high officer of the Court.  Soon after
  the hour appointed for performance of the rite what was his
  Majesty's surprise to see calmly approaching the throne the man
  who should have been at that time ten minutes dead!
\subparagraph{}    "Seventeen hundred impossible dragons!" shouted the enraged
  monarch.  "Did I not sentence you to stand in the market-place and
  have your head struck off by the public executioner at three
  o'clock?  And is it not now 3:10?"
\subparagraph{}    "Son of a thousand illustrious deities," answered the
  condemned minister, "all that you say is so true that the truth is
  a lie in comparison.  But your heavenly Majesty's sunny and
  vitalizing wishes have been pestilently disregarded.  With joy I
  ran and placed my unworthy body in the market-place.  The
  executioner appeared with his bare scimetar, ostentatiously
  whirled it in air, and then, tapping me lightly upon the neck,
  strode away, pelted by the populace, with whom I was ever a
  favorite.  I am come to pray for justice upon his own dishonorable
  and treasonous head."
\subparagraph{}      "To what regiment of executioners does the black-boweled
  caitiff belong?" asked the Mikado.
\subparagraph{}      "To the gallant Ninety-eight Hundred and Thirty-seventh -- I
  know the man.  His name is Sakko-Samshi."
      "Let him be brought before me," said the Mikado to an
  attendant, and a half-hour later the culprit stood in the
  Presence.
\subparagraph{}      "Thou bastard son of a three-legged hunchback without thumbs!"
  roared the sovereign~-- "why didst thou but lightly tap the neck
  that it should have been thy pleasure to sever?"
\subparagraph{}      "Lord of Cranes of Cherry Blooms," replied the executioner,
  unmoved, "command him to blow his nose with his fingers."
      Being commanded, Jijiji Ri laid hold of his nose and trumpeted
  like an elephant, all expecting to see the severed head flung
  violently from him.  Nothing occurred:  the performance prospered
  peacefully to the close, without incident.
\subparagraph{}      All eyes were now turned on the executioner, who had grown as
  white as the snows on the summit of Fujiama.  His legs trembled
  and his breath came in gasps of terror.
\subparagraph{}      "Several kinds of spike-tailed brass lions!" he cried; "I am a
  ruined and disgraced swordsman!  I struck the villain feebly
  because in flourishing the scimetar I had accidentally passed it
  through my own neck!  Father of the Moon, I resign my office."
\subparagraph{}      So saying, he gasped his top-knot, lifted off his head, and
  advancing to the throne laid it humbly at the Mikado's feet.

\paragraph{SCRAP-BOOK, n.}  A book that is commonly edited by a fool.  Many
persons of some small distinction compile scrap-books containing
whatever they happen to read about themselves or employ others to
collect.  One of these egotists was addressed in the lines following,
by Agamemnon Melancthon Peters:

\begin{quote}   Dear Frank, that scrap-book where you boast \\
      You keep a record true \\
  Of every kind of peppered roast \\
          That's made of you; \\
 \\
  Wherein you paste the printed gibes \\
      That revel round your name, \\
  Thinking the laughter of the scribes \\
          Attests your fame; \\
 \\
  Where all the pictures you arrange \\
      That comic pencils trace -- \\
  Your funny figure and your strange \\
          Semitic face -- \\
 \\
  Pray lend it me.  Wit I have not, \\
      Nor art, but there I'll list \\
  The daily drubbings you'd have got \\
          Had God a fist.  \end{quote}

\paragraph{SCRIBBLER, n.}  A professional writer whose views are antagonistic to
one's own.

\paragraph{SCRIPTURES, n.}  The sacred books of our holy religion, as
distinguished from the false and profane writings on which all other
faiths are based.

\paragraph{SEAL, n.}  A mark impressed upon certain kinds of documents to attest
their authenticity and authority.  Sometimes it is stamped upon wax,
and attached to the paper, sometimes into the paper itself.  Sealing,
in this sense, is a survival of an ancient custom of inscribing
important papers with cabalistic words or signs to give them a magical
efficacy independent of the authority that they represent.  In the
British museum are preserved many ancient papers, mostly of a
sacerdotal character, validated by necromantic pentagrams and other
devices, frequently initial letters of words to conjure with; and in
many instances these are attached in the same way that seals are
appended now.  As nearly every reasonless and apparently meaningless
custom, rite or observance of modern times had origin in some remote
utility, it is pleasing to note an example of ancient nonsense
evolving in the process of ages into something really useful.  Our
word "sincere" is derived from {\em sine cero}, without wax, but the
learned are not in agreement as to whether this refers to the absence
of the cabalistic signs, or to that of the wax with which letters were
formerly closed from public scrutiny.  Either view of the matter will
serve one in immediate need of an hypothesis.  The initials L.S.,
commonly appended to signatures of legal documents, mean {\em locum
sigillis}, the place of the seal, although the seal is no longer used
-- an admirable example of conservatism distinguishing Man from the
beasts that perish.  The words {\em locum sigillis} are humbly suggested
as a suitable motto for the Pribyloff Islands whenever they shall take
their place as a sovereign State of the American Union.

\paragraph{SEINE, n.}  A kind of net for effecting an involuntary change of
environment.  For fish it is made strong and coarse, but women are
more easily taken with a singularly delicate fabric weighted with
small, cut stones.

\begin{quote}   The devil casting a seine of lace, \\
      (With precious stones 'twas weighted) \\
  Drew it into the landing place \\
      And its contents calculated. \\
 \\
  All souls of women were in that sack -- \\
      A draft miraculous, precious! \\
  But ere he could throw it across his back \\
      They'd all escaped through the meshes. \\
 \\
Baruch de Loppis \end{quote}


\paragraph{SELF-ESTEEM, n.}  An erroneous appraisement.

\paragraph{SELF-EVIDENT, adj.}  Evident to one's self and to nobody else.

\paragraph{SELFISH, adj.}  Devoid of consideration for the selfishness of others.

\paragraph{SENATE, n.}  A body of elderly gentlemen charged with high duties and
misdemeanors.

\paragraph{SERIAL, n.}  A literary work, usually a story that is not true,
creeping through several issues of a newspaper or magazine.
Frequently appended to each installment is a "synposis of preceding
chapters" for those who have not read them, but a direr need is a
synposis of succeeding chapters for those who do not intend to read
{\em them}.  A synposis of the entire work would be still better.
\subparagraph{}   The late James F. Bowman was writing a serial tale for a weekly
paper in collaboration with a genius whose name has not come down to
us.  They wrote, not jointly but alternately, Bowman supplying the
installment for one week, his friend for the next, and so on, world
without end, they hoped.  Unfortunately they quarreled, and one Monday
morning when Bowman read the paper to prepare himself for his task, he
found his work cut out for him in a way to surprise and pain him.  His
collaborator had embarked every character of the narrative on a ship
and sunk them all in the deepest part of the Atlantic.

\paragraph{SEVERALTY, n.}  Separateness, as, lands in severalty, i.e., lands held
individually, not in joint ownership.  Certain tribes of Indians are
believed now to be sufficiently civilized to have in severalty the
lands that they have hitherto held as tribal organizations, and could
not sell to the Whites for waxen beads and potato whiskey.

\begin{quote}   Lo! the poor Indian whose unsuited mind \\
  Saw death before, hell and the grave behind; \\
  Whom thrifty settler ne'er besought to stay -- \\
  His small belongings their appointed prey; \\
  Whom Dispossession, with alluring wile, \\
  Persuaded elsewhere every little while! \\
  His fire unquenched and his undying worm \\
  By "land in severalty" (charming term!) \\
  Are cooled and killed, respectively, at last, \\
  And he to his new holding anchored fast!  \end{quote}

\paragraph{SHERIFF, n.}  In America the chief executive office of a country, whose
most characteristic duties, in some of the Western and Southern
States, are the catching and hanging of rogues.

\begin{quote}   John Elmer Pettibone Cajee \\
  (I write of him with little glee) \\
  Was just as bad as he could be. \\
 \\
  'Twas frequently remarked:  "I swon! \\
  The sun has never looked upon \\
  So bad a man as Neighbor John." \\
 \\
  A sinner through and through, he had \\
  This added fault:  it made him mad \\
  To know another man was bad. \\
 \\
  In such a case he thought it right \\
  To rise at any hour of night \\
  And quench that wicked person's light. \\
 \\
  Despite the town's entreaties, he \\
  Would hale him to the nearest tree \\
  And leave him swinging wide and free. \\
 \\
  Or sometimes, if the humor came, \\
  A luckless wight's reluctant frame \\
  Was given to the cheerful flame. \\
 \\
  While it was turning nice and brown, \\
  All unconcerned John met the frown \\
  Of that austere and righteous town. \\
 \\
  "How sad," his neighbors said, "that he \\
  So scornful of the law should be -- \\
  An anar c, h, i, s, t." \\
 \\
  (That is the way that they preferred \\
  To utter the abhorrent word, \\
  So strong the aversion that it stirred.) \\
 \\
  "Resolved," they said, continuing, \\
  "That Badman John must cease this thing \\
  Of having his unlawful fling. \\
 \\
  "Now, by these sacred relics" -- here \\
  Each man had out a souvenir \\
  Got at a lynching yesteryear -- \\
 \\
  "By these we swear he shall forsake \\
  His ways, nor cause our hearts to ache \\
  By sins of rope and torch and stake. \\
 \\
  "We'll tie his red right hand until \\
  He'll have small freedom to fulfil \\
  The mandates of his lawless will." \\
 \\
  So, in convention then and there, \\
  They named him Sheriff.  The affair \\
  Was opened, it is said, with prayer. \\
 \\
J. Milton Sloluck \end{quote}


\paragraph{SIREN, n.}  One of several musical prodigies famous for a vain attempt
to dissuade Odysseus from a life on the ocean wave.  Figuratively, any
lady of splendid promise, dissembled purpose and disappointing
performance.

\paragraph{SLANG, n.}  The grunt of the human hog ({\em Pignoramus intolerabilis})
with an audible memory.  The speech of one who utters with his tongue
what he thinks with his ear, and feels the pride of a creator in
accomplishing the feat of a parrot.  A means (under Providence) of
setting up as a wit without a capital of sense.

\paragraph{SMITHAREEN, n.}  A fragment, a decomponent part, a remain.  The word is
used variously, but in the following verse on a noted female reformer
who opposed bicycle-riding by women because it "led them to the devil"
it is seen at its best:

\begin{quote}   The wheels go round without a sound -- \\
      The maidens hold high revel; \\
  In sinful mood, insanely gay, \\
  True spinsters spin adown the way \\
      From duty to the devil! \\
  They laugh, they sing, and -- ting-a-ling! \\
      Their bells go all the morning; \\
  Their lanterns bright bestar the night \\
      Pedestrians a-warning. \\
  With lifted hands Miss Charlotte stands, \\
      Good-Lording and O-mying, \\
  Her rheumatism forgotten quite, \\
      Her fat with anger frying. \\
  She blocks the path that leads to wrath, \\
      Jack Satan's power defying. \\
  The wheels go round without a sound \\
      The lights burn red and blue and green. \\
  What's this that's found upon the ground? \\
      Poor Charlotte Smith's a smithareen! \\
 \\
John William Yope \end{quote}


\paragraph{SOPHISTRY, n.}  The controversial method of an opponent, distinguished
from one's own by superior insincerity and fooling.  This method is
that of the later Sophists, a Grecian sect of philosophers who began
by teaching wisdom, prudence, science, art and, in brief, whatever men
ought to know, but lost themselves in a maze of quibbles and a fog of
words.

\begin{quote}   His bad opponent's "facts" he sweeps away, \\
  And drags his sophistry to light of day; \\
  Then swears they're pushed to madness who resort \\
  To falsehood of so desperate a sort. \\
  Not so; like sods upon a dead man's breast, \\
  He lies most lightly who the least is pressed. \\
 \\
Polydore Smith \end{quote}


\paragraph{SORCERY, n.}  The ancient prototype and forerunner of political
influence.  It was, however, deemed less respectable and sometimes was
punished by torture and death.  Augustine Nicholas relates that a poor
peasant who had been accused of sorcery was put to the torture to
compel a confession.  After enduring a few gentle agonies the
suffering simpleton admitted his guilt, but naively asked his
tormentors if it were not possible to be a sorcerer without knowing
it.

\paragraph{SOUL, n.}  A spiritual entity concerning which there hath been brave
disputation.  Plato held that those souls which in a previous state of
existence (antedating Athens) had obtained the clearest glimpses of
eternal truth entered into the bodies of persons who became
philosophers.  Plato himself was a philosopher.  The souls that had
least contemplated divine truth animated the bodies of usurpers and
despots.  Dionysius I, who had threatened to decapitate the broad-
browed philosopher, was a usurper and a despot.  Plato, doubtless, was
not the first to construct a system of philosophy that could be quoted
against his enemies; certainly he was not the last.
\subparagraph{}   "Concerning the nature of the soul," saith the renowned author of
{\em Diversiones Sanctorum}, "there hath been hardly more argument than
that of its place in the body.  Mine own belief is that the soul hath
her seat in the abdomen -- in which faith we may discern and interpret
a truth hitherto unintelligible, namely that the glutton is of all men
most devout.  He is said in the Scripture to 'make a god of his belly'
-- why, then, should he not be pious, having ever his Deity with him
to freshen his faith?  Who so well as he can know the might and
majesty that he shrines?  Truly and soberly, the soul and the stomach
are one Divine Entity; and such was the belief of Promasius, who
nevertheless erred in denying it immortality.  He had observed that
its visible and material substance failed and decayed with the rest of
the body after death, but of its immaterial essence he knew nothing.
This is what we call the Appetite, and it survives the wreck and reek
of mortality, to be rewarded or punished in another world, according
to what it hath demanded in the flesh.  The Appetite whose coarse
clamoring was for the unwholesome viands of the general market and the
public refectory shall be cast into eternal famine, whilst that which
firmly through civilly insisted on ortolans, caviare, terrapin,
anchovies, {\em pates de foie gras} and all such Christian comestibles
shall flesh its spiritual tooth in the souls of them forever and ever,
and wreak its divine thirst upon the immortal parts of the rarest and
richest wines ever quaffed here below.  Such is my religious faith,
though I grieve to confess that neither His Holiness the Pope nor His
Grace the Archbishop of Canterbury (whom I equally and profoundly
revere) will assent to its dissemination."

\paragraph{SPOOKER, n.}  A writer whose imagination concerns itself with
supernatural phenomena, especially in the doings of spooks.  One of
the most illustrious spookers of our time is Mr. William D. Howells,
who introduces a well-credentialed reader to as respectable and
mannerly a company of spooks as one could wish to meet.  To the terror
that invests the chairman of a district school board, the Howells
ghost adds something of the mystery enveloping a farmer from another
township.

\paragraph{STORY, n.}  A narrative, commonly untrue.  The truth of the stories
here following has, however, not been successfully impeached.

\paragraph{}   One evening Mr. Rudolph Block, of New York, found himself seated
at dinner alongside Mr. Percival Pollard, the distinguished critic.
\subparagraph{}   "Mr. Pollard," said he, "my book, {\em The Biography of a Dead Cow},
is published anonymously, but you can hardly be ignorant of its
authorship.  Yet in reviewing it you speak of it as the work of the
Idiot of the Century.  Do you think that fair criticism?"
\subparagraph{}   "I am very sorry, sir," replied the critic, amiably, "but it did
not occur to me that you really might not wish the public to know who
wrote it."
\\
\paragraph{}   Mr. W.C. Morrow, who used to live in San Jose, California, was
addicted to writing ghost stories which made the reader feel as if a
stream of lizards, fresh from the ice, were streaking it up his back
and hiding in his hair.  San Jose was at that time believed to be
haunted by the visible spirit of a noted bandit named Vasquez, who had
been hanged there.  The town was not very well lighted, and it is
putting it mildly to say that San Jose was reluctant to be out o'
nights.  One particularly dark night two gentlemen were abroad in the
loneliest spot within the city limits, talking loudly to keep up their
courage, when they came upon Mr. J.J. Owen, a well-known journalist.
\subparagraph{}   "Why, Owen," said one, "what brings you here on such a night as
this?  You told me that this is one of Vasquez' favorite haunts!  And
you are a believer.  Aren't you afraid to be out?"
\subparagraph{}   "My dear fellow," the journalist replied with a drear autumnal
cadence in his speech, like the moan of a leaf-laden wind, "I am
afraid to be in.  I have one of Will Morrow's stories in my pocket and
I don't dare to go where there is light enough to read it."
\\
\paragraph{}   Rear-Admiral Schley and Representative Charles F. Joy were
standing near the Peace Monument, in Washington, discussing the
question, Is success a failure?  Mr. Joy suddenly broke off in the
middle of an eloquent sentence, exclaiming:  "Hello!  I've heard that
band before.  Santlemann's, I think."
\subparagraph{}   "I don't hear any band," said Schley.
\subparagraph{}  "Come to think, I don't either," said Joy; "but I see General
Miles coming down the avenue, and that pageant always affects me in
the same way as a brass band.  One has to scrutinize one's impressions
pretty closely, or one will mistake their origin."
\subparagraph{}   While the Admiral was digesting this hasty meal of philosophy
General Miles passed in review, a spectacle of impressive dignity.
When the tail of the seeming procession had passed and the two
observers had recovered from the transient blindness caused by its
effulgence~--
\subparagraph{}   "He seems to be enjoying himself," said the Admiral.
\subparagraph{}  "There is nothing," assented Joy, thoughtfully, "that he enjoys
one-half so well."
\\
\paragraph{}   The illustrious statesman, Champ Clark, once lived about a mile
from the village of Jebigue, in Missouri.  One day he rode into town
on a favorite mule, and, hitching the beast on the sunny side of a
street, in front of a saloon, he went inside in his character of
teetotaler, to apprise the barkeeper that wine is a mocker.  It was a
dreadfully hot day.  Pretty soon a neighbor came in and seeing Clark,
said:
\subparagraph{}   "Champ, it is not right to leave that mule out there in the sun.
He'll roast, sure! -- he was smoking as I passed him."
\subparagraph{}   "O, he's all right," said Clark, lightly; "he's an inveterate
smoker."
\subparagraph{}   The neighbor took a lemonade, but shook his head and repeated that 
it was not right. 
\subparagraph{}   He was a conspirator.  There had been a fire the night before:  a 
stable just around the corner had burned and a number of horses had 
put on their immortality, among them a young colt, which was roasted
to a rich nut-brown.  Some of the boys had turned Mr. Clark's mule
loose and substituted the mortal part of the colt.  Presently another
man entered the saloon.
\subparagraph{}   "For mercy's sake!" he said, taking it with sugar, "do remove that 
mule, barkeeper:  it smells." 
\subparagraph{}   "Yes," interposed Clark, "that animal has the best nose in 
Missouri.  But if he doesn't mind, you shouldn't." 
\subparagraph{}   In the course of human events Mr. Clark went out, and there, 
apparently, lay the incinerated and shrunken remains of his charger. 
The boys did not have any fun out of Mr. Clarke, who looked at the
body and, with the non-committal expression to which he owes so much
of his political preferment, went away.  But walking home late that
night he saw his mule standing silent and solemn by the wayside in the
misty moonlight.  Mentioning the name of Helen Blazes with uncommon
emphasis, Mr. Clark took the back track as hard as ever he could hook
it, and passed the night in town.
\\
\paragraph{}   General H.H. Wotherspoon, president of the Army War College, has a 
pet rib-nosed baboon, an animal of uncommon intelligence but 
imperfectly beautiful.  Returning to his apartment one evening, the
General was surprised and pained to find Adam (for so the creature is
named, the general being a Darwinian) sitting up for him and wearing
his master's best uniform coat, epaulettes and all.
\subparagraph{}   "You confounded remote ancestor!" thundered the great strategist, 
"what do you mean by being out of bed after naps? -- and with my coat 
on!"
\subparagraph{}   Adam rose and with a reproachful look got down on all fours in the 
manner of his kind and, scuffling across the room to a table, returned 
with a visiting-card:  General Barry had called and, judging by an
empty champagne bottle and several cigar-stumps, had been hospitably
entertained while waiting.  The general apologized to his faithful
progenitor and retired.  The next day he met General Barry, who said:
\subparagraph{}   "Spoon, old man, when leaving you last evening I forgot to ask you 
about those excellent cigars.  Where did you get them?" 
\subparagraph{}   General Wotherspoon did not deign to reply, but walked away. 
  "Pardon me, please," said Barry, moving after him; "I was joking 
of course.  Why, I knew it was not you before I had been in the room 
fifteen minutes."

\paragraph{SUCCESS, n.}  The one unpardonable sin against one's fellows.  In
literature, and particularly in poetry, the elements of success are
exceedingly simple, and are admirably set forth in the following lines
by the reverend Father Gassalasca Jape, entitled, for some mysterious
reason, "John A. Joyce."

\begin{quote}   The bard who would prosper must carry a book, \\
      Do his thinking in prose and wear \\
  A crimson cravat, a far-away look \\
      And a head of hexameter hair. \\
  Be thin in your thought and your body'll be fat; \\
  If you wear your hair long you needn't your hat.  \end{quote}

\paragraph{SUFFRAGE, n.}  Expression of opinion by means of a ballot.  The right
of suffrage (which is held to be both a privilege and a duty) means,
as commonly interpreted, the right to vote for the man of another
man's choice, and is highly prized.  Refusal to do so has the bad name
of "incivism."  The incivilian, however, cannot be properly arraigned
for his crime, for there is no legitimate accuser.  If the accuser is
himself guilty he has no standing in the court of opinion; if not, he
profits by the crime, for A's abstention from voting gives greater
weight to the vote of B.  By female suffrage is meant the right of a
woman to vote as some man tells her to.  It is based on female
responsibility, which is somewhat limited.  The woman most eager to
jump out of her petticoat to assert her rights is first to jump back
into it when threatened with a switching for misusing them.

\paragraph{SYCOPHANT, n.}  One who approaches Greatness on his belly so that he
may not be commanded to turn and be kicked.  He is sometimes an
editor.

\begin{quote}   As the lean leech, its victim found, is pleased \\
  To fix itself upon a part diseased \\
  Till, its black hide distended with bad blood, \\
  It drops to die of surfeit in the mud, \\
  So the base sycophant with joy descries \\
  His neighbor's weak spot and his mouth applies, \\
  Gorges and prospers like the leech, although, \\
  Unlike that reptile, he will not let go. \\
  Gelasma, if it paid you to devote \\
  Your talent to the service of a goat, \\
  Showing by forceful logic that its beard \\
  Is more than Aaron's fit to be revered; \\
  If to the task of honoring its smell \\
  Profit had prompted you, and love as well, \\
  The world would benefit at last by you \\
  And wealthy malefactors weep anew -- \\
  Your favor for a moment's space denied \\
  And to the nobler object turned aside. \\
  Is't not enough that thrifty millionaires \\
  Who loot in freight and spoliate in fares, \\
  Or, cursed with consciences that bid them fly \\
  To safer villainies of darker dye, \\
  Forswearing robbery and fain, instead, \\
  To steal (they call it "cornering") our bread \\
  May see you groveling their boots to lick \\
  And begging for the favor of a kick? \\
  Still must you follow to the bitter end \\
  Your sycophantic disposition's trend, \\
  And in your eagerness to please the rich \\
  Hunt hungry sinners to their final ditch? \\
  In Morgan's praise you smite the sounding wire, \\
  And sing hosannas to great Havemeyher! \\
  What's Satan done that him you should eschew? \\
  He too is reeking rich -- deducting {\em you}.  \end{quote}

\paragraph{SYLLOGISM, n.}  A logical formula consisting of a major and a minor
assumption and an inconsequent.  (See LOGIC.)

\paragraph{SYLPH, n.}  An immaterial but visible being that inhabited the air when
the air was an element and before it was fatally polluted with factory
smoke, sewer gas and similar products of civilization.  Sylphs were
allied to gnomes, nymphs and salamanders, which dwelt, respectively,
in earth, water and fire, all now insalubrious.  Sylphs, like fowls of
the air, were male and female, to no purpose, apparently, for if they
had progeny they must have nested in accessible places, none of the
chicks having ever been seen.

\paragraph{SYMBOL, n.}  Something that is supposed to typify or stand for
something else.  Many symbols are mere "survivals" -- things which
having no longer any utility continue to exist because we have
inherited the tendency to make them; as funereal urns carved on
memorial monuments.  They were once real urns holding the ashes of the
dead.  We cannot stop making them, but we can give them a name that
conceals our helplessness.

\paragraph{SYMBOLIC, adj.}  Pertaining to symbols and the use and interpretation
of symbols.

\begin{quote}   They say 'tis conscience feels compunction; \\
  I hold that that's the stomach's function, \\
  For of the sinner I have noted \\
  That when he's sinned he's somewhat bloated, \\
  Or ill some other ghastly fashion \\
  Within that bowel of compassion. \\
  True, I believe the only sinner \\
  Is he that eats a shabby dinner. \\
  You know how Adam with good reason, \\
  For eating apples out of season, \\
  Was "cursed."  But that is all symbolic: \\
  The truth is, Adam had the colic. \\
 \\
G.J. \end{quote}




\section*{T}



\paragraph{T} the twentieth letter of the English alphabet, was by the Greeks
absurdly called {\em tau}.  In the alphabet whence ours comes it had the
form of the rude corkscrew of the period, and when it stood alone
(which was more than the Phoenicians could always do) signified
{\em Tallegal}, translated by the learned Dr. Brownrigg, "tanglefoot."

\paragraph{TABLE D'HOTE, n.}  A caterer's thrifty concession to the universal
passion for irresponsibility.

\begin{quote}   Old Paunchinello, freshly wed, \\
      Took Madam P. to table, \\
  And there deliriously fed \\
      As fast as he was able. \\
 \\
  "I dote upon good grub," he cried, \\
      Intent upon its throatage. \\
  "Ah, yes," said the neglected bride, \\
      "You're in your {\em table d'hotage}." \\
 \\
Associated Poets \end{quote}


\paragraph{TAIL, n.}  The part of an animal's spine that has transcended its
natural limitations to set up an independent existence in a world of
its own.  Excepting in its foetal state, Man is without a tail, a
privation of which he attests an hereditary and uneasy consciousness
by the coat-skirt of the male and the train of the female, and by a
marked tendency to ornament that part of his attire where the tail
should be, and indubitably once was.  This tendency is most observable
in the female of the species, in whom the ancestral sense is strong
and persistent.  The tailed men described by Lord Monboddo are now
generally regarded as a product of an imagination unusually
susceptible to influences generated in the golden age of our pithecan
past.

\paragraph{TAKE, v.t.}  To acquire, frequently by force but preferably by stealth.

\paragraph{TALK, v.t.}  To commit an indiscretion without temptation, from an
impulse without purpose.

\paragraph{TARIFF, n.}  A scale of taxes on imports, designed to protect the
domestic producer against the greed of his consumer.

\begin{quote}   The Enemy of Human Souls \\
  Sat grieving at the cost of coals; \\
  For Hell had been annexed of late, \\
  And was a sovereign Southern State. \\
 \\
  "It were no more than right," said he, \\
  "That I should get my fuel free. \\
  The duty, neither just nor wise, \\
  Compels me to economize -- \\
  Whereby my broilers, every one, \\
  Are execrably underdone. \\
  What would they have? -- although I yearn \\
  To do them nicely to a turn, \\
  I can't afford an honest heat. \\
  This tariff makes even devils cheat! \\
  I'm ruined, and my humble trade \\
  All rascals may at will invade: \\
  Beneath my nose the public press \\
  Outdoes me in sulphureousness; \\
  The bar ingeniously applies \\
  To my undoing my own lies; \\
  My medicines the doctors use \\
  (Albeit vainly) to refuse \\
  To me my fair and rightful prey \\
  And keep their own in shape to pay; \\
  The preachers by example teach \\
  What, scorning to perform, I teach; \\
  And statesmen, aping me, all make \\
  More promises than they can break. \\
  Against such competition I \\
  Lift up a disregarded cry. \\
  Since all ignore my just complaint, \\
  By Hokey-Pokey!  I'll turn saint!" \\
  Now, the Republicans, who all \\
  Are saints, began at once to bawl \\
  Against {\em his} competition; so \\
  There was a devil of a go! \\
  They locked horns with him, tete-a-tete \\
  In acrimonious debate, \\
  Till Democrats, forlorn and lone, \\
  Had hopes of coming by their own. \\
  That evil to avert, in haste \\
  The two belligerents embraced; \\
  But since 'twere wicked to relax \\
  A tittle of the Sacred Tax, \\
  'Twas finally agreed to grant \\
  The bold Insurgent-protestant \\
  A bounty on each soul that fell \\
  Into his ineffectual Hell. \\
 \\
Edam Smith \end{quote}


\paragraph{TECHNICALITY, n.}  In an English court a man named Home was tried for
slander in having accused his neighbor of murder.  His exact words
were:  "Sir Thomas Holt hath taken a cleaver and stricken his cook
upon the head, so that one side of the head fell upon one shoulder and
the other side upon the other shoulder."  The defendant was acquitted
by instruction of the court, the learned judges holding that the words
did not charge murder, for they did not affirm the death of the cook,
that being only an inference.

\paragraph{TEDIUM, n.}  Ennui, the state or condition of one that is bored.  Many
fanciful derivations of the word have been affirmed, but so high an
authority as Father Jape says that it comes from a very obvious
source -- the first words of the ancient Latin hymn {\em Te Deum
Laudamus}.  In this apparently natural derivation there is something
that saddens.

\paragraph{TEETOTALER, n.}  One who abstains from strong drink, sometimes totally,
sometimes tolerably totally.

\paragraph{TELEPHONE, n.}  An invention of the devil which abrogates some of the
advantages of making a disagreeable person keep his distance.

\paragraph{TELESCOPE, n.}  A device having a relation to the eye similar to that
of the telephone to the ear, enabling distant objects to plague us
with a multitude of needless details.  Luckily it is unprovided with a
bell summoning us to the sacrifice.

\paragraph{TENACITY, n.}  A certain quality of the human hand in its relation to
the coin of the realm.  It attains its highest development in the hand
of authority and is considered a serviceable equipment for a career in
politics.  The following illustrative lines were written of a
Californian gentleman in high political preferment, who has passed to
his accounting:

\begin{quote}   Of such tenacity his grip \\
  That nothing from his hand can slip. \\
  Well-buttered eels you may o'erwhelm \\
  In tubs of liquid slippery-elm \\
  In vain -- from his detaining pinch \\
  They cannot struggle half an inch! \\
  'Tis lucky that he so is planned \\
  That breath he draws not with his hand, \\
  For if he did, so great his greed \\
  He'd draw his last with eager speed. \\
  Nay, that were well, you say.  Not so \\
  He'd draw but never let it go!  \end{quote}

\paragraph{THEOSOPHY, n.}  An ancient faith having all the certitude of religion
and all the mystery of science.  The modern Theosophist holds, with
the Buddhists, that we live an incalculable number of times on this
earth, in as many several bodies, because one life is not long enough
for our complete spiritual development; that is, a single lifetime
does not suffice for us to become as wise and good as we choose to
wish to become.  To be absolutely wise and good -- that is perfection;
and the Theosophist is so keen-sighted as to have observed that
everything desirous of improvement eventually attains perfection.
Less competent observers are disposed to except cats, which seem
neither wiser nor better than they were last year.  The greatest and
fattest of recent Theosophists was the late Madame Blavatsky, who had
no cat.

\paragraph{TIGHTS, n.}  An habiliment of the stage designed to reinforce the
general acclamation of the press agent with a particular publicity.
Public attention was once somewhat diverted from this garment to Miss
Lillian Russell's refusal to wear it, and many were the conjectures as
to her motive, the guess of Miss Pauline Hall showing a high order of
ingenuity and sustained reflection.  It was Miss Hall's belief that
nature had not endowed Miss Russell with beautiful legs.  This theory
was impossible of acceptance by the male understanding, but the
conception of a faulty female leg was of so prodigious originality as
to rank among the most brilliant feats of philosophical speculation!
It is strange that in all the controversy regarding Miss Russell's
aversion to tights no one seems to have thought to ascribe it to what
was known among the ancients as "modesty."  The nature of that
sentiment is now imperfectly understood, and possibly incapable of
exposition with the vocabulary that remains to us.  The study of lost
arts has, however, been recently revived and some of the arts
themselves recovered.  This is an epoch of {\em renaissances}, and there
is ground for hope that the primitive "blush" may be dragged from its
hiding-place amongst the tombs of antiquity and hissed on to the
stage.

\paragraph{TOMB, n.}  The House of Indifference.  Tombs are now by common consent
invested with a certain sanctity, but when they have been long
tenanted it is considered no sin to break them open and rifle them,
the famous Egyptologist, Dr. Huggyns, explaining that a tomb may be
innocently "glened" as soon as its occupant is done "smellynge," the
soul being then all exhaled.  This reasonable view is now generally
accepted by archaeologists, whereby the noble science of Curiosity has
been greatly dignified.

\paragraph{TOPE, v.}  To tipple, booze, swill, soak, guzzle, lush, bib, or swig.
In the individual, toping is regarded with disesteem, but toping
nations are in the forefront of civilization and power.  When pitted
against the hard-drinking Christians the abstemious Mahometans go down
like grass before the scythe.  In India one hundred thousand beef-
eating and brandy-and-soda guzzling Britons hold in subjection two
hundred and fifty million vegetarian abstainers of the same Aryan
race.  With what an easy grace the whisky-loving American pushed the
temperate Spaniard out of his possessions!  From the time when the
Berserkers ravaged all the coasts of western Europe and lay drunk in
every conquered port it has been the same way:  everywhere the nations
that drink too much are observed to fight rather well and not too
righteously.  Wherefore the estimable old ladies who abolished the
canteen from the American army may justly boast of having materially
augmented the nation's military power.

\paragraph{TORTOISE, n.}  A creature thoughtfully created to supply occasion for
the following lines by the illustrious Ambat Delaso:



\begin{quote}
  TO MY PET TORTOISE \\
  \\
  My friend, you are not graceful -- not at all; \\
  Your gait's between a stagger and a sprawl. \\
 \\
  Nor are you beautiful:  your head's a snake's \\
  To look at, and I do not doubt it aches. \\
 \\
  As to your feet, they'd make an angel weep. \\
  'Tis true you take them in whene'er you sleep. \\
 \\
  No, you're not pretty, but you have, I own, \\
  A certain firmness -- mostly you're [sic] backbone. \\
 \\
  Firmness and strength (you have a giant's thews) \\
  Are virtues that the great know how to use -- \\
 \\
  I wish that they did not; yet, on the whole, \\
  You lack -- excuse my mentioning it -- Soul. \\
 \\
  So, to be candid, unreserved and true, \\
  I'd rather you were I than I were you. \\
 \\
  Perhaps, however, in a time to be, \\
  When Man's extinct, a better world may see \\
 \\
  Your progeny in power and control, \\
  Due to the genesis and growth of Soul. \\
 \\
  So I salute you as a reptile grand \\
  Predestined to regenerate the land. \\
 \\
  Father of Possibilities, O deign \\
  To accept the homage of a dying reign! \\
 \\
  In the far region of the unforeknown \\
  I dream a tortoise upon every throne. \\
 \\
  I see an Emperor his head withdraw \\
  Into his carapace for fear of Law; \\
 \\
  A King who carries something else than fat, \\
  Howe'er acceptably he carries that; \\
 \\
  A President not strenuously bent \\
  On punishment of audible dissent -- \\
 \\
  Who never shot (it were a vain attack) \\
  An armed or unarmed tortoise in the back; \\
 \\
  Subject and citizens that feel no need \\
  To make the March of Mind a wild stampede; \\
 \\
  All progress slow, contemplative, sedate, \\
  And "Take your time" the word, in Church and State. \\
 \\
  O Tortoise, 'tis a happy, happy dream, \\
  My glorious testudinous regime! \\
 \\
  I wish in Eden you'd brought this about \\
  By slouching in and chasing Adam out.  \end{quote}

\paragraph{TREE, n.}  A tall vegetable intended by nature to serve as a penal
apparatus, though through a miscarriage of justice most trees bear
only a negligible fruit, or none at all.  When naturally fruited, the
tree is a beneficient agency of civilization and an important factor
in public morals.  In the stern West and the sensitive South its fruit
(white and black respectively) though not eaten, is agreeable to the
public taste and, though not exported, profitable to the general
welfare.  That the legitimate relation of the tree to justice was no
discovery of Judge Lynch (who, indeed, conceded it no primacy over the
lamp-post and the bridge-girder) is made plain by the following
passage from Morryster, who antedated him by two centuries:

\begin{quote}       While in yt londe I was carried to see ye Ghogo tree, whereof \\
  I had hearde moch talk; but sayynge yt I saw naught remarkabyll in \\
  it, ye hed manne of ye villayge where it grewe made answer as \\
  followeth: \\
      "Ye tree is not nowe in fruite, but in his seasonne you shall \\
  see dependynge fr. his braunches all soch as have affroynted ye \\
  King his Majesty." \\
      And I was furder tolde yt ye worde "Ghogo" sygnifyeth in yr \\
  tong ye same as "rapscal" in our owne. \\
 \\
{\em Trauvells in ye Easte} \end{quote}


\paragraph{TRIAL, n.}  A formal inquiry designed to prove and put upon record the
blameless characters of judges, advocates and jurors.  In order to
effect this purpose it is necessary to supply a contrast in the person
of one who is called the defendant, the prisoner, or the accused.  If
the contrast is made sufficiently clear this person is made to undergo
such an affliction as will give the virtuous gentlemen a comfortable
sense of their immunity, added to that of their worth.  In our day the
accused is usually a human being, or a socialist, but in mediaeval
times, animals, fishes, reptiles and insects were brought to trial.  A
beast that had taken human life, or practiced sorcery, was duly
arrested, tried and, if condemned, put to death by the public
executioner.  Insects ravaging grain fields, orchards or vineyards
were cited to appeal by counsel before a civil tribunal, and after
testimony, argument and condemnation, if they continued {\em in
contumaciam} the matter was taken to a high ecclesiastical court,
where they were solemnly excommunicated and anathematized.  In a
street of Toledo, some pigs that had wickedly run between the
viceroy's legs, upsetting him, were arrested on a warrant, tried and
punished.  In Naples and ass was condemned to be burned at the stake,
but the sentence appears not to have been executed.  D'Addosio relates
from the court records many trials of pigs, bulls, horses, cocks,
dogs, goats, etc., greatly, it is believed, to the betterment of their
conduct and morals.  In 1451 a suit was brought against the leeches
infesting some ponds about Berne, and the Bishop of Lausanne,
instructed by the faculty of Heidelberg University, directed that some
of "the aquatic worms" be brought before the local magistracy.  This
was done and the leeches, both present and absent, were ordered to
leave the places that they had infested within three days on pain of
incurring "the malediction of God."  In the voluminous records of this
{\em cause celebre} nothing is found to show whether the offenders braved
the punishment, or departed forthwith out of that inhospitable
jurisdiction.

\paragraph{TRICHINOSIS, n.}  The pig's reply to proponents of porcophagy.
\begin{quote}   Moses Mendlessohn having fallen ill sent for a Christian
physician, who at once diagnosed the philosopher's disorder as 
trichinosis, but tactfully gave it another name.  "You need and
immediate change of diet," he said; "you must eat six ounces of pork
every other day." \\
  \\
   "Pork?" shrieked the patient -- "pork?  Nothing shall induce me to
touch it!" \\
  "Do you mean that?" the doctor gravely asked. \\
  "I swear it!" \\
  "Good! -- then I will undertake to cure you."
\end{quote}

\paragraph{TRINITY, n.}  In the multiplex theism of certain Christian churches,
three entirely distinct deities consistent with only one.  Subordinate
deities of the polytheistic faith, such as devils and angels, are not
dowered with the power of combination, and must urge individually
their claims to adoration and propitiation.  The Trinity is one of the
most sublime mysteries of our holy religion.  In rejecting it because
it is incomprehensible, Unitarians betray their inadequate sense of
theological fundamentals.  In religion we believe only what we do not
understand, except in the instance of an intelligible doctrine that
contradicts an incomprehensible one.  In that case we believe the
former as a part of the latter.

\paragraph{TROGLODYTE, n.}  Specifically, a cave-dweller of the paleolithic
period, after the Tree and before the Flat.  A famous community of
troglodytes dwelt with David in the Cave of Adullam.  The colony
consisted of "every one that was in distress, and every one that was
in debt, and every one that was discontented" -- in brief, all the
Socialists of Judah.

\paragraph{TRUCE, n.}  Friendship.

\paragraph{TRUTH, n.}  An ingenious compound of desirability and appearance.
Discovery of truth is the sole purpose of philosophy, which is the
most ancient occupation of the human mind and has a fair prospect of
existing with increasing activity to the end of time.

\paragraph{TRUTHFUL, adj.}  Dumb and illiterate.

\paragraph{TRUST, n.}  In American politics, a large corporation composed in
greater part of thrifty working men, widows of small means, orphans in
the care of guardians and the courts, with many similar malefactors
and public enemies.

\paragraph{TURKEY, n.}  A large bird whose flesh when eaten on certain religious
anniversaries has the peculiar property of attesting piety and
gratitude.  Incidentally, it is pretty good eating.

\paragraph{TWICE, adv.}  Once too often.

\paragraph{TYPE, n.}  Pestilent bits of metal suspected of destroying
civilization and enlightenment, despite their obvious agency in this
incomparable dictionary.

\paragraph{TZETZE (or TSETSE) FLY, n.}  An African insect ({\em Glossina morsitans})
whose bite is commonly regarded as nature's most efficacious remedy
for insomnia, though some patients prefer that of the American
novelist ({\em Mendax interminabilis}).



\section*{U}



\paragraph{UBIQUITY, n.}  The gift or power of being in all places at one time,
but not in all places at all times, which is omnipresence, an
attribute of God and the luminiferous ether only.  This important
distinction between ubiquity and omnipresence was not clear to the
mediaeval Church and there was much bloodshed about it.  Certain
Lutherans, who affirmed the presence everywhere of Christ's body were
known as Ubiquitarians.  For this error they were doubtless damned,
for Christ's body is present only in the eucharist, though that
sacrament may be performed in more than one place simultaneously.  In
recent times ubiquity has not always been understood -- not even by
Sir Boyle Roche, for example, who held that a man cannot be in two
places at once unless he is a bird.

\paragraph{UGLINESS, n.}  A gift of the gods to certain women, entailing virtue
without humility.

\paragraph{ULTIMATUM, n.}  In diplomacy, a last demand before resorting to
concessions.

\begin{quote}   Having received an ultimatum from Austria, the Turkish Ministry 
met to consider it. \\
  \\
   "O servant of the Prophet," said the Sheik of the Imperial Chibouk 
to the Mamoosh of the Invincible Army, "how many unconquerable 
soldiers have we in arms?" \\
  \\
   "Upholder of the Faith," that dignitary replied after examining 
his memoranda, "they are in numbers as the leaves of the forest!" \\
  \\
   "And how many impenetrable battleships strike terror to the hearts 
of all Christian swine?" he asked the Imaum of the Ever Victorious 
Navy. \\
  \\
   "Uncle of the Full Moon," was the reply, "deign to know that they 
are as the waves of the ocean, the sands of the desert and the stars 
of Heaven!" \\
  \\
   For eight hours the broad brow of the Sheik of the Imperial 
Chibouk was corrugated with evidences of deep thought:  he was 
calculating the chances of war.  Then, "Sons of angels," he said, "the
die is cast!  I shall suggest to the Ulema of the Imperial Ear that he
advise inaction.  In the name of Allah, the council is adjourned."
\end{quote}

\paragraph{UN-AMERICAN, adj.}  Wicked, intolerable, heathenish.

\paragraph{UNCTION, n.}  An oiling, or greasing.  The rite of extreme unction
consists in touching with oil consecrated by a bishop several parts of
the body of one engaged in dying.  Marbury relates that after the rite
had been administered to a certain wicked English nobleman it was
discovered that the oil had not been properly consecrated and no other
could be obtained.  When informed of this the sick man said in anger:
"Then I'll be damned if I die!"
\subparagraph{}   "My son," said the priest, "this is what we fear."

\paragraph{UNDERSTANDING, n.}  A cerebral secretion that enables one having it to
know a house from a horse by the roof on the house.  Its nature and
laws have been exhaustively expounded by Locke, who rode a house, and
Kant, who lived in a horse.

\begin{quote}   His understanding was so keen \\
  That all things which he'd felt, heard, seen, \\
  He could interpret without fail \\
  If he was in or out of jail. \\
  He wrote at Inspiration's call \\
  Deep disquisitions on them all, \\
  Then, pent at last in an asylum, \\
  Performed the service to compile 'em. \\
  So great a writer, all men swore, \\
  They never had not read before. \\
 \\
Jorrock Wormley \end{quote}


\paragraph{UNITARIAN, n.}  One who denies the divinity of a Trinitarian.

\paragraph{UNIVERSALIST, n.}  One who forgoes the advantage of a Hell for persons
of another faith.

\paragraph{URBANITY, n.}  The kind of civility that urban observers ascribe to
dwellers in all cities but New York.  Its commonest expression is
heard in the words, "I beg your pardon," and it is not consistent with
disregard of the rights of others.

\begin{quote}   The owner of a powder mill \\
  Was musing on a distant hill -- \\
      Something his mind foreboded -- \\
  When from the cloudless sky there fell \\
  A deviled human kidney!  Well, \\
      The man's mill had exploded. \\
  His hat he lifted from his head; \\
  "I beg your pardon, sir," he said; \\
      "I didn't know 'twas loaded." \\
 \\
Swatkin \end{quote}


\paragraph{USAGE, n.}  The First Person of the literary Trinity, the Second and
Third being Custom and Conventionality.  Imbued with a decent
reverence for this Holy Triad an industrious writer may hope to
produce books that will live as long as the fashion.

\paragraph{UXORIOUSNESS, n.}  A perverted affection that has strayed to one's own
wife.



\section*{V}



\paragraph{VALOR, n.}  A soldierly compound of vanity, duty and the gambler's
hope.
\begin{quote}   "Why have you halted?" roared the commander of a division and
Chickamauga, who had ordered a charge; "move forward, sir, at once." \\
  \\
   "General," said the commander of the delinquent brigade, "I am
persuaded that any further display of valor by my troops will bring
them into collision with the enemy."
\end{quote}

\paragraph{VANITY, n.}  The tribute of a fool to the worth of the nearest ass.

\begin{quote}   They say that hens do cackle loudest when \\
      There's nothing vital in the eggs they've laid; \\
      And there are hens, professing to have made \\
  A study of mankind, who say that men \\
  Whose business 'tis to drive the tongue or pen \\
      Make the most clamorous fanfaronade \\
      O'er their most worthless work; and I'm afraid \\
  They're not entirely different from the hen. \\
  Lo! the drum-major in his coat of gold, \\
      His blazing breeches and high-towering cap -- \\
  Imperiously pompous, grandly bold, \\
      Grim, resolute, an awe-inspiring chap! \\
  Who'd think this gorgeous creature's only virtue \\
  Is that in battle he will never hurt you? \\
 \\
Hannibal Hunsiker \end{quote}


\paragraph{VIRTUES, n.pl.}  Certain abstentions.

\paragraph{VITUPERATION, n.}  Saite, as understood by dunces and all such as
suffer from an impediment in their wit.

\paragraph{VOTE, n.}  The instrument and symbol of a freeman's power to make a
fool of himself and a wreck of his country.



\section*{W}



\paragraph{W (double U)} has, of all the letters in our alphabet, the only
cumbrous name, the names of the others being monosyllabic.  This
advantage of the Roman alphabet over the Grecian is the more valued
after audibly spelling out some simple Greek word, like
{\em epixoriambikos}.  Still, it is now thought by the learned that other
agencies than the difference of the two alphabets may have been
concerned in the decline of "the glory that was Greece" and the rise
of "the grandeur that was Rome."  There can be no doubt, however, that
by simplifying the name of W (calling it "wow," for example) our
civilization could be, if not promoted, at least better endured.

\paragraph{WALL STREET, n.}  A symbol for sin for every devil to rebuke.  That
Wall Street is a den of thieves is a belief that serves every
unsuccessful thief in place of a hope in Heaven.  Even the great and
good Andrew Carnegie has made his profession of faith in the matter.

\begin{quote}   Carnegie the dauntless has uttered his call \\
  To battle:  "The brokers are parasites all!" \\
  Carnegie, Carnegie, you'll never prevail; \\
  Keep the wind of your slogan to belly your sail, \\
  Go back to your isle of perpetual brume, \\
  Silence your pibroch, doff tartan and plume: \\
  Ben Lomond is calling his son from the fray -- \\
  Fly, fly from the region of Wall Street away! \\
  While still you're possessed of a single baubee \\
  (I wish it were pledged to endowment of me) \\
  'Twere wise to retreat from the wars of finance \\
  Lest its value decline ere your credit advance. \\
  For a man 'twixt a king of finance and the sea, \\
  Carnegie, Carnegie, your tongue is too free! \\
 \\
Anonymus Bink \end{quote}


\paragraph{WAR, n.}  A by-product of the arts of peace.  The most menacing
political condition is a period of international amity.  The student
of history who has not been taught to expect the unexpected may justly
boast himself inaccessible to the light.  "In time of peace prepare
for war" has a deeper meaning than is commonly discerned; it means,
not merely that all things earthly have an end -- that change is the
one immutable and eternal law -- but that the soil of peace is thickly
sown with the seeds of war and singularly suited to their germination
and growth.  It was when Kubla Khan had decreed his "stately pleasure
dome" -- when, that is to say, there were peace and fat feasting in
Xanadu -- that he

\begin{quote}                       heard from afar \\
  Ancestral voices prophesying war.
 \end{quote}
  
\subparagraph{}  One of the greatest of poets, Coleridge was one of the wisest of
men, and it was not for nothing that he read us this parable.  Let us
have a little less of "hands across the sea," and a little more of
that elemental distrust that is the security of nations.  War loves to
come like a thief in the night; professions of eternal amity provide
the night.

\paragraph{WASHINGTONIAN, n.}  A Potomac tribesman who exchanged the privilege of
governing himself for the advantage of good government.  In justice to
him it should be said that he did not want to.

\begin{quote}   They took away his vote and gave instead \\
  The right, when he had earned, to {\em eat} his bread. \\
  In vain -- he clamors for his "boss," pour soul, \\
  To come again and part him from his roll. \\
 \\
Offenbach Stutz \end{quote}


\paragraph{WEAKNESSES, n.pl.}  Certain primal powers of Tyrant Woman wherewith she
holds dominion over the male of her species, binding him to the
service of her will and paralyzing his rebellious energies.

\paragraph{WEATHER, n.}  The climate of the hour.  A permanent topic of
conversation among persons whom it does not interest, but who have
inherited the tendency to chatter about it from naked arboreal
ancestors whom it keenly concerned.  The setting up official weather
bureaus and their maintenance in mendacity prove that even governments
are accessible to suasion by the rude forefathers of the jungle.

\begin{quote}   Once I dipt into the future far as human eye could see, \\
  And I saw the Chief Forecaster, dead as any one can be -- \\
  Dead and damned and shut in Hades as a liar from his birth, \\
  With a record of unreason seldom paralleled on earth. \\
  While I looked he reared him solemnly, that incandescent youth, \\
  From the coals that he'd preferred to the advantages of truth. \\
  He cast his eyes about him and above him; then he wrote \\
  On a slab of thin asbestos what I venture here to quote -- \\
  For I read it in the rose-light of the everlasting glow: \\
  "Cloudy; variable winds, with local showers; cooler; snow." \\
 \\
Halcyon Jones \end{quote}


\paragraph{WEDDING, n.}  A ceremony at which two persons undertake to become one,
one undertakes to become nothing, and nothing undertakes to become
supportable.

\paragraph{WEREWOLF, n.}  A wolf that was once, or is sometimes, a man.  All
werewolves are of evil disposition, having assumed a bestial form to
gratify a beastial appetite, but some, transformed by sorcery, are as
humane and is consistent with an acquired taste for human flesh.
\subparagraph{}   Some Bavarian peasants having caught a wolf one evening, tied it
to a post by the tail and went to bed.  The next morning nothing was
there!  Greatly perplexed, they consulted the local priest, who told
them that their captive was undoubtedly a werewolf and had resumed its
human for during the night.  "The next time that you take a wolf," the
good man said, "see that you chain it by the leg, and in the morning
you will find a Lutheran."

\paragraph{WHANGDEPOOTENAWAH, n.}  In the Ojibwa tongue, disaster; an unexpected
affliction that strikes hard.

\begin{quote}   Should you ask me whence this laughter, \\
  Whence this audible big-smiling, \\
  With its labial extension, \\
  With its maxillar distortion \\
  And its diaphragmic rhythmus \\
  Like the billowing of an ocean, \\
  Like the shaking of a carpet, \\
  I should answer, I should tell you: \\
  From the great deeps of the spirit, \\
  From the unplummeted abysmus \\
  Of the soul this laughter welleth \\
  As the fountain, the gug-guggle, \\
  Like the river from the canon [sic], \\
  To entoken and give warning \\
  That my present mood is sunny. \\
  Should you ask me further question -- \\
  Why the great deeps of the spirit, \\
  Why the unplummeted abysmus \\
  Of the soule extrudes this laughter, \\
  This all audible big-smiling, \\
  I should answer, I should tell you \\
  With a white heart, tumpitumpy, \\
  With a true tongue, honest Injun: \\
  William Bryan, he has Caught It, \\
  Caught the Whangdepootenawah! \\
 \\
  Is't the sandhill crane, the shankank, \\
  Standing in the marsh, the kneedeep, \\
  Standing silent in the kneedeep \\
  With his wing-tips crossed behind him \\
  And his neck close-reefed before him, \\
  With his bill, his william, buried \\
  In the down upon his bosom, \\
  With his head retracted inly, \\
  While his shoulders overlook it? \\
  Does the sandhill crane, the shankank, \\
  Shiver grayly in the north wind, \\
  Wishing he had died when little, \\
  As the sparrow, the chipchip, does? \\
  No 'tis not the Shankank standing, \\
  Standing in the gray and dismal \\
  Marsh, the gray and dismal kneedeep. \\
  No, 'tis peerless William Bryan \\
  Realizing that he's Caught It, \\
  Caught the Whangdepootenawah!  \end{quote}

\paragraph{WHEAT, n.}  A cereal from which a tolerably good whisky can with some
difficulty be made, and which is used also for bread.  The French are
said to eat more bread {\em per capita} of population than any other
people, which is natural, for only they know how to make the stuff
palatable.

\paragraph{WHITE, adj. and n.}  Black.

\paragraph{WIDOW, n.}  A pathetic figure that the Christian world has agreed to
take humorously, although Christ's tenderness towards widows was one
of the most marked features of his character.

\paragraph{WINE, n.}  Fermented grape-juice known to the Women's Christian Union
as "liquor," sometimes as "rum."  Wine, madam, is God's next best gift
to man.

\paragraph{WIT, n.}  The salt with which the American humorist spoils his
intellectual cookery by leaving it out.

\paragraph{WITCH, n.}  (1)  Any ugly and repulsive old woman, in a wicked league
with the devil.  (2)  A beautiful and attractive young woman, in
wickedness a league beyond the devil.

\paragraph{WITTICISM, n.}  A sharp and clever remark, usually quoted, and seldom
noted; what the Philistine is pleased to call a "joke."

\paragraph{WOMAN, n.}

\begin{quote}       An animal usually living in the vicinity of Man, and having a \\
  rudimentary susceptibility to domestication.  It is credited by \\
  many of the elder zoologists with a certain vestigial docility \\
  acquired in a former state of seclusion, but naturalists of the \\
  postsusananthony period, having no knowledge of the seclusion, \\
  deny the virtue and declare that such as creation's dawn beheld, \\
  it roareth now.  The species is the most widely distributed of all \\
  beasts of prey, infesting all habitable parts of the globe, from \\
  Greeland's spicy mountains to India's moral strand.  The popular \\
  name (wolfman) is incorrect, for the creature is of the cat kind. \\
  The woman is lithe and graceful in its movement, especially the \\
  American variety ({\em felis pugnans}), is omnivorous and can be \\
  taught not to talk. \\
 \\
Balthasar Pober \end{quote}


\paragraph{WORMS'-MEAT, n.}  The finished product of which we are the raw
material.  The contents of the Taj Mahal, the Tombeau Napoleon and the
Granitarium.  Worms'-meat is usually outlasted by the structure that
houses it, but "this too must pass away."  Probably the silliest work
in which a human being can engage is construction of a tomb for
himself.  The solemn purpose cannot dignify, but only accentuates by
contrast the foreknown futility.

\begin{quote}   Ambitious fool! so mad to be a show! \\
  How profitless the labor you bestow \\
      Upon a dwelling whose magnificence \\
  The tenant neither can admire nor know. \\
 \\
  Build deep, build high, build massive as you can, \\
  The wanton grass-roots will defeat the plan \\
      By shouldering asunder all the stones \\
  In what to you would be a moment's span. \\
 \\
  Time to the dead so all unreckoned flies \\
  That when your marble is all dust, arise, \\
      If wakened, stretch your limbs and yawn -- \\
  You'll think you scarcely can have closed your eyes. \\
 \\
  What though of all man's works your tomb alone \\
  Should stand till Time himself be overthrown? \\
      Would it advantage you to dwell therein \\
  Forever as a stain upon a stone? \\
 \\
Joel Huck \end{quote}


\paragraph{WORSHIP, n.}  Homo Creator's testimony to the sound construction and
fine finish of Deus Creatus.  A popular form of abjection, having an
element of pride.

\paragraph{WRATH, n.}  Anger of a superior quality and degree, appropriate to
exalted characters and momentous occasions; as, "the wrath of God,"
"the day of wrath," etc.  Amongst the ancients the wrath of kings was
deemed sacred, for it could usually command the agency of some god for
its fit manifestation, as could also that of a priest.  The Greeks
before Troy were so harried by Apollo that they jumped out of the
frying-pan of the wrath of Cryses into the fire of the wrath of
Achilles, though Agamemnon, the sole offender, was neither fried nor
roasted.  A similar noted immunity was that of David when he incurred
the wrath of Yahveh by numbering his people, seventy thousand of whom
paid the penalty with their lives.  God is now Love, and a director of
the census performs his work without apprehension of disaster.



\section*{X}



\paragraph{X} in our alphabet being a needless letter has an added invincibility
to the attacks of the spelling reformers, and like them, will
doubtless last as long as the language.  X is the sacred symbol of ten
dollars, and in such words as Xmas, Xn, etc., stands for Christ, not,
as is popular supposed, because it represents a cross, but because the
corresponding letter in the Greek alphabet is the initial of his name
-- {\em Xristos}.  If it represented a cross it would stand for St.
Andrew, who "testified" upon one of that shape.  In the algebra of
psychology x stands for Woman's mind.  Words beginning with X are
Grecian and will not be defined in this standard English dictionary.



\section*{Y}



\paragraph{YANKEE, n.}  In Europe, an American.  In the Northern States of our
Union, a New Englander.  In the Southern States the word is unknown.
(See DAMNYANK.)

\paragraph{YEAR, n.}  A period of three hundred and sixty-five disappointments.

\paragraph{YESTERDAY, n.}  The infancy of youth, the youth of manhood, the entire
past of age.
\begin{quote}   But yesterday I should have thought me blest \\
      To stand high-pinnacled upon the peak \\
      Of middle life and look adown the bleak \\
  And unfamiliar foreslope to the West, \\
  Where solemn shadows all the land invest \\
      And stilly voices, half-remembered, speak \\
      Unfinished prophecy, and witch-fires freak \\
  The haunted twilight of the Dark of Rest. \\
  Yea, yesterday my soul was all aflame \\
      To stay the shadow on the dial's face \\
  At manhood's noonmark!  Now, in God His name \\
      I chide aloud the little interspace \\
  Disparting me from Certitude, and fain \\
  Would know the dream and vision ne'er again. \\
 \\
Baruch Arnegriff \end{quote}


\subparagraph{}   It is said that in his last illness the poet Arnegriff was
attended at different times by seven doctors.

\paragraph{YOKE, n.}  An implement, madam, to whose Latin name, {\em jugum}, we owe
one of the most illuminating words in our language -- a word that
defines the matrimonial situation with precision, point and poignancy.
A thousand apologies for withholding it.

\paragraph{YOUTH, n.}  The Period of Possibility, when Archimedes finds a fulcrum,
Cassandra has a following and seven cities compete for the honor of
endowing a living Homer.

\begin{quote}       Youth is the true Saturnian Reign, the Golden Age on earth \\
  again, when figs are grown on thistles, and pigs betailed with \\
  whistles and, wearing silken bristles, live ever in clover, and \\
  cows fly over, delivering milk at every door, and Justice never \\
  is heard to snore, and every assassin is made a ghost and, \\
  howling, is cast into Baltimost! \\
 \\
Polydore Smith \end{quote}




\section*{Z}



\paragraph{ZANY, n.}  A popular character in old Italian plays, who imitated with
ludicrous incompetence the {\em buffone}, or clown, and was therefore the
ape of an ape; for the clown himself imitated the serious characters
of the play.  The zany was progenitor to the specialist in humor, as
we to-day have the unhappiness to know him.  In the zany we see an
example of creation; in the humorist, of transmission.  Another
excellent specimen of the modern zany is the curate, who apes the
rector, who apes the bishop, who apes the archbishop, who apes the
devil.

\paragraph{ZANZIBARI, n.}  An inhabitant of the Sultanate of Zanzibar, off the
eastern coast of Africa.  The Zanzibaris, a warlike people, are best
known in this country through a threatening diplomatic incident that
occurred a few years ago.  The American consul at the capital occupied
a dwelling that faced the sea, with a sandy beach between.  Greatly to
the scandal of this official's family, and against repeated
remonstrances of the official himself, the people of the city
persisted in using the beach for bathing.  One day a woman came down
to the edge of the water and was stooping to remove her attire (a pair
of sandals) when the consul, incensed beyond restraint, fired a charge
of bird-shot into the most conspicuous part of her person.
Unfortunately for the existing {\em entente cordiale} between two great
nations, she was the Sultana.

\paragraph{ZEAL, n.}  A certain nervous disorder afflicting the young and
inexperienced.  A passion that goeth before a sprawl.

\begin{quote}   When Zeal sought Gratitude for his reward \\
  He went away exclaiming:  "O my Lord!" \\
  "What do you want?" the Lord asked, bending down. \\
  "An ointment for my cracked and bleeding crown." \\
 \\
Jum Coople \end{quote}


\paragraph{ZENITH, n.}  The point in the heavens directly overhead to a man
standing or a growing cabbage.  A man in bed or a cabbage in the pot
is not considered as having a zenith, though from this view of the
matter there was once a considerably dissent among the learned, some
holding that the posture of the body was immaterial.  These were
called Horizontalists, their opponents, Verticalists.  The
Horizontalist heresy was finally extinguished by Xanobus, the
philosopher-king of Abara, a zealous Verticalist.  Entering an
assembly of philosophers who were debating the matter, he cast a
severed human head at the feet of his opponents and asked them to
determine its zenith, explaining that its body was hanging by the
heels outside.  Observing that it was the head of their leader, the
Horizontalists hastened to profess themselves converted to whatever
opinion the Crown might be pleased to hold, and Horizontalism took its
place among {\em fides defuncti}.

\paragraph{ZEUS, n.}  The chief of Grecian gods, adored by the Romans as Jupiter
and by the modern Americans as God, Gold, Mob and Dog.  Some explorers
who have touched upon the shores of America, and one who professes to
have penetrated a considerable distance to the interior, have thought
that these four names stand for as many distinct deities, but in his
monumental work on Surviving Faiths, Frumpp insists that the natives
are monotheists, each having no other god than himself, whom he
worships under many sacred names.

\paragraph{ZIGZAG, v.t.}  To move forward uncertainly, from side to side, as one
carrying the white man's burden.  (From {\em zed}, {\em z}, and {\em jag}, an
Icelandic word of unknown meaning.)

\begin{quote}   He zedjagged so uncomen wyde \\
  Thet non coude pas on eyder syde; \\
  So, to com saufly thruh, I been \\
  Constreynet for to doodge betwene. \\
 \\
Munwele \end{quote}


\paragraph{ZOOLOGY, n.}  The science and history of the animal kingdom, including
its king, the House Fly ({\em Musca maledicta}).  The father of Zoology
was Aristotle, as is universally conceded, but the name of its mother
has not come down to us.  Two of the science's most illustrious
expounders were Buffon and Oliver Goldsmith, from both of whom we
learn ({\em L'Histoire generale des animaux} and {\em A History of Animated
Nature}) that the domestic cow sheds its horn every two years.
\end{document}
